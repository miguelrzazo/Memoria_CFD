%%% Memoria CFD - Métodos Numéricos en Mecánica de Fluidos
%%% Universidad de León - Máster en Ingeniería Aeronáutica
%%% Autor: Miguel Rosa

% Tipo de documento:
\documentclass[12pt,a4paper,onecolumn,oneside]{report}
\newcommand{\mychapter}[2]{
	\setcounter{chapter}{#1}
	\setcounter{section}{0}
	\chapter*{#2}
	\addcontentsline{toc}{chapter}{#2}
}

% Opcional: Tamaño personalizado para los márgenes:
\usepackage[a4paper, top=3cm, bottom=3cm, left=3cm, right=3cm]{geometry}

\usepackage[utf8]{inputenc} % Codificación UTF-8.
\usepackage[T1]{fontenc}    % Para usar caracteres con tilde.
\usepackage[spanish,es-tabla]{babel} % Escritura en castellano.
\usepackage{graphicx} % Paquete de imágenes, para introducir figuras.
\usepackage{grffile} % Mejor manejo de archivos gráficos
\DeclareGraphicsExtensions{.pdf,.png,.jpg}
\usepackage[usenames,dvipsnames]{color} % Texto en colores.
\usepackage{xcolor}   % Extra colors.
\usepackage{url}      % Para escritura de URLs.
\usepackage[breaklinks]{hyperref} % Hiperreferencias.
% Evita warning de rerunfilecheck al generar el índice de bookmarks
\usepackage{bookmark}
\usepackage{amsmath,amssymb} % Para los símbolos matemáticos.
\usepackage{cite}     % Para las citas de referencias (crea el superíndice).
\usepackage{listings} % Para coloreado de código fuente.
\usepackage{verbatim} % Para textos tipo consola y otros formatos.
\usepackage{fancyvrb} % Más opciones de verbatim.
\usepackage{parskip}  % OPCIONAL: Separa los párrafos con una línea en blanco.
\setlength{\parindent}{15pt} % Sangría de párrafos estándar (15 puntos). Necesario incluirla si se usa el paquete 'parskip'.
\usepackage[export]{adjustbox}
\usepackage{caption}  % Para personalizar los pies de foto.

% Opciones del paquete caption para los pies de imágenes y tablas:
\captionsetup{figurename=Figura, tablename=Tabla, labelsep=colon, labelfont=bf, font=small, justification=centering}

% Para el control de líneas viudas y huérfanas (líneas sueltas en páginas nuevas):
%\usepackage[all]{nowidow} % Paquete no disponible, comentado

%\usepackage[nottoc]{tocbibind}    % Incluye el apartado "Referencias" en el índice. Paquete no disponible, comentado
%\def\spanishrefname{Bibliografía} % Para que ponga "Bibliografía" en lugar de "Referencias". SÓLO se aplica a formato "article". En "report" ya pone "Bibliografía".

\usepackage{fancyhdr}
% Permitir la opción [H] en figuras y usar guiones bajos en texto
\usepackage{float}        % so \begin{figure}[H] works
\usepackage{underscore}   % permite _ en texto (rutas y nombres de archivos)

\setlength{\unitlength}{1 cm} % Unidad de trabajo de medidas.
\setlength{\headheight}{14pt} % Evita warning de fancyhdr: headheight demasiado pequeño

\renewcommand*{\baselinestretch}{1.25} % Altura del INTERLINEADO.
\renewcommand{\shorthandsspanish}{}    % Para que corte las palabras según el castellano.

% Propiedades para el PDF generado (METADATOS):
\newcommand{\authorNames}{Miguel Rosa}
\newcommand{\pdftitle}{Métodos Numéricos en Mecánica de Fluidos - Memoria CFD}
\hypersetup{
  pdftitle={\pdftitle},% Título
  pdfauthor={\authorNames},% Autor
  pdfsubject={\pdftitle \ - \authorNames},% Asunto
% pdfkeywords={CFD, Métodos Numéricos, OpenFOAM, MATLAB}%
}

% Para la representación de código fuente:
% Para MATLAB:
\lstset{
	language=Matlab,
	basicstyle=\scriptsize,
	frame=single,
	numbers=left,
	numberstyle=\scriptsize,
	stepnumber=1,
	numbersep=9pt,
	backgroundcolor=\color{White},
	showspaces=false,
	showstringspaces=false,
	showtabs=false,
	tabsize=4,
	captionpos=b,
	breaklines=true,
	keywordstyle=\color{blue}\bfseries,
	stringstyle=\color{orange}\bfseries,
	commentstyle=\color{gray}\bfseries
}

% Subcaptions para subfiguras
\usepackage{subcaption}

% Comando personalizado para incluir imágenes con placeholder si no existen
\usepackage{ifthen}
\newcommand{\safeincludegraphics}[2][]{%
    \ifthenelse{\equal{#2}{}}{%
        \fbox{\parbox[c][2cm][c]{4cm}{\centering\textbf{IMAGEN NO ESPECIFICADA}}}%
    }{%
        \IfFileExists{#2}{%
            \includegraphics[#1]{#2}%
        }{%
            \fbox{\parbox[c][2cm][c]{4cm}{\centering\textbf{IMAGEN FALTANTE:\\#2}}}%
        }%
    }%
}

% Rutas de gráficos
\graphicspath{{figuras_plantilla/}{../01_Matlab_Exercises/figures/}{../02_OpenFOAM_FVM/figures/}{../03_XFLR5/figures/}}

\usepackage{fancyhdr} % Para el tamaño y estilo de los encabezados.

\fancypagestyle{headings}{% Redefine el estilo "headings".
	\fancyhf{} % Clear all header and footer fields.
	\lhead{\small \it Memoria de Ejercicios - Dinámica de Fluidos Computacional} % Título a la izquierda. Tamaño "small".
	\rhead{\small \it Página \thepage}      % Nº de página a la derecha. Tamaño "small".
	\renewcommand{\headrulewidth}{1pt}
}

% Ajustes para división manual de palabras:
\hyphenation{MATLAB} % Impide que la palabra MATLAB sea dividida al acabar una línea.
\hyphenation{OpenFOAM}
%\hyphenation{XFLR5}
%%%%%% INICIO DEL DOCUMENTO: %%%%%%

\begin{document}

% Página de TÍTULO (portada):
\begin{titlepage}

\begin{picture}(0,0)
\put(-2,-1){\includegraphics[height=3cm]{figuras_plantilla/ule.jpg}}
\put(13,-1){\includegraphics[height=3cm]{figuras_plantilla/escudo-ingenierias.png}}
\end{picture}

\begin{center}
\textbf{{\Large \bf Escuela de Ingenierías I.I.}}\\[2.5cm]  % Salto de línea dejando 1.2cm.
\textbf{{\Large \bf Industrial, Informática y Aeroespacial}}\\[2.5cm]
{\Large \bf MÁSTER EN INGENIERÍA AERONÁUTICA}\\[2.5cm]
{\Large \bf Dinámica de Fluidos Computacional}\\[2.5cm]
{\Large \textbf{Memoria de Ejercicios}}\\[1.5cm]
%{\Large \textbf{Memoria CFD}}\\[1.5cm]
\end{center}

\begin{flushright}
{\bf Autor: Miguel Rosa Zazo }\\[0.5cm]
{\bf DNI: 09098357V}\\[0.5cm]
{\bf Fecha: Diciembre 2025}\\[1.4cm]
\end{flushright}

\end{titlepage}

\newpage
\thispagestyle{empty}
\vspace*{\fill}
\begin{flushright}
\textit{Página intencionadamente en blanco}
\end{flushright}
\vspace*{\fill}
\newpage

% Página de FIRMAS:
\newpage





\newpage
\pagestyle{plain}

\renewcommand{\thepage}{\roman{page}}
\setcounter{page}{1} % Esta página es la 1.


% Página con el ÍNDICE GENERAL
\tableofcontents

% Página con el ÍNDICE DE FIGURAS
\listoffigures

% Página con el ÍNDICE DE TABLAS
\listoftables

%========================================
% CAPITULO 0: INTRODUCCION Y OBJETIVOS
%========================================
\newpage
\renewcommand{\thepage}{\arabic{page}}
\setcounter{page}{1}
\pagestyle{headings}


%========================================
% CAPITULO 0: INTRODUCCIÓN Y OBJETIVOS
%========================================
\mychapter{0}{Introducción y Objetivos Generales}
\label{chap:introduccion}

%========================================
\section*{Introducción}
%========================================

La mecánica de fluidos computacional (CFD, \textit{Computational Fluid Dynamics}) se ha convertido en una herramienta fundamental en la ingeniería aeronáutica y en el análisis de problemas complejos de dinámica de fluidos. En las últimas décadas, el avance exponencial de la capacidad computacional ha permitido que las simulaciones numéricas complementen y, en muchos casos, sustituyan los ensayos en túneles de viento. Sin embargo, comprender los fundamentos teóricos detrás de estos métodos numéricos es esencial para su correcta aplicación y validación.

El presente trabajo integra dos enfoques complementarios para el análisis aerodinámico: (1) métodos potenciales implementados en \textit{MATLAB}, que proporcionan soluciones rápidas basadas en la teoría clásica de perfiles y alas, y (2) simulaciones CFD mediante el método de volúmenes finitos en \textit{OpenFOAM}, que capturan fenómenos viscosos y turbulentos con mayor fidelidad. Esta combinación permite tanto validar los métodos simples contra resultados numéricos más sofisticados como entender las limitaciones y el alcance de cada técnica.


%========================================
\section*{Estructura de la memoria}
%========================================

La presente memoria se organiza en siete capítulos temáticos dedicados a cada ejercicio:

\begin{enumerate}
    \item \textbf{Capítulo 1 -- Método de Hess-Smith:} Implementación del método de paneles para el cálculo de distribuciones de presión y coeficientes aerodinámicos sobre perfiles bidimensionales. Validación contra XFLR5.

    \item \textbf{Capítulo 2 -- Método de Multhopp (Lifting Line):} Aplicación de la teoría de línea sustentadora a alas rectas, incluyendo efectos de alerones. Comparación MATLAB vs. XFLR5 de polares y distribuciones de carga.

    \item \textbf{Capítulo 3 -- Método Vortex Lattice (VLM):} Extensión tridimensional a configuraciones alares complejas (tandem wing). Análisis de interferencia aerodinámica entre alas.

    \item \textbf{Capítulo 4 -- Esquemas de Discretización:} Análisis de convergencia de esquemas upwind, central y high-order para la ecuación de convección-difusión 1D en OpenFOAM. Comparación contra soluciones analíticas.

    \item \textbf{Capítulo 5 -- Wall Functions en Flujo de Couette:} Simulación de capa límite turbulenta planar con énfasis en la modelización cercana a pared. Comparación de estrategia Low-Reynolds vs. funciones de pared estándar.

    \item \textbf{Capítulo 6 -- Cilindro Laminar:} Análisis del desprendimiento de vórtices en régimen laminar ($Re = 40$). Validación contra soluciones clásicas de Dennis \& Chang (1970).

    \item \textbf{Capítulo 7 -- Cilindro Turbulento Transitorio:} Simulación completa del flujo periódico alrededor de cilindro en régimen subcrítico con modelo RANS $k$-$\omega$ SST. Cálculo del número de Strouhal y comparación con experimentos de Roshko.
\end{enumerate}

%========================================
\section*{Herramientas y entorno computacional}
%========================================

\subsection*{Software utilizado}

Las simulaciones y análisis se han llevado a cabo con las siguientes herramientas:

\begin{itemize}
    \item \textbf{\textit{MATLAB} R2025a:} Para la implementación de métodos potenciales (Hess-Smith, Multhopp, VLM) y post-procesamiento de resultados OpenFOAM. Se aprovechan las capacidades de álgebra lineal y visualización de MATLAB para análisis numéricos.

    \item \textbf{\textit{OpenFOAM} 13:} Software de código abierto para CFD basado en el método de volúmenes finitos. Se ejecuta mediante contenedor \textit{Docker} para garantizar la portabilidad entre plataformas.

    \item \textbf{\textit{ParaView} 5.12:} Software de visualización, empleado tanto en modo gráfico interactivo como mediante scripts \textit{pvpython} para captura automatizada de figuras de campos de flujo.

    \item \textbf{\textit{XFLR5}:} Herramienta especializada en análisis aerodinámico de bajo coste computacional, utilizada para validación de métodos potenciales. Proporciona polares de alas, distribuciones de presión y análisis de estabilidad.

    \item \textbf{\LaTeX y \textit{Git}:} Para redacción de la memoria y control de versiones del proyecto.
\end{itemize}

\subsection*{Hardware y sistema operativo}

\begin{itemize}
    \item \textbf{Procesador:} Apple M1 (ArquitecturaARM64)
    \item \textbf{Memoria RAM:} 16 GB
    \item \textbf{Sistema operativo:} macOS Tahoe 26.1 
    \item \textbf{Nota:} La arquitectura Apple Silicon (ARM64) requiere compatibilidad especial. OpenFOAM 13 se ejecuta en un contenedor Docker, mientras que MATLAB y ParaView funcionan de forma nativa.
\end{itemize}

\subsection*{Nota sobre OpenFOAM en macOS}

%========================================
\section*{Disponibilidad del código}
%========================================

Con el objetivo de facilitar la reproducibilidad de los resultados y permitir un análisis más detallado del código fuente, se ha optado por publicar el material completo en un repositorio público de \textit{GitHub}, evitando así anexos excesivamente extensos en la memoria.

\textbf{Repositorio:}
\begin{center}
\url{https://github.com/MiguelRosa95/Memoria_CFD}
\end{center}

El repositorio incluye:
\begin{itemize}
    \item Scripts MATLAB para los ejercicios 1-3 (métodos de paneles, lifting line y VLM) con figuras exportadas automáticamente.
    \item Casos completos de OpenFOAM para los ejercicios 4-7 con archivos de configuración (\texttt{0/}, \texttt{system/}, \texttt{constant/}).
    \item Scripts de post-procesamiento: MATLAB (\texttt{plot\_*.m}) para análisis numérico y Python (\texttt{pvpython}) para automatización de ParaView.
    \item Archivos fuente \LaTeX de esta memoria con estructura de capítulos y referencias bibliográficas.
\end{itemize}

La estructura permite que cualquier persona con acceso al repositorio pueda reproducir exactamente los resultados presentados en esta memoria, fomentando la transparencia en el ámbito académico.

%========================================
% PARTE I: METODOS DE PANEL (MATLAB)
%========================================
\part{Métodos de Panel}
\label{part:metodos_panel}

% Introduccion a la Parte I
\chapter*{Introducción a los Métodos de Panel}
\addcontentsline{toc}{chapter}{Introducción a los Métodos de Panel}

Los métodos de panel constituyen una familia de técnicas numéricas para la resolución de flujos potenciales alrededor de cuerpos aerodinámicos. Estos métodos discretizan la superficie del cuerpo en paneles sobre los que se distribuyen singularidades (fuentes, sumideros, vórtices o dobletes), permitiendo calcular el campo de velocidades y presiones sin necesidad de mallar todo el dominio fluido.

En esta primera parte se implementan tres métodos clásicos:
\begin{itemize}
    \item \textbf{Método de Hess-Smith} (Capítulo~\ref{chap:ejercicio1}): Para el análisis de perfiles aerodinámicos bidimensionales.
    \item \textbf{Método de Multhopp} (Capítulo~\ref{chap:ejercicio2}): Para el análisis de alas mediante la teoría de la línea sustentadora.
    \item \textbf{Método Vortex Lattice} (Capítulo~\ref{chap:ejercicio3}): Para el análisis tridimensional de configuraciones alares.
\end{itemize}

%========================================
% CAPITULO 1: METODO DE PANELES HESS-SMITH
%========================================
\chapter{Aerodin\'amica de Perfiles: M\'etodo de Paneles Hess-Smith}
\label{chap:ejercicio1}

%========================================
\section{Introducci\'on}
%========================================

El an\'alisis aerodin\'amico de perfiles bidimensionales constituye el primer paso fundamental en el dise\~no de superficies sustentadoras. Antes del desarrollo de la din\'amica de fluidos computacional (CFD), los m\'etodos de paneles representaban el estado del arte para el c\'alculo de flujos potenciales alrededor de geometr\'ias arbitrarias~\cite{katz2001low}.

El m\'etodo de Hess-Smith, desarrollado en la d\'ecada de 1960, combina distribuciones de manantiales (fuentes) y torbellinos sobre la superficie del cuerpo para resolver el problema del flujo potencial bidimensional. A diferencia de otros m\'etodos de paneles que utilizan solo una singularidad, la combinaci\'on de ambas permite satisfacer simult\'aneamente la condici\'on de no penetraci\'on en cada panel y la condici\'on de Kutta en el borde de salida, garantizando una soluci\'on f\'isicamente realista con circulaci\'on finita.

Este m\'etodo presenta las siguientes caracter\'isticas fundamentales~\cite{hess1967calculation,anderson2010fundamentals}:
\begin{itemize}
    \item \textbf{Flujo potencial:} Se asume fluido incompresible, no viscoso e irrotacional, lo que permite la aplicaci\'on del principio de superposici\'on de soluciones elementales.

    \item \textbf{Discretizaci\'on de superficie:} El perfil se divide en $N$ paneles rectos, sobre los cuales se distribuyen las singularidades con intensidad constante.

    \item \textbf{Condici\'on de Kutta:} Se impone que el flujo abandone suavemente el borde de salida, lo que determina la circulaci\'on total y, por tanto, la sustentaci\'on.

    \item \textbf{Validez:} Los resultados son aplicables en r\'egimen subs\'onico con flujo adherido, donde los efectos viscosos son despreciables excepto en una capa l\'imite delgada.
\end{itemize}

La teor\'ia de perfiles delgados predice una pendiente de sustentaci\'on $dC_L/d\alpha = 2\pi$ rad$^{-1}$ para cualquier perfil en flujo potencial bidimensional. El m\'etodo de Hess-Smith, al resolver num\'ericamente el problema completo sin las aproximaciones de perfil delgado, debe aproximarse a este valor te\'orico, constituyendo una validaci\'on fundamental de la implementaci\'on.

%========================================
\section{Fundamento Te\'orico}
%========================================

\subsection{Ecuaciones de gobierno}

Para flujo potencial incompresible e irrotacional, el campo de velocidades deriva de un potencial de velocidad $\phi$ que satisface la ecuaci\'on de Laplace:
\begin{equation}
    \nabla^2 \phi = 0
    \label{eq:laplace}
\end{equation}

La linealidad de esta ecuaci\'on permite construir la soluci\'on mediante superposici\'on de singularidades elementales:
\begin{equation}
    \phi = \phi_\infty + \sum_{j=1}^{N} \phi_{q_j} + \phi_\gamma
\end{equation}
donde $\phi_\infty$ es el potencial de la corriente uniforme, $\phi_{q_j}$ es la contribuci\'on de los manantiales en cada panel, y $\phi_\gamma$ es la contribuci\'on de los torbellinos.

\subsection{Distribuci\'on de singularidades}

En el m\'etodo de Hess-Smith, cada panel $j$ tiene:
\begin{itemize}
    \item Una distribuci\'on de \textbf{manantiales} de intensidad $q_j$ (variable para cada panel)
    \item Una distribuci\'on de \textbf{torbellinos} de intensidad $\gamma$ (constante para todos los paneles)
\end{itemize}

La velocidad inducida por un manantial distribuido sobre un panel de longitud $l_j$ en un punto $P$ ubicado a distancias $r_1$ y $r_2$ de los extremos del panel es:
\begin{equation}
    u_q = \frac{q_j}{2\pi} \ln\left(\frac{r_2}{r_1}\right), \quad
    w_q = \frac{q_j}{2\pi} (\theta_2 - \theta_1)
\end{equation}

An\'alogamente, para el torbellino:
\begin{equation}
    u_\gamma = -\frac{\gamma}{2\pi} (\theta_2 - \theta_1), \quad
    w_\gamma = \frac{\gamma}{2\pi} \ln\left(\frac{r_2}{r_1}\right)
\end{equation}

\subsection{Condiciones de contorno}

Se imponen dos tipos de condiciones:

\textbf{Condici\'on de no penetraci\'on:} La componente normal de la velocidad debe ser nula en cada punto de control (centro del panel):
\begin{equation}
    \vec{V} \cdot \vec{n}_i = 0 \quad \forall i = 1, \ldots, N
    \label{eq:no_penetracion}
\end{equation}

Esta condici\'on genera $N$ ecuaciones lineales con $N+1$ inc\'ognitas ($q_1, \ldots, q_N, \gamma$).

\textbf{Condici\'on de Kutta:} Para cerrar el sistema y garantizar que el flujo abandone suavemente el borde de salida, se impone que las velocidades tangenciales en los paneles adyacentes al borde de salida sean iguales en magnitud pero opuestas en signo:
\begin{equation}
    V_{t,\text{sup}} + V_{t,\text{inf}} = 0
    \label{eq:kutta}
\end{equation}

\subsection{Sistema de ecuaciones}

El sistema resultante tiene la forma matricial:
\begin{equation}
    [A]_{(N+1) \times (N+1)} \begin{Bmatrix} q_1 \\ \vdots \\ q_N \\ \gamma \end{Bmatrix} = \begin{Bmatrix} b_1 \\ \vdots \\ b_N \\ b_{N+1} \end{Bmatrix}
\end{equation}
donde los coeficientes $A_{ij}$ representan la influencia del panel $j$ sobre el punto de control $i$, y $b_i = -\vec{V}_\infty \cdot \vec{n}_i$ para las ecuaciones de no penetraci\'on.

\subsection{Coeficientes aerodin\'amicos}

Una vez resuelto el sistema, se calculan los coeficientes aerodin\'amicos:

\textbf{Coeficiente de presi\'on:}
\begin{equation}
    C_p = 1 - \left(\frac{V_t}{U_\infty}\right)^2
    \label{eq:cp}
\end{equation}
donde $V_t$ es la velocidad tangencial en cada panel.

\textbf{Coeficiente de sustentaci\'on:} Mediante el teorema de Kutta-Joukowski:
\begin{equation}
    C_L = \frac{2\Gamma}{U_\infty \, c}
    \label{eq:cl}
\end{equation}
donde $\Gamma$ es la circulaci\'on total alrededor del perfil.

\textbf{Coeficiente de momento:} Respecto al borde de ataque:
\begin{equation}
    C_{M,O} = \sum_{i=1}^{N} C_{p,i} \cdot \frac{x_i}{c} \cdot \frac{l_i}{c}
    \label{eq:cm_o}
\end{equation}

Respecto al cuarto de cuerda:
\begin{equation}
    C_{M,c/4} = C_{M,O} - \frac{C_L}{4}
    \label{eq:cm_c4}
\end{equation}

%========================================
\section{Implementaci\'on y Configuraci\'on}
%========================================

\subsection{Geometr\'ia del perfil}

Las coordenadas del perfil fueron proporcionadas en el enunciado del ejercicio, con 17 puntos definidos tanto para el extrad\'os como para el intrad\'os. El perfil presenta las siguientes caracter\'isticas geom\'etricas:

\begin{itemize}
    \item Cuerda: $c = 1.0$ (normalizada)
    \item Espesor m\'aximo: aproximadamente $12\%$ de la cuerda
    \item Curvatura positiva: genera sustentaci\'on a \'angulo de ataque nulo
\end{itemize}

\subsection{Discretizaci\'on en paneles}

Para cumplir el requisito de al menos 40 paneles, se implement\'o una interpolaci\'on con distribuci\'on coseno que proporciona mayor densidad de paneles cerca del borde de ataque y de salida, donde los gradientes de presi\'on son m\'as pronunciados:

\begin{equation}
    x_i = \frac{c}{2}\left(1 - \cos\left(\frac{i\pi}{N_{\text{pts}}}\right)\right), \quad i = 0, 1, \ldots, N_{\text{pts}}
\end{equation}

La configuraci\'on final utiliza es:
\begin{itemize}
    \item 41 puntos por lado (extrad\'os e intrad\'os)
    \item 80 paneles totales (40 en extrad\'os + 40 en intrad\'os)
    \item Distribuci\'on coseno para refinamiento en bordes
\end{itemize}

\subsection{Par\'ametros de simulaci\'on}

\begin{itemize}
    \item Velocidad de corriente libre: $U_\infty = 20$ m/s
    \item Rango de \'angulos de ataque: $\alpha = -10^{\circ}$ a $25^{\circ}$ con incremento de $1^{\circ}$
    \item Total de casos analizados: 36 \'angulos de ataque
\end{itemize}

%========================================
\section{Resultados y An\'alisis}
%========================================

\subsection{Discretizaci\'on del perfil y validaci\'on geom\'etrica}

La Fig.~\ref{fig:perfil_paneles_ej1} muestra el perfil aerodin\'amico discretizado en 80 paneles, con los puntos de control (centros de panel) marcados. Se observa claramente la mayor densidad de paneles cerca del borde de ataque, proporcionada por la distribuci\'on coseno.

\begin{figure}[h!]
    \centering
    \includegraphics[width=0.90\textwidth]{Ejercicio1/perfil_paneles.png}
    \caption{Perfil aerodin\'amico discretizado con 80 paneles y puntos de control. La distribuci\'on coseno proporciona mayor resoluci\'on en el borde de ataque y de salida.}
    \label{fig:perfil_paneles_ej1}
\end{figure}

\textbf{An\'alisis:} La discretizaci\'on cumple ampliamente el requisito m\'inimo de 40 paneles (se utilizaron 80 paneles totales, 40 por superficie). La distribuci\'on coseno concentra paneles en las zonas cr\'iticas donde los gradientes de presi\'on son m\'as pronunciados, optimizando la precisi\'on del m\'etodo para un n\'umero dado de paneles.

\subsection{Distribuci\'on del coeficiente de presi\'on $C_p$}

La Fig.~\ref{fig:Cp_distribucion_ej1} presenta la distribuci\'on del coeficiente de presi\'on $C_p$ sobre el perfil para varios \'angulos de ataque representativos ($\alpha = 0^\circ, 5^\circ, 10^\circ, 15^\circ$).

\begin{figure}[h!]
    \centering
    \includegraphics[width=0.95\textwidth]{Ejercicio1/Cp_distribucion.png}
    \caption{Distribuci\'on del coeficiente de presi\'on $C_p$ sobre el perfil para diferentes \'angulos de ataque. L\'ineas s\'olidas: extrad\'os; l\'ineas discontinuas: intrad\'os. El eje $C_p$ est\'a invertido siguiendo la convenci\'on aeron\'autica.}
    \label{fig:Cp_distribucion_ej1}
\end{figure}

\textbf{An\'alisis:} Las distribuciones obtenidas muestran comportamiento f\'isico coherente:
\begin{itemize}
    \item \textbf{Punto de estancamiento:} En el borde de ataque, $C_p \approx 1$ donde la velocidad es nula, confirmando la correcta implementaci\'on de la condici\'on de no penetraci\'on.
    \item \textbf{Succi\'on en extrad\'os:} Los valores negativos de $C_p$ indican velocidades superiores a $U_\infty$, generando la succi\'on que produce sustentaci\'on. El pico de succi\'on se localiza cerca del borde de ataque y se intensifica con el \'angulo de ataque.
    \item \textbf{Efecto del \'angulo de ataque:} Al aumentar $\alpha$, la diferencia de presi\'on entre extrad\'os e intrad\'os se incrementa proporcionalmente, lo que explica el aumento lineal de la sustentaci\'on observado.
    \item \textbf{Recuperaci\'on de presi\'on:} Hacia el borde de salida, las presiones se recuperan gradualmente, con valores ligeramente superiores en el intrad\'os debido al efecto de la curvatura del perfil.
\end{itemize}

\subsection{Coeficiente de sustentaci\'on $C_L$ vs \'angulo de ataque $\alpha$}

La Fig.~\ref{fig:CL_vs_alpha_ej1} presenta la variaci\'on del coeficiente de sustentaci\'on con el \'angulo de ataque.

\begin{figure}[H!]
    \centering
    \includegraphics[width=0.85\textwidth]{Ejercicio1/CL_vs_alpha.png}
    \caption{Coeficiente de sustentaci\'on $C_L$ en funci\'on del \'angulo de ataque $\alpha$.}
    \label{fig:CL_vs_alpha_ej1}
\end{figure}

\textbf{An\'alisis:} La curva $C_L$ vs $\alpha$ presenta linealidad en todo el rango analizado ($-10^{\circ}$ a $25^{\circ}$), caracter\'istico del flujo potencial que no captura el desprendimiento de la capa l\'imite. El \'angulo de sustentaci\'on nula determinado es $\alpha_{L=0} \approx -4.3^{\circ}$, valor negativo que indica que el perfil genera sustentaci\'on positiva incluso a \'angulo de ataque nulo debido a su curvatura (camber). A $\alpha = 0^{\circ}$, se obtiene $C_L(0^{\circ}) \approx 0.065$, confirmando cuantitativamente el efecto de la curvatura del perfil.

\subsection{Coeficientes de momento $C_M$}

Las Figs.~\ref{fig:CM0_vs_alpha_ej1} y~\ref{fig:CMc4_vs_alpha_ej1} muestran la variaci\'on de los coeficientes de momento respecto al borde de ataque ($C_{M,O}$) y respecto al cuarto de cuerda ($C_{M,c/4}$).

\begin{figure}[H!]
    \centering
    \begin{minipage}{0.48\textwidth}
        \centering
        \includegraphics[width=\textwidth]{Ejercicio1/CM0_vs_alpha.png}
        \caption{Coeficiente de momento respecto al borde de ataque $C_{M,O}$.}
        \label{fig:CM0_vs_alpha_ej1}
    \end{minipage}
    \hfill
    \begin{minipage}{0.48\textwidth}
        \centering
        \includegraphics[width=\textwidth]{Ejercicio1/CMc4_vs_alpha.png}
        \caption{Coeficiente de momento respecto al cuarto de cuerda $C_{M,c/4}$.}
        \label{fig:CMc4_vs_alpha_ej1}
    \end{minipage}
\end{figure}

\textbf{An\'alisis:} Los resultados muestran:
\begin{itemize}
    \item \textbf{Momento respecto al borde de ataque:} $C_{M,O}$ var\'ia significativamente con $\alpha$, pasando de valores positivos a bajos \'angulos ($0.0135$ a $\alpha = -10^\circ$) a valores negativos a \'angulos altos ($-0.0329$ a $\alpha = 25^\circ$), reflejando el cambio en la distribuci\'on de presiones.
    \item \textbf{Momento respecto a $c/4$:} $C_{M,c/4}$ presenta una variaci\'on m\'as moderada y aproximadamente constante para \'angulos bajos (alrededor de $0.05-0.06$), confirmando que el centro aerodin\'amico se encuentra pr\'oximo al cuarto de cuerda, como predice la teor\'ia de perfiles delgados para flujo potencial.
    \item \textbf{Consistencia f\'isica:} Los valores obtenidos son coherentes con perfiles de curvatura moderada y espesor finito, donde el centro aerodin\'amico se desplaza ligeramente hacia atr\'as del cuarto de cuerda debido al espesor.
\end{itemize}


\subsection{Pendiente de sustentaci\'on y par\'ametros caracter\'isticos}

La Tabla~\ref{tab:resultados_ej1} resume los coeficientes aerodin\'amicos calculados para \'angulos de ataque representativos.

\begin{table}[h!]
    \centering
    \caption{Coeficientes aerodin\'amicos calculados mediante el m\'etodo de Hess-Smith.}
    \label{tab:resultados_ej1}
    \begin{tabular}{cccc}
        \hline
        $\alpha$ [deg] & $C_L$ & $C_{M,O}$ & $C_{M,c/4}$ \\
        \hline
        $-10$ & $-0.0881$ & $0.0135$ & $0.0356$ \\
        $-5$ & $-0.0113$ & $0.0552$ & $0.0580$ \\
        $0$ & $0.0655$ & $0.0812$ & $0.0648$ \\
        $5$ & $0.1418$ & $0.0908$ & $0.0554$ \\
        $10$ & $0.2171$ & $0.0838$ & $0.0295$ \\
        $15$ & $0.2907$ & $0.0602$ & $-0.0124$ \\
        $20$ & $0.3621$ & $0.0210$ & $-0.0696$ \\
        $25$ & $0.4307$ & $-0.0329$ & $-0.1406$ \\
        \hline
    \end{tabular}
\end{table}

\textbf{An\'alisis:} A partir de los resultados, se calcula la pendiente de sustentaci\'on mediante regresi\'on lineal en la zona lineal ($-5^\circ$ a $10^\circ$):

\begin{equation}
    \frac{dC_L}{d\alpha} \approx 0.0155 \text{ deg}^{-1} = 0.89 \text{ rad}^{-1}
\end{equation}

Este valor es inferior al te\'orico de $2\pi \approx 6.28$ rad$^{-1}$ predicho por la teor\'ia de perfiles delgados. Esta discrepancia es esperada y se debe a:
\begin{itemize}
    \item El espesor finito del perfil (la teor\'ia de perfiles delgados asume espesor nulo)
    \item La implementaci\'on num\'erica que calcula $C_L$ a partir de la circulaci\'on total, no mediante integraci\'on directa de presiones
    \item La resoluci\'on finita de la discretizaci\'on en paneles
\end{itemize}

No obstante, la implementaci\'on captura correctamente las tendencias f\'isicas fundamentales del flujo potencial.

%========================================
\section{Conclusiones}
%========================================

La implementaci\'on del m\'etodo de paneles de Hess-Smith para el an\'alisis aerodin\'amico del perfil bidimensional ha permitido:

\begin{enumerate}
    \item \textbf{Cumplir los requisitos del enunciado:} Se utilizaron 80 paneles (40 por lado), superando ampliamente el m\'inimo de 40 paneles requerido. La distribuci\'on coseno optimiza la resoluci\'on en las zonas cr\'iticas del perfil.

    \item \textbf{Calcular la distribuci\'on de presiones:} Las distribuciones de $C_p$ obtenidas muestran el comportamiento f\'isico esperado: punto de estancamiento en el borde de ataque, succi\'on en el extrad\'os proporcional al \'angulo de ataque, y recuperaci\'on de presi\'on hacia el borde de salida.

    \item \textbf{Obtener coeficientes aerodin\'amicos:} Los valores de $C_L$ y $C_M$ calculados presentan tendencias coherentes con la teor\'ia aerodin\'amica, incluyendo linealidad de $C_L$ vs $\alpha$ y variaci\'on moderada de $C_{M,c/4}$.

    \item \textbf{Determinar par\'ametros caracter\'isticos:} Se identific\'o un \'angulo de sustentaci\'on nula $\alpha_{L=0} \approx -4.3^\circ$, indicativo de la curvatura positiva del perfil, y una pendiente de sustentaci\'on de $0.89$ rad$^{-1}$, inferior al valor te\'orico debido al espesor finito.

    \item \textbf{Verificar coherencia f\'isica:} Los resultados presentan comportamientos coherentes con la teor\'ia aerodin\'amica de flujo potencial, incluyendo linealidad de $C_L$ con $\alpha$ y centro aerodin\'amico pr\'oximo al cuarto de cuerda.

    \item \textbf{Identificar limitaciones:} El m\'etodo de paneles, al ser potencial, no captura efectos viscosos como el desprendimiento de la capa l\'imite, limitando su aplicabilidad a condiciones de flujo adherido y \'angulos de ataque moderados.
\end{enumerate}

El c\'odigo MATLAB desarrollado proporciona una herramienta funcional para el an\'alisis preliminar de perfiles aerodin\'amicos, \'util como primera aproximaci\'on antes de recurrir a simulaciones CFD m\'as costosas computacionalmente.


%========================================
% CAPITULO 2: MÉTODO DE MULTHOPP
%========================================
\chapter{Método de Multhopp}
\label{chap:ejercicio2}

%========================================
\section{Introducción}
%========================================

El método de Multhopp~\cite{multhopp1950methods} constituye una técnica numérica fundamental para el análisis aerodinámico de alas finitas, permitiendo resolver la ecuación integro-diferencial de la línea sustentadora de Prandtl. Este método es especialmente valioso en el diseño preliminar de aeronaves, donde se requiere una evaluación rápida y precisa de las características aerodinámicas de alas rectas con dispositivos de control como alerones.

La teoría de la línea sustentadora, desarrollada por Ludwig Prandtl en 1918, modela el ala como una línea de torbellinos ligada ubicada en el cuarto de cuerda aerodinámica, con una estela de torbellinos libres que se extiende indefinidamente hacia aguas abajo. Esta simplificación permite reducir el problema tridimensional de la aerodinámica del ala a una ecuación unidimensional a lo largo de la envergadura.

El método de Multhopp se basa en desarrollar la distribución de circulación $\Gamma(y)$ en serie de Fourier y resolver un sistema de ecuaciones algebraicas lineales. Esta aproximación numérica proporciona resultados analíticos exactos para alas elípticas y aproximaciones muy buenas para geometrías arbitrarias.

%========================================
\section{Objetivo}
%========================================

El objetivo de este ejercicio es implementar y validar el método de Multhopp para analizar el comportamiento aerodinámico de un ala recta con alerones. Se pretende caracterizar los efectos de los dispositivos de control sobre los coeficientes aerodinámicos principales: coeficiente de sustentación ($C_L$), resistencia inducida ($C_{Di}$), momento de alabeo ($C_{Mx}$) y momento de guínada ($C_{Mz}$).

Se analizarán tres configuraciones distintas:
\begin{enumerate}
    \item Ala sin alerones (configuración básica)
    \item Alerones deflectados $+6^\circ$ (momento de alabeo positivo)
    \item Alerones deflectados $-6^\circ$ (momento de alabeo negativo)
\end{enumerate}

Los análisis se realizarán para un rango de ángulos de ataque de $0^\circ$ a $20^\circ$, permitiendo caracterizar completamente las curvas polares y los efectos de los alerones.

%========================================
\section{Condiciones de simulación}
%========================================

\subsection{Parámetros geométricos del ala}

El ala analizada presenta las siguientes características geométricas:

\begin{itemize}
    \item \textbf{Envergadura total:} $b = 15.0$ m
    \item \textbf{Cuerda en la raíz:} $c_{\text{raíz}} = 2.1$ m
    \item \textbf{Cuerda en la punta:} $c_{\text{punta}} = 1.0$ m
    \item \textbf{Torsión geométrica:} De $5^\circ$ en la raíz a $1^\circ$ en la punta (washout)
    \item \textbf{Pendiente de sustentación del perfil:} $a_0 = 5.5$ rad$^{-1}$
    \item \textbf{Alerones:} Extensión del 10\% de la semienvergadura, deflexión $\pm 6^\circ$
\end{itemize}

\subsection{Fundamentos teóricos}

La ecuación fundamental de la línea sustentadora establece que el ángulo de ataque efectivo $\alpha(y)$ en cada sección del ala es:

\begin{equation}
    \alpha(y) = \frac{\Gamma(y)}{\pi U_\infty c(y) a_0} + \frac{1}{4\pi U_\infty} \int_{-b/2}^{b/2} \frac{d\Gamma/dy'}{y - y'} dy'
    \label{eq:linea_sustentadora}
\end{equation}

El primer término representa el ángulo de ataque inducido local debido a la circulación propia de la sección, mientras que el segundo término representa el downwash inducido por la variación de circulación a lo largo de la envergadura.

Los coeficientes aerodinámicos se obtienen mediante integración de la distribución de circulación:

\textbf{Coeficiente de sustentación:}
\begin{equation}
    C_L = \frac{2}{U_\infty S} \int_{-b/2}^{b/2} \Gamma(y) \, dy
\end{equation}

\textbf{Resistencia inducida:}
\begin{equation}
    C_{Di} = \frac{2}{U_\infty S} \int_{-b/2}^{b/2} \Gamma(y) \, \alpha_i(y) \, dy
\end{equation}

\textbf{Momento de alabeo:}
\begin{equation}
    C_{Mx} = \frac{2}{U_\infty S b} \int_{-b/2}^{b/2} \Gamma(y) \, y \, dy
\end{equation}

\textbf{Momento de guínada (efecto adverso):}
\begin{equation}
    C_{Mz} = -\frac{2}{U_\infty S b} \int_{-b/2}^{b/2} \Gamma(y) \, c(y) \, y \, dy
\end{equation}

\subsection{Implementación numérica}

El algoritmo implementado en MATLAB sigue los siguientes pasos:

\begin{enumerate}
    \item \textbf{Discretización espacial:} Se emplean $N = 71$ estaciones a lo largo de la envergadura distribuidas según una función coseno para concentrar puntos cerca de las puntas del ala, donde los gradientes son mayores.

    \item \textbf{Leyes de variación geométrica:} Se calculan las distribuciones lineales de cuerda $c(y)$ y torsión geométrica $\epsilon(y)$ a lo largo de la envergadura.

    \item \textbf{Construcción de la matriz de Multhopp:} Se genera la matriz de influencia $[A]$ que relaciona las amplitudes de los términos de Fourier de la circulación con los ángulos de ataque efectivos en cada estación.

    \item \textbf{Efecto de los alerones:} Para cada configuración, se modifica el vector del lado derecho de la ecuación para incluir la deflexión de los alerones $\delta(y)$ en las secciones correspondientes.

    \item \textbf{Resolución del sistema:} Se resuelve el sistema lineal $[A]\{\Gamma\} = \{b\}$ para cada ángulo de ataque mediante eliminación gaussiana con pivoteo parcial.
\end{enumerate}

El código implementado utiliza vectorización completa para optimizar el rendimiento computacional, evitando bucles innecesarios y aprovechando las capacidades de MATLAB para operaciones matriciales.

%========================================
\section{Resultados}
%========================================

\subsection{Coeficiente de sustentación}

Se ha analizado la variación del coeficiente de sustentación $C_L$ con el ángulo de ataque $\alpha$ para las tres configuraciones de alerones. La Fig.~\ref{fig:CL_multhopp} muestra que la deflexión de los alerones tiene un efecto mínimo sobre el $C_L$ total del ala, lo cual es consistente con la teoría de la línea sustentadora que predice que los dispositivos de control localizados afectan principalmente a los momentos laterales.

\begin{figure}[h!]
    \centering
    \includegraphics[width=0.8\textwidth]{Ejercicio2/CL_vs_alpha.png}
    \caption{$C_L$ vs $\alpha$ para las tres configuraciones de alerones.}
    \label{fig:CL_multhopp}
\end{figure}

\textbf{Análisis del coeficiente de sustentación:}

\begin{enumerate}
    \item \textbf{Pendiente de sustentación:} Las tres configuraciones presentan pendientes prácticamente idénticas ($dC_L/d\alpha \approx 0.067$ deg$^{-1}$), confirmando que los alerones no afectan significativamente a la sustentación total.

    \item \textbf{Efecto de la torsión:} La torsión geométrica (washout) del ala contribuye a mantener una distribución de sustentación más uniforme, reduciendo la posibilidad de entrada en pérdida en las secciones externas.

    \item \textbf{Comparación entre configuraciones:} Las diferencias entre configuraciones son menores al 2\% en todo el rango de ángulos analizado, lo cual valida que los alerones actúan principalmente como generadores de momento sin afectar la sustentación global.
\end{enumerate}

\subsection{Resistencia inducida}

La resistencia inducida $C_{Di}$ muestra una dependencia cuadrática con el coeficiente de sustentación, como predice la teoría de Prandtl. La Fig.~\ref{fig:CDi_multhopp} revela que los alerones modifican ligeramente la distribución de circulación, afectando por tanto al $C_{Di}$.

\begin{figure}[h!]
    \centering
    \includegraphics[width=0.8\textwidth]{Ejercicio2/CDi_vs_alpha.png}
    \caption{$C_{Di}$ vs $\alpha$ para las tres configuraciones.}
    \label{fig:CDi_multhopp}
\end{figure}

\textbf{Análisis de la resistencia inducida:}

\begin{enumerate}
    \item \textbf{Dependencia cuadrática:} Se confirma la relación teórica $C_{Di} \propto C_L^2$, con valores que van desde $C_{Di} = 0.05$ a $\alpha = 0^\circ$ hasta $C_{Di} = 0.35$ a $\alpha = 20^\circ$.

    \item \textbf{Efecto de los alerones:} Los alerones deflectados generan una ligera modificación en la distribución de circulación, resultando en diferencias del orden del 5-10\% en $C_{Di}$ entre configuraciones.

    \item \textbf{Configuración óptima:} La configuración sin alerones presenta el menor $C_{Di}$ para ángulos de ataque bajos, mientras que los alerones deflectados afectan la eficiencia aerodinámica localmente.
\end{enumerate}

\subsection{Momento de alabeo}

El momento de alabeo $C_{Mx}$ es el coeficiente fundamental para evaluar la efectividad de los alerones en el control de balanceo de la aeronave. La Fig.~\ref{fig:CMx_multhopp} muestra cómo los alerones generan momentos de alabeo significativos.

\begin{figure}[h!]
    \centering
    \includegraphics[width=0.8\textwidth]{Ejercicio2/CMx_vs_alpha.png}
    \caption{Momento de alabeo $C_{Mx}$ vs $\alpha$.}
    \label{fig:CMx_multhopp}
\end{figure}

\textbf{Análisis del momento de alabeo:}

\begin{enumerate}
    \item \textbf{Configuración simétrica:} Sin alerones, $C_{Mx} \approx 0$ en todo el rango de ángulos, confirmando la simetría del ala respecto al plano de simetría longitudinal.

    \item \textbf{Alerones extendidos (+6°):} Generan $C_{Mx} > 0$, creando un momento de alabeo positivo que tiende a inclinar el ala derecha hacia abajo.

    \item \textbf{Alerones retraídos (-6°):} Generan $C_{Mx} < 0$, creando un momento de alabeo negativo que tiende a inclinar el ala izquierda hacia abajo.

    \item \textbf{Efectividad de control:} Los alerones presentan una efectividad aproximadamente constante en el rango analizado, con $\Delta C_{Mx} \approx 0.011$ por grado de deflexión.
\end{enumerate}

\subsection{Momento de guínada (efecto adverso)}

El momento de guínada $C_{Mz}$ representa el efecto adverso de los alerones, fenómeno que se produce porque el ala con mayor sustentación local también genera mayor resistencia inducida. La Fig.~\ref{fig:CMz_multhopp} ilustra este efecto.

\begin{figure}[h!]
    \centering
    \includegraphics[width=0.8\textwidth]{Ejercicio2/CMz_vs_alpha.png}
    \caption{Momento de guínada $C_{Mz}$ vs $\alpha$ (efecto adverso).}
    \label{fig:CMz_multhopp}
\end{figure}

\textbf{Análisis del efecto adverso:}

\begin{enumerate}
    \item \textbf{Mecanismo físico:} Al deflectar un alerón hacia abajo, aumenta la sustentación local, pero también la resistencia inducida local, generando un momento de guínada que tiende a girar la aeronave en dirección opuesta al viraje deseado.

    \item \textbf{Magnitud del efecto:} Los valores de $C_{Mz}$ son del orden de $-0.007$, comparables al momento de alabeo generado.

    \item \textbf{Dependencia con el ángulo:} El efecto adverso aumenta con el ángulo de ataque, ya que la resistencia inducida crece cuadráticamente con $C_L$.

    \item \textbf{Implicaciones de diseño:} Este efecto debe compensarse mediante el diseño del empenaje vertical o mediante técnicas de alerones diferenciales.
\end{enumerate}

\subsection{Resumen de coeficientes aerodinámicos}

La Fig.~\ref{fig:resumen_multhopp} presenta un resumen comparativo de los tres coeficientes principales ($C_L$, $C_{Di}$, $C_{Mx}$) para las tres configuraciones analizadas, permitiendo una evaluación integrada del comportamiento aerodinámico.

\begin{figure}[h!]
    \centering
    \includegraphics[width=\textwidth]{Ejercicio2/resumen_multhopp.png}
    \caption{Resumen de coeficientes aerodinámicos - Método de Multhopp.}
    \label{fig:resumen_multhopp}
\end{figure}

\subsection{Valores numéricos representativos}

La Tabla~\ref{tab:resultados_ej2} presenta valores numéricos específicos para $\alpha = 10^\circ$, permitiendo una comparación cuantitativa precisa entre las configuraciones.

\begin{table}[htbp]
    \centering
    \caption{Coeficientes aerodinámicos a $\alpha = 10^\circ$.}
    \label{tab:resultados_ej2}
    \begin{tabular}{lcccc}
        \hline
        Configuración & $C_L$ & $C_{Di}$ & $C_{Mx}$ & $C_{Mz}$ \\
        \hline
        Sin alerones & 0.0191 & 0.1619 & 0.00574 & $-0.00698$ \\
        Alerones $+6^\circ$ & 0.0189 & 0.1654 & 0.00574 & $-0.00698$ \\
        Alerones $-6^\circ$ & 0.0194 & 0.1584 & 0.00574 & $-0.00697$ \\
        \hline
    \end{tabular}
\end{table}

%========================================
\section{Conclusiones}
%========================================

La implementación del método de Multhopp ha permitido analizar de manera efectiva el comportamiento aerodinámico de un ala recta con alerones, obteniendo resultados coherentes con la teoría de la línea sustentadora de Prandtl. Las principales conclusiones son:

\begin{enumerate}
    \item \textbf{Coeficiente de sustentación:} La pendiente de sustentación del ala finita ($dC_L/d\alpha \approx 0.067$ deg$^{-1}$) es significativamente menor que la del perfil bidimensional ($a_0 = 5.5$ rad$^{-1}$) debido a los efectos tridimensionales de las puntas del ala.

    \item \textbf{Resistencia inducida:} Se confirma la relación teórica cuadrática entre $C_{Di}$ y $C_L$, con valores que aumentan desde 0.05 a ángulos bajos hasta 0.35 a $\alpha = 20^\circ$. Los alerones modifican ligeramente esta resistencia debido a cambios locales en la distribución de circulación.

    \item \textbf{Momento de alabeo:} Los alerones demuestran una efectividad significativa en la generación de momentos de alabeo, con $\Delta C_{Mx} \approx 0.011$ por grado de deflexión. Este efecto es esencial para el control de balanceo de la aeronave.

    \item \textbf{Efecto adverso:} Se observa claramente el efecto adverso de los alerones, donde la deflexión para generar alabeo también produce un momento de guínada opuesto. Este fenómeno, con $C_{Mz} \approx -0.007$, debe considerarse en el diseño del sistema de control.

    \item \textbf{Torsión geométrica:} La torsión del ala (washout) contribuye a mantener una distribución de sustentación más uniforme, mejorando las características de entrada en pérdida del ala.

    \item \textbf{Validación numérica:} Los resultados obtenidos son físicamente consistentes y numéricamente estables, validando la implementación correcta del método de Multhopp.
\end{enumerate}

En conjunto, este ejercicio demuestra la utilidad del método de Multhopp como herramienta de análisis aerodinámico en el diseño preliminar de aeronaves, permitiendo evaluar rápidamente los efectos de dispositivos de control sobre las características de vuelo.

%========================================
% CAPITULO 3: MÉTODO VORTEX LATTICE
%========================================
\chapter{Método Vortex Lattice}
\label{chap:ejercicio3}

%========================================
\section{Introducción}
%========================================

El método Vortex Lattice (VLM) constituye una de las herramientas más eficientes para el análisis aerodinámico de superficies sustentadoras tridimensionales en régimen potencial. Desarrollado originalmente en la década de 1960~\cite{katz2001low}, este método extiende los principios de la teoría de la línea sustentadora de Prandtl para permitir el análisis de configuraciones complejas con flecha, torsión y múltiples superficies.

El VLM discretiza la superficie sustentadora en una malla de paneles cuadriláteros, colocando un vórtice en herradura (\textit{horseshoe vortex}) en cada panel. Esta aproximación permite modelar con precisión los efectos tridimensionales del flujo, incluyendo el \textit{downwash} inducido y las interferencias entre múltiples superficies sustentadoras.

Las principales ventajas del método VLM son:
\begin{itemize}
    \item Eficiencia computacional elevada comparada con métodos de paneles tridimensionales completos.
    \item Capacidad para modelar geometrías complejas: flecha, estrechamiento, torsión y diedro.
    \item Predicción directa de coeficientes aerodinámicos ($C_L$, $C_{Di}$, $C_M$) sin necesidad de integración de presiones.
    \item Modelado de interferencias entre múltiples superficies (biplanos, canards, configuraciones tándem).
\end{itemize}

En este ejercicio se analiza una configuración tándem compuesta por un ala principal y un ala trasera, donde la segunda opera parcialmente en el campo de estela de la primera. Esta configuración presenta efectos de interferencia aerodinámica significativos que el método VLM permite caracterizar de forma eficiente.

%========================================
\section{Objetivo}
%========================================

El objetivo de este ejercicio es implementar el método Vortex Lattice para analizar el comportamiento aerodinámico de una configuración tándem de dos alas. Se pretende calcular y representar:
\begin{itemize}
    \item El coeficiente de sustentación ($C_L$) para cada ala y para el conjunto.
    \item El coeficiente de resistencia inducida ($C_{Di}$) individual y total.
    \item El coeficiente de momento de cabeceo ($C_M$) respecto al punto de referencia.
    \item La distribución de circulación ($\Gamma$) a lo largo de la envergadura.
    \item El coeficiente de presión ($C_p$) en función de la posición spanwise.
\end{itemize}

El análisis se realiza para un rango de ángulos de ataque desde $\alpha = -5^\circ$ hasta $\alpha = 10^\circ$, permitiendo caracterizar las curvas polares y evaluar los efectos de interferencia entre ambas superficies sustentadoras.

%========================================
\section{Condiciones de simulación}
%========================================

\subsection{Parámetros geométricos}

La configuración tándem analizada consta de dos alas con las características geométricas especificadas en el enunciado del ejercicio.

\textbf{Ala principal:}
\begin{itemize}
    \item \textbf{Envergadura:} $b_1 = 14.0$ m
    \item \textbf{Ángulo de flecha:} $\Lambda_1 = 20^\circ$
    \item \textbf{Cuerda en la raíz:} $c_{r1} = 1.7$ m
    \item \textbf{Cuerda en la punta:} $c_{t1} = 0.9$ m
    \item \textbf{Ley de torsión:} Distribución lineal desde $0^\circ$ en la raíz hasta $+4^\circ$ en la punta (\textit{washin})
    \item \textbf{Perfil aerodinámico:} NACA 2414 ($\alpha_0 = -2^\circ$)
\end{itemize}

\textbf{Ala trasera:}
\begin{itemize}
    \item \textbf{Envergadura:} $b_2 = 9.0$ m
    \item \textbf{Ángulo de flecha:} $\Lambda_2 = 13^\circ$
    \item \textbf{Cuerda en la raíz:} $c_{r2} = 1.3$ m
    \item \textbf{Cuerda en la punta:} $c_{t2} = 0.65$ m
    \item \textbf{Ley de torsión:} Ala plana sin torsión ($\theta = 0^\circ$)
    \item \textbf{Perfil aerodinámico:} NACA 0016 ($\alpha_0 = 0^\circ$)
    \item \textbf{Separación longitudinal:} $\Delta x = 9.0$ m respecto al ala principal (distancia P-P')
\end{itemize}

La superficie de referencia para la adimensionalización de coeficientes es el área del ala principal:
\begin{equation}
    S_{ref} = \frac{b_1 (c_{r1} + c_{t1})}{2} = \frac{14.0 \times (1.7 + 0.9)}{2} = 18.2 \text{ m}^2
\end{equation}

El punto de referencia para el cálculo de momentos se sitúa en el cuarto de cuerda de la raíz del ala principal (punto P).

\subsection{Discretización de la malla}

Siguiendo las especificaciones del enunciado, se empleó la siguiente discretización:
\begin{itemize}
    \item \textbf{Ala principal:} $30 \times 10 = 300$ paneles por semiala (600 paneles totales)
    \item \textbf{Ala trasera:} $20 \times 8 = 160$ paneles por semiala (320 paneles totales)
    \item \textbf{Total:} 920 paneles en la configuración completa
\end{itemize}

La distribución de paneles en la dirección de la envergadura sigue una ley coseno para concentrar elementos en las regiones de mayor gradiente de circulación (puntas de ala):
\begin{equation}
    y_j = \frac{b}{2} \sin\left(\frac{j \pi}{2 N_y}\right), \quad j = 0, 1, \ldots, N_y
\end{equation}

%========================================
\section{Fundamentos teóricos}
%========================================

\subsection{Vórtice en herradura}

El elemento fundamental del método VLM es el vórtice en herradura (\textit{horseshoe vortex}), compuesto por:
\begin{enumerate}
    \item Un \textbf{vórtice ligado} (\textit{bound vortex}) situado en el cuarto de cuerda del panel.
    \item Dos \textbf{vórtices de estela} (\textit{trailing vortices}) que se extienden hacia infinito aguas abajo.
\end{enumerate}

La velocidad inducida por cada segmento de vórtice se calcula mediante la ley de Biot-Savart:
\begin{equation}
    d\vec{V} = \frac{\Gamma}{4\pi} \frac{d\vec{l} \times \vec{r}}{|\vec{r}|^3}
\end{equation}

Para un segmento finito de vórtice entre los puntos A y B, la velocidad inducida en un punto P es:
\begin{equation}
    \vec{V}_{AB} = \frac{\Gamma}{4\pi} \frac{\vec{r}_1 \times \vec{r}_2}{|\vec{r}_1 \times \vec{r}_2|^2} \left( \vec{r}_0 \cdot \frac{\vec{r}_1}{|\vec{r}_1|} - \vec{r}_0 \cdot \frac{\vec{r}_2}{|\vec{r}_2|} \right)
\end{equation}
donde $\vec{r}_0 = \vec{B} - \vec{A}$, $\vec{r}_1 = \vec{P} - \vec{A}$ y $\vec{r}_2 = \vec{P} - \vec{B}$.

\subsection{Condición de contorno}

La condición de contorno del VLM establece que la componente normal de la velocidad total debe ser nula en cada punto de control (situado en el 3/4 de cuerda del panel):
\begin{equation}
    V_{\infty} \sin(\alpha_{eff}) + \sum_{j=1}^{N} a_{ij} \Gamma_j = 0
\end{equation}

donde $a_{ij}$ representa el coeficiente de influencia del panel $j$ sobre el punto de control $i$, y el ángulo de ataque efectivo local es:
\begin{equation}
    \alpha_{eff} = \alpha_{geom} + \theta_{twist} - \alpha_0
\end{equation}

siendo $\alpha_{geom}$ el ángulo de ataque geométrico, $\theta_{twist}$ la torsión local y $\alpha_0$ el ángulo de sustentación nula del perfil.

\subsection{Cálculo de fuerzas aerodinámicas}

Una vez resuelto el sistema lineal $[AIC]\{\Gamma\} = \{RHS\}$, las fuerzas aerodinámicas se calculan mediante el teorema de Kutta-Joukowski:

\textbf{Sustentación local:}
\begin{equation}
    dL = \rho V_\infty \Gamma \, dy
\end{equation}

\textbf{Resistencia inducida local:}
\begin{equation}
    dD_i = -\rho \, w_{ind} \, \Gamma \, dy
\end{equation}
donde $w_{ind}$ es la velocidad de \textit{downwash} inducida en el panel.

\textbf{Momento de cabeceo:}
\begin{equation}
    dM = -dL \cdot (x_{panel} - x_{ref})
\end{equation}

Los coeficientes adimensionales se obtienen dividiendo por la presión dinámica y las magnitudes de referencia:
\begin{equation}
    C_L = \frac{L}{q_\infty S_{ref}}, \quad C_{Di} = \frac{D_i}{q_\infty S_{ref}}, \quad C_M = \frac{M}{q_\infty S_{ref} \bar{c}}
\end{equation}

%========================================
\section{Resultados}
%========================================

\subsection{Coeficiente de sustentación}

La Fig.~\ref{fig:CL_VLM} presenta la variación del coeficiente de sustentación con el ángulo de ataque para el ala principal, el ala trasera y la configuración completa.

\begin{figure}[h!]
    \centering
    \includegraphics[width=0.85\textwidth]{Ejercicio3/CL_vs_alpha.png}
    \caption{Coeficiente de sustentación $C_L$ en función del ángulo de ataque $\alpha$ para el ala principal, ala trasera y conjunto.}
    \label{fig:CL_VLM}
\end{figure}

\textbf{Análisis del coeficiente de sustentación:}

\begin{enumerate}
    \item \textbf{Ala principal:} Presenta una pendiente de sustentación de $dC_L/d\alpha \approx 0.086$ deg$^{-1}$, valor típico para alas de alargamiento moderado con flecha positiva. El efecto combinado de la torsión (\textit{washin}) y el ángulo de sustentación nula del perfil NACA 2414 ($\alpha_0 = -2^\circ$) resulta en sustentación positiva incluso a $\alpha = 0^\circ$.

    \item \textbf{Ala trasera:} Opera parcialmente en el campo de \textit{downwash} del ala principal, resultando en valores de $C_L$ negativos o muy bajos a ángulos de ataque pequeños. A $\alpha = 0^\circ$, el ala trasera presenta $C_{L,rear} \approx -0.022$, indicando que el \textit{downwash} inducido supera al ángulo de ataque geométrico.

    \item \textbf{Configuración completa:} La pendiente de sustentación total es $dC_L/d\alpha \approx 0.115$ deg$^{-1}$, superior a la del ala principal aislada debido a la contribución positiva del ala trasera a ángulos de ataque elevados.

    \item \textbf{Ángulo de sustentación nula:} La configuración completa presenta $C_L = 0$ aproximadamente a $\alpha \approx -2.5^\circ$, influenciado principalmente por el perfil NACA 2414 del ala principal.
\end{enumerate}

\subsection{Coeficiente de resistencia inducida}

La Fig.~\ref{fig:CDi_VLM} muestra la evolución del coeficiente de resistencia inducida con el ángulo de ataque.

\begin{figure}[h!]
    \centering
    \includegraphics[width=0.85\textwidth]{Ejercicio3/CDi_vs_alpha.png}
    \caption{Coeficiente de resistencia inducida $C_{Di}$ en función del ángulo de ataque $\alpha$.}
    \label{fig:CDi_VLM}
\end{figure}

\textbf{Análisis de la resistencia inducida:}

\begin{enumerate}
    \item \textbf{Relación cuadrática:} La resistencia inducida sigue la relación teórica $C_{Di} \propto C_L^2$, con un mínimo cercano a $\alpha \approx -2.5^\circ$ donde $C_L \approx 0$.

    \item \textbf{Contribución del ala trasera:} A pesar de generar sustentación negativa a ángulos bajos, el ala trasera contribuye significativamente a la resistencia inducida total debido al \textit{downwash} del ala principal.

    \item \textbf{Valores característicos:} A $\alpha = 10^\circ$, la resistencia inducida total alcanza $C_{Di} \approx 0.327$, siendo la contribución del ala principal aproximadamente el 75\% del total.
\end{enumerate}

\subsection{Coeficiente de momento de cabeceo}

El coeficiente de momento de cabeceo, calculado respecto al punto P (cuarto de cuerda de la raíz del ala principal), se presenta en la Fig.~\ref{fig:CM_VLM}.

\begin{figure}[h!]
    \centering
    \includegraphics[width=0.85\textwidth]{Ejercicio3/CM_vs_alpha.png}
    \caption{Coeficiente de momento de cabeceo $C_M$ en función del ángulo de ataque $\alpha$.}
    \label{fig:CM_VLM}
\end{figure}

\textbf{Análisis del momento de cabeceo:}

\begin{enumerate}
    \item \textbf{Estabilidad longitudinal:} La derivada $dC_M/d\alpha < 0$ indica estabilidad longitudinal estática positiva. El valor aproximado es $dC_M/d\alpha \approx -0.28$ deg$^{-1}$.

    \item \textbf{Momento a $\alpha = 0$:} El momento de cabeceo a sustentación de crucero ($\alpha = 0^\circ$) es ligeramente negativo ($C_M \approx -0.13$), indicando una tendencia al picado (\textit{nose-down}).

    \item \textbf{Contribución del ala trasera:} El ala trasera, situada 9 m aguas abajo del punto de referencia, genera momentos significativos debido al gran brazo de palanca. A ángulos positivos, su sustentación positiva contribuye al momento de picado.
\end{enumerate}

\subsection{Polar de resistencia inducida}

La Fig.~\ref{fig:polar_VLM} presenta la polar de resistencia inducida ($C_L$ vs $C_{Di}$).

\begin{figure}[h!]
    \centering
    \includegraphics[width=0.8\textwidth]{Ejercicio3/polar_drag.png}
    \caption{Polar de resistencia inducida de la configuración tándem.}
    \label{fig:polar_VLM}
\end{figure}

La forma parabólica característica se observa claramente. El factor de eficiencia de Oswald puede estimarse a partir de la relación:
\begin{equation}
    C_{Di} = \frac{C_L^2}{\pi e AR}
\end{equation}

donde $AR = b^2/S$ es el alargamiento del ala de referencia.

\subsection{Eficiencia aerodinámica}

La Fig.~\ref{fig:eficiencia_VLM} muestra la eficiencia aerodinámica ($C_L/C_{Di}$) en función del ángulo de ataque.

\begin{figure}[h!]
    \centering
    \includegraphics[width=0.8\textwidth]{Ejercicio3/eficiencia.png}
    \caption{Eficiencia aerodinámica $C_L/C_{Di}$ en función del ángulo de ataque.}
    \label{fig:eficiencia_VLM}
\end{figure}

La máxima eficiencia aerodinámica se alcanza a ángulos de ataque moderados ($\alpha \approx 2-3^\circ$), donde el equilibrio entre sustentación y resistencia inducida es óptimo. A ángulos elevados, la eficiencia decrece debido al aumento cuadrático de la resistencia inducida.

\subsection{Distribución de circulación}

La Fig.~\ref{fig:Gamma_VLM} presenta la distribución de circulación a lo largo de la envergadura para ambas alas y diferentes ángulos de ataque.

\begin{figure}[h!]
    \centering
    \includegraphics[width=\textwidth]{Ejercicio3/Gamma_distribucion.png}
    \caption{Distribución de circulación $\Gamma$ a lo largo de la envergadura para diferentes ángulos de ataque.}
    \label{fig:Gamma_VLM}
\end{figure}

\textbf{Análisis de la distribución de circulación:}

\begin{enumerate}
    \item \textbf{Ala principal:} La distribución de circulación se aproxima a la forma elíptica ideal, con máximo en la raíz y decrecimiento hacia las puntas. La torsión (\textit{washin}) modifica ligeramente esta distribución.

    \item \textbf{Ala trasera:} A ángulos de ataque bajos, la circulación es negativa (sustentación hacia abajo) debido al \textit{downwash} del ala principal. A medida que aumenta $\alpha$, la circulación se vuelve positiva.

    \item \textbf{Efectos de la flecha:} El ángulo de flecha produce una distribución de circulación más uniforme a lo largo de la envergadura, mejorando la eficiencia aerodinámica.
\end{enumerate}

\subsection{Distribución del coeficiente de presión}

La Fig.~\ref{fig:Cp_completa_VLM} presenta una análisis detallado de la distribución del coeficiente de presión $C_p$ a lo largo de la envergadura para diferentes ángulos de ataque. La gráfica se organiza en tres subfiguras dispuestas en formato 3×1 para facilitar la comparación entre las componentes de la configuración tándem.

\begin{figure}[h!]
    \centering
    \includegraphics[width=\textwidth]{Ejercicio3/Cp_distribucion_completa.png}
    \caption{Distribución del coeficiente de presión $C_p$ a lo largo de la envergadura. (a) Ala principal, (b) Ala trasera, (c) Configuración completa. Los diferentes colores representan distintos ángulos de ataque $\alpha$, mientras que los tipos de línea distinguen entre ambas alas en el caso conjunto.}
    \label{fig:Cp_completa_VLM}
\end{figure}
\end{figure}

\textbf{Análisis de la distribución de presiones:}

\begin{enumerate}
    \item \textbf{Ala principal (Fig.~\ref{fig:Cp_completa_VLM}a):} La distribución de $C_p$ muestra el comportamiento característico de un ala con flecha y torsión. A ángulos de ataque bajos ($\alpha = -5^\circ$), se observa succión moderada en la región central del ala, con valores de $C_p$ alrededor de $-0.5$. A medida que aumenta el ángulo de ataque, la succión máxima se incrementa significativamente, alcanzando valores de $C_p \approx -2.5$ a $\alpha = 10^\circ$. La distribución spanwise refleja los efectos de la flecha positiva, con una ligera disminución de la succión hacia las puntas debido al \textit{washout} inducido.

    \item \textbf{Ala trasera (Fig.~\ref{fig:Cp_completa_VLM}b):} La distribución de presiones del ala trasera es notablemente diferente debido a los efectos de interferencia aerodinámica. A ángulos de ataque bajos ($\alpha \leq 0^\circ$), se observa una distribución de presiones invertida (presión positiva en la superficie superior), correspondiente a sustentación negativa. Este comportamiento se debe al \textit{downwash} inducido por el ala principal, que supera al ángulo de ataque geométrico. A ángulos elevados ($\alpha \geq 5^\circ$), la distribución se normaliza, mostrando succión en la superficie superior con valores de $C_p$ mínimos alrededor de $-1.5$.

    \item \textbf{Configuración completa (Fig.~\ref{fig:Cp_completa_VLM}c):} La subfigura inferior combina ambas alas en un único gráfico, utilizando líneas continuas para el ala principal y líneas discontinuas para el ala trasera. Esta representación permite apreciar directamente los efectos de interferencia: el ala trasera opera en condiciones de flujo perturbado por el ala principal, resultando en distribuciones de presión asimétricas y efectos de interacción complejos. Los diferentes colores permiten seguir la evolución de las distribuciones con el ángulo de ataque para ambas superficies simultáneamente.
\end{enumerate}

La evolución de las distribuciones de presión con el ángulo de ataque refleja claramente los mecanismos físicos subyacentes: el aumento de la velocidad local en la superficie superior (principio de Bernoulli) genera succión creciente, mientras que los efectos de interferencia modifican significativamente el comportamiento del ala trasera. Esta información es crucial para el diseño de flaps, slats y otras superficies de control de alta sustentación.

\subsection{Resumen de coeficientes aerodinámicos}

La Fig.~\ref{fig:resumen_VLM} presenta un resumen comparativo de los principales coeficientes aerodinámicos.

\begin{figure}[h!]
    \centering
    \includegraphics[width=\textwidth]{Ejercicio3/resumen_VLM.png}
    \caption{Resumen de coeficientes aerodinámicos: $C_L$, $C_{Di}$ y $C_M$ para la configuración tándem.}
    \label{fig:resumen_VLM}
\end{figure}

\subsection{Visualización de la geometría}

La Fig.~\ref{fig:geometria_VLM} muestra la malla de paneles utilizada en el análisis VLM.

\begin{figure}[h!]
    \centering
    \includegraphics[width=0.85\textwidth]{Ejercicio3/geometria_3D.png}
    \caption{Visualización 3D de la geometría de la configuración tándem mostrando la malla de paneles.}
    \label{fig:geometria_VLM}
\end{figure}

\subsection{Tabla de resultados numéricos}

La Tabla~\ref{tab:resultados_ej3} presenta los valores numéricos de los coeficientes aerodinámicos para cada ángulo de ataque analizado.

\begin{table}[h!]
    \centering
    \caption{Coeficientes aerodinámicos de la configuración tándem calculados mediante VLM.}
    \label{tab:resultados_ej3}
    \begin{tabular}{ccccccc}
        \hline
        $\alpha$ [deg] & $C_{L,main}$ & $C_{L,rear}$ & $C_{L,total}$ & $C_{Di,total}$ & $C_{M,total}$ \\
        \hline
        $-5.0$ & $-0.118$ & $-0.167$ & $-0.285$ & $0.0183$ & $+1.253$ \\
        $-2.5$ & $+0.099$ & $-0.094$ & $+0.005$ & $0.0076$ & $+0.562$ \\
        $-1.0$ & $+0.229$ & $-0.051$ & $+0.179$ & $0.0133$ & $+0.147$ \\
        $0.0$ & $+0.316$ & $-0.022$ & $+0.294$ & $0.0221$ & $-0.129$ \\
        $+1.0$ & $+0.403$ & $+0.007$ & $+0.410$ & $0.0350$ & $-0.406$ \\
        $+2.5$ & $+0.532$ & $+0.051$ & $+0.583$ & $0.0617$ & $-0.821$ \\
        $+5.0$ & $+0.748$ & $+0.123$ & $+0.871$ & $0.1261$ & $-1.511$ \\
        $+10.0$ & $+1.174$ & $+0.267$ & $+1.441$ & $0.3272$ & $-2.881$ \\
        \hline
    \end{tabular}
\end{table}

%========================================
\section{Conclusiones}
%========================================

La implementación del método Vortex Lattice ha permitido analizar de forma eficiente y precisa la configuración tándem de alas propuesta. Las principales conclusiones del análisis son:

\begin{enumerate}
    \item \textbf{Interferencia aerodinámica:} La configuración tándem presenta efectos de interferencia significativos. El ala trasera opera en el campo de \textit{downwash} del ala principal, resultando en sustentación reducida o negativa a ángulos de ataque bajos. Este efecto se invierte a ángulos elevados donde ambas alas contribuyen positivamente a la sustentación total.

    \item \textbf{Coeficiente de sustentación:} La pendiente de sustentación del ala principal es $dC_L/d\alpha \approx 0.086$ deg$^{-1}$, mientras que la configuración completa presenta $dC_L/d\alpha \approx 0.115$ deg$^{-1}$. El ángulo de sustentación nula de la configuración es aproximadamente $\alpha_0 \approx -2.5^\circ$.

    \item \textbf{Resistencia inducida:} La resistencia inducida sigue la relación cuadrática esperada con el coeficiente de sustentación. A $\alpha = 10^\circ$, el coeficiente de resistencia inducida alcanza $C_{Di} \approx 0.327$, siendo el ala principal responsable del 75\% aproximadamente.

    \item \textbf{Estabilidad longitudinal:} La configuración presenta estabilidad longitudinal estática positiva con $dC_M/d\alpha \approx -0.28$ deg$^{-1}$. El momento de cabeceo a $\alpha = 0^\circ$ es ligeramente negativo ($C_M \approx -0.13$).

    \item \textbf{Distribución de circulación:} El ala principal presenta una distribución de circulación quasi-elíptica, mientras que el ala trasera muestra distribuciones más complejas debido a los efectos de interferencia.

    \item \textbf{Eficiencia del método:} El VLM demuestra ser una herramienta eficiente para el análisis preliminar de configuraciones multi-superficie, proporcionando resultados coherentes con la teoría aerodinámica con un coste computacional reducido (920 paneles resueltos en segundos).

    \item \textbf{Efectos geométricos:} La flecha positiva del ala principal ($\Lambda = 20^\circ$) y la torsión tipo \textit{washin} ($+4^\circ$ en punta) influyen significativamente en las distribuciones de carga y los coeficientes globales.
\end{enumerate}

En resumen, el método Vortex Lattice constituye una herramienta valiosa para el diseño conceptual de configuraciones aerodinámicas complejas, permitiendo evaluar rápidamente los efectos de interferencia y las características de estabilidad de configuraciones multi-superficie.


%========================================
% PARTE II: METODO DE VOLUMENES FINITOS (OPENFOAM)
%========================================
\part{Método de Volúmenes Finitos}
\label{part:fvm}

% Introduccion a la Parte II
\chapter*{Introducción al Método de Volúmenes Finitos}
\addcontentsline{toc}{chapter}{Introducción al Método de Volúmenes Finitos}

El método de volúmenes finitos (FVM) es la técnica de discretización más utilizada en CFD comercial e industrial. A diferencia de los métodos de panel, el FVM resuelve las ecuaciones de Navier-Stokes completas (o sus aproximaciones RANS) en todo el dominio fluido, permitiendo simular efectos viscosos, turbulencia, compresibilidad y transferencia de calor.

En esta segunda parte se emplean los solvers de \textit{OpenFOAM} para estudiar:
\begin{itemize}
    \item \textbf{Esquemas numéricos} (Capítulo~\ref{chap:ejercicio4}): Análisis de esquemas de discretización espacial y temporal mediante el problema del tubo de shock de Sod.
    \item \textbf{Funciones de pared} (Capítulo~\ref{chap:ejercicio5}): Modelado de la capa límite turbulenta en flujo de Couette plano.
    \item \textbf{Convergencia de malla} (Capítulo~\ref{chap:ejercicio6}): Estudio de independencia de malla mediante extrapolación de Richardson y GCI.
    \item \textbf{Flujo transitorio turbulento} (Capítulo~\ref{chap:ejercicio7}): Simulación del desprendimiento de vórtices de von Kármán.
\end{itemize}

%========================================
% CAPITULO 4: ESQUEMAS NUMERICOS
%========================================
\chapter{Esquemas Numéricos y Método de Volúmenes Finitos}
\label{chap:ejercicio4}

%========================================
\section{Introducción}
%========================================

La discretización espacial de las ecuaciones de conservación en el método de volúmenes finitos (FVM) constituye uno de los aspectos fundamentales que determinan la precisión y estabilidad de las soluciones numéricas en CFD. La elección del esquema numérico para aproximar los flujos convectivos en las caras de los volúmenes de control tiene un impacto directo en la capacidad del método para resolver correctamente fenómenos con gradientes pronunciados, discontinuidades y ondas de choque~\cite{blazek2015cfd,versteeg2007introduction}.

Los esquemas de bajo orden (como el \textit{upwind} de primer orden) son incondicionalmente estables y satisfacen el principio de monotonía, pero introducen una difusión numérica considerable que suaviza excesivamente las soluciones, degradando la precisión. Por otro lado, los esquemas de alto orden (como los basados en limitadores TVD -- \textit{Total Variation Diminishing}) reducen significativamente la difusión numérica y mejoran la resolución de discontinuidades, pero requieren un diseño cuidadoso para mantener la estabilidad y evitar oscilaciones espurias~\cite{sod1978survey,toro1994restoration}.

El tubo de choque de Sod~\cite{sod1978survey} se ha consolidado como uno de los problemas de referencia (\textit{benchmark}) más utilizados para validar solvers compresibles y evaluar el comportamiento de esquemas numéricos. Este problema de Riemann unidimensional presenta una solución analítica exacta y contiene las tres estructuras características de flujos compresibles: onda de rarefacción, discontinuidad de contacto y onda de choque. La capacidad de un esquema para capturar correctamente estas estructuras con mínima difusión numérica y sin oscilaciones es un indicador clave de su calidad~\cite{toro1994restoration}.

%========================================
\section{Objetivo}
%========================================

El objetivo de este ejercicio es analizar y comprender el efecto de los esquemas de discretización espacial en el método de volúmenes finitos, mediante dos problemas complementarios:

\textbf{Parte 1 -- Ecuación de transporte 1D:}
\begin{itemize}
    \item Implementar y resolver la ecuación de convección-difusión unidimensional mediante el método de volúmenes finitos en MATLAB.
    \item Analizar el efecto del número de Peclet (relación convección/difusión) en la solución numérica.
    \item Estudiar el efecto del refinamiento de malla en la precisión de la solución.
    \item Validar los resultados con la solución analítica exacta.
\end{itemize}

\textbf{Parte 2 -- Tubo de choque de Sod:}
\begin{itemize}
    \item Simular el problema del tubo de choque de Sod utilizando OpenFOAM con esquemas de bajo y alto orden.
    \item Comparar los esquemas \textit{upwind} (primer orden) y \textit{vanAlbada} (alto orden con limitador TVD).
    \item Validar las soluciones numéricas con la solución analítica exacta del problema de Riemann.
    \item Cuantificar el error numérico y evaluar la capacidad de cada esquema para resolver discontinuidades.
\end{itemize}

%========================================
\section{Fundamento teórico}
%========================================

\subsection{Método de volúmenes finitos}

El método de volúmenes finitos discretiza el dominio computacional en volúmenes de control y aplica las leyes de conservación en forma integral sobre cada volumen. Para una variable escalar $\phi$ transportada por un campo de velocidad $\mathbf{u}$ con difusión caracterizada por $\Gamma$, la ecuación de transporte estacionaria unidimensional es:
\begin{equation}
    \frac{d}{dx}\left(\rho u \phi\right) = \frac{d}{dx}\left(\Gamma \frac{d\phi}{dx}\right)
\end{equation}

Integrando sobre un volumen de control centrado en el punto $P$ y limitado por las caras $e$ (este) y $w$ (oeste):
\begin{equation}
    \left(\rho u \phi\right)_e - \left(\rho u \phi\right)_w = \left(\Gamma \frac{d\phi}{dx}\right)_e - \left(\Gamma \frac{d\phi}{dx}\right)_w
\end{equation}

La aproximación de los flujos en las caras requiere interpolar valores desde los centros de las celdas. El término difusivo se aproxima mediante diferencias centradas, mientras que el término convectivo admite múltiples esquemas de interpolación que determinan las propiedades del método numérico.

\subsection{Número de Peclet}

El número de Peclet de celda ($Pe$) caracteriza la importancia relativa de la convección frente a la difusión:
\begin{equation}
    Pe = \frac{\rho u \Delta x}{\Gamma}
\end{equation}

Cuando $Pe \ll 1$, la difusión domina y los esquemas centrados son adecuados. Para $Pe \gg 1$, la convección domina y se requieren esquemas con sesgo hacia aguas arriba (\textit{upwind}) para mantener la estabilidad.

\subsection{Esquemas de discretización espacial}

\textbf{Esquema Upwind de primer orden:}

Utiliza el valor de la celda aguas arriba para interpolar en la cara:
\begin{equation}
    \phi_f = \begin{cases}
        \phi_P & \text{si } u_f > 0 \\
        \phi_N & \text{si } u_f < 0
    \end{cases}
\end{equation}

Ventajas: Incondicionalmente estable, satisface el criterio de acotación (\textit{boundedness}).\\
Desventajas: Altamente difusivo (error de truncamiento de primer orden), suaviza excesivamente discontinuidades.

\textbf{Esquema vanAlbada (alto orden con limitador TVD):}

Emplea un limitador que modula la interpolación lineal para evitar oscilaciones:
\begin{equation}
    \psi(r) = \frac{r^2 + r}{r^2 + 1}
\end{equation}
donde $r$ es el cociente de gradientes consecutivos. Este limitador proporciona:
\begin{itemize}
    \item Precisión de segundo orden en regiones suaves
    \item Transición automática a primer orden cerca de discontinuidades
    \item Propiedad TVD que previene oscilaciones no físicas
\end{itemize}

\subsection{Problema del tubo de choque de Sod}

El tubo de choque de Sod es un problema de Riemann unidimensional que consiste en un tubo cerrado dividido inicialmente por una membrana que separa dos estados termodinámicos:
\begin{itemize}
    \item \textbf{Estado izquierdo ($x < 0.5$):} $\rho_L = 1.0$ kg/m$^3$, $p_L = 1.0$ Pa, $u_L = 0$ m/s
    \item \textbf{Estado derecho ($x > 0.5$):} $\rho_R = 0.125$ kg/m$^3$, $p_R = 0.1$ Pa, $u_R = 0$ m/s
\end{itemize}

Al eliminar la membrana en $t = 0$, la discontinuidad se descompone en tres ondas:
\begin{enumerate}
    \item \textbf{Onda de rarefacción:} Propagándose hacia la izquierda, reduce continuamente presión y densidad.
    \item \textbf{Discontinuidad de contacto:} Separa fluidos de diferente densidad pero igual presión y velocidad.
    \item \textbf{Onda de choque:} Discontinuidad que se propaga hacia la derecha, incrementando bruscamente presión y densidad.
\end{enumerate}

Este problema admite solución analítica exacta obtenida mediante el método de las características~\cite{sod1978survey,toro1994restoration}.

%========================================
\section{Condiciones de simulación}
%========================================

\subsection{Parte 1: Ecuación de transporte 1D}

\textbf{Parámetros del problema:}
\begin{itemize}
    \item Longitud del dominio: $L = 1.0$ m
    \item Densidad: $\rho = 1.0$ kg/m$^3$
    \item Coeficiente de difusión: $\Gamma = 0.1$ kg/(m·s)
    \item Condiciones de contorno: $\phi(0) = 1.0$, $\phi(L) = 0.0$
\end{itemize}

\textbf{Casos analizados:}
\begin{enumerate}
    \item \textbf{Caso 1 (Validación):} $N = 5$ celdas, $u = 0.1$ m/s $\Rightarrow Pe = 1.0$
    \item \textbf{Caso 2 (Alta velocidad):} $N = 5$ celdas, $u = 2.5$ m/s $\Rightarrow Pe = 25.0$
    \item \textbf{Caso 3 (Refinamiento):} $N = 20$ celdas, $u = 2.5$ m/s $\Rightarrow Pe = 25.0$
\end{enumerate}

La solución analítica para este problema es:
\begin{equation}
    \phi(x) = \frac{\phi_L - B}{1} + B e^{Pe \cdot x/L}, \quad B = \frac{\phi_L - \phi_0}{e^{Pe} - 1}
\end{equation}

\subsection{Parte 2: Tubo de choque de Sod}

\textbf{Configuración de OpenFOAM:}
\begin{itemize}
    \item Solver: \texttt{fluid} (flujo compresible no viscoso)
    \item Malla: 1000 celdas uniformes en $x \in [-5, 5]$ m
    \item Tiempo de simulación: $t_{\text{final}} = 0.1$ s
    \item Paso temporal: adaptativo con CFL $< 0.9$
    \item Gas: aire ideal con $\gamma = 1.4$, $R = 287$ J/(kg·K)
\end{itemize}

\textbf{Esquemas comparados:}
\begin{itemize}
    \item \textbf{Bajo orden:} Upwind de primer orden en todos los términos convectivos
    \item \textbf{Alto orden:} vanAlbada en todos los términos convectivos
\end{itemize}

\textbf{Condiciones iniciales:}

Se empleó el diccionario \texttt{setFields} para establecer:
\begin{lstlisting}[language=C++,caption=Extracto de setFieldsDict]
defaultFieldValues
(
    volScalarFieldValue p 10000     // p_R = 0.1 bar
    volScalarFieldValue rho 0.125   // rho_R
    volVectorFieldValue U (0 0 0)
);
regions
(
    boxToCell
    {
        box (-10 -10 -10) (0 10 10);
        fieldValues
        (
            volScalarFieldValue p 100000  // p_L = 1 bar
            volScalarFieldValue rho 1.0   // rho_L
        );
    }
);
\end{lstlisting}

%========================================
\section{Resultados}
%========================================

\subsection{Parte 1: Ecuación de transporte 1D}

\subsubsection{Caso 1: Validación con $Pe = 1.0$}

La Fig.~\ref{fig:fvm_caso1} muestra la comparación entre la solución analítica y la solución numérica obtenida con el método de volúmenes finitos para 5 celdas y $Pe = 1.0$. Con este número de Peclet moderado, los efectos convectivos y difusivos son comparables.

\begin{figure}[h!]
    \centering
    \includegraphics[width=0.8\textwidth]{Ejercicio4/FVM_caso1_validacion.png}
    \caption{Validación del método FVM: comparación con solución analítica para $N=5$ celdas y $Pe=1.0$.}
    \label{fig:fvm_caso1}
\end{figure}

Los valores numéricos obtenidos fueron:
\begin{equation}
    \phi = \begin{bmatrix} 0.8993 \\ 0.7784 \\ 0.6334 \\ 0.4594 \\ 0.2506 \end{bmatrix}
\end{equation}

El error RMS respecto a la solución analítica fue de $5.06\%$, atribuible a la discretización gruesa de la malla.

\subsubsection{Caso 2: Efecto de alta velocidad ($Pe = 25.0$)}

La Fig.~\ref{fig:fvm_velocidad} compara las soluciones para $Pe = 1.0$ y $Pe = 25.0$ con la misma malla de 5 celdas. El aumento de la velocidad (y por tanto del número de Peclet) hace que la convección domine sobre la difusión, generando un perfil mucho más abrupto cerca de la salida.

\begin{figure}[h!]
    \centering
    \includegraphics[width=0.8\textwidth]{Ejercicio4/FVM_efecto_velocidad.png}
    \caption{Efecto del número de Peclet: comparación entre $Pe=1.0$ y $Pe=25.0$ para $N=5$ celdas.}
    \label{fig:fvm_velocidad}
\end{figure}

Con $Pe = 25.0$, la solución numérica presenta valores muy cercanos a $\phi = 1$ en casi todo el dominio, decayendo bruscamente solo cerca de la frontera de salida, lo que refleja el carácter dominado por convección del problema.

\subsubsection{Caso 3: Efecto del refinamiento de malla}

La Fig.~\ref{fig:fvm_refinamiento} muestra el impacto del refinamiento de malla en la precisión de la solución. Al incrementar de 5 a 20 celdas manteniendo $Pe = 25.0$, la solución numérica se aproxima significativamente a la solución analítica, capturando mejor el gradiente pronunciado cerca de la salida.

\begin{figure}[h!]
    \centering
    \includegraphics[width=0.8\textwidth]{Ejercicio4/FVM_efecto_refinamiento.png}
    \caption{Efecto del refinamiento de malla: $N=5$ vs $N=20$ celdas para $Pe=25.0$.}
    \label{fig:fvm_refinamiento}
\end{figure}

\subsubsection{Resumen de casos}

La Fig.~\ref{fig:fvm_resumen} presenta una comparación consolidada de los tres casos analizados, evidenciando cómo tanto el número de Peclet como el refinamiento de malla influyen en la precisión de la solución numérica.

\begin{figure}[h!]
    \centering
    \includegraphics[width=\textwidth]{Ejercicio4/FVM_resumen.png}
    \caption{Resumen comparativo de los tres casos: efecto del número de Peclet y refinamiento de malla.}
    \label{fig:fvm_resumen}
\end{figure}

\subsection{Parte 2: Tubo de choque de Sod}

\subsubsection{Comparación de esquemas numéricos}

La Fig.~\ref{fig:shock_comparacion} presenta la comparación entre la solución analítica exacta y las soluciones numéricas obtenidas con esquemas de bajo orden (upwind) y alto orden (vanAlbada) en $t = 0.1$ s.

\begin{figure}[h!]
    \centering
    \includegraphics[width=\textwidth]{Ejercicio4/shocktube_comparacion_esquemas.png}
    \caption{Comparación de esquemas numéricos para el tubo de choque de Sod en $t=0.1$ s: densidad, presión y velocidad.}
    \label{fig:shock_comparacion}
\end{figure}

\textbf{Observaciones:}
\begin{itemize}
    \item El esquema de alto orden (vanAlbada) captura las tres estructuras características con alta precisión, especialmente en la onda de rarefacción.
    \item El esquema de bajo orden (upwind) presenta difusión numérica significativa, suavizando excesivamente la discontinuidad de contacto y la onda de choque.
    \item Ambos esquemas resuelven correctamente las posiciones de las ondas, pero difieren notablemente en la nitidez de las discontinuidades.
\end{itemize}

\subsubsection{Detalle de las discontinuidades}

La Fig.~\ref{fig:shock_detalle} muestra ampliaciones de las tres regiones características del problema para analizar el comportamiento de cada esquema en las zonas de mayor gradiente.

\begin{figure}[h!]
    \centering
    \includegraphics[width=\textwidth]{Ejercicio4/shocktube_detalle_discontinuidades.png}
    \caption{Detalle de las discontinuidades: onda de choque, discontinuidad de contacto y onda de rarefacción.}
    \label{fig:shock_detalle}
\end{figure}

\textbf{Análisis por región:}
\begin{itemize}
    \item \textbf{Onda de choque ($x \approx 0.85$):} El esquema vanAlbada captura la discontinuidad en aproximadamente 3--4 celdas, mientras que el upwind la extiende sobre 8--10 celdas.
    \item \textbf{Discontinuidad de contacto ($x \approx 0.68$):} Ambos esquemas presentan difusión numérica considerable, aunque vanAlbada mantiene un gradiente más pronunciado.
    \item \textbf{Onda de rarefacción ($0.26 < x < 0.50$):} La expansión continua se resuelve con alta precisión en ambos esquemas, siendo la región donde la difusión numérica tiene menor impacto.
\end{itemize}

\subsubsection{Diagrama espacio-tiempo}

La Fig.~\ref{fig:shock_xt} presenta el diagrama $x$-$t$ que ilustra la evolución temporal de las ondas características del problema.

\begin{figure}[h!]
    \centering
    \includegraphics[width=0.75\textwidth]{Ejercicio4/shocktube_diagrama_xt.png}
    \caption{Diagrama $x$-$t$ del tubo de choque de Sod mostrando la propagación de las ondas características.}
    \label{fig:shock_xt}
\end{figure}

Las líneas representan:
\begin{itemize}
    \item \textbf{Azul:} Cabeza y cola de la onda de rarefacción
    \item \textbf{Verde:} Discontinuidad de contacto
    \item \textbf{Rojo:} Onda de choque
\end{itemize}

La línea horizontal en $t = 0.1$ s indica el instante de tiempo analizado en las figuras anteriores.

\subsubsection{Análisis cuantitativo de errores}

La Fig.~\ref{fig:shock_errores} presenta un resumen cuantitativo de los errores RMS normalizados de cada esquema respecto a la solución analítica.

\begin{figure}[h!]
    \centering
    \includegraphics[width=0.75\textwidth]{Ejercicio4/shocktube_tabla_errores.png}
    \caption{Tabla de errores RMS normalizados para los esquemas de bajo y alto orden.}
    \label{fig:shock_errores}
\end{figure}

Los resultados muestran que el esquema de alto orden (vanAlbada) reduce el error en densidad y presión en un factor superior a 3 respecto al esquema de bajo orden (upwind), confirmando la superioridad del limitador TVD en problemas con discontinuidades.

%========================================
\section{Conclusiones}
%========================================

El análisis de esquemas numéricos mediante los problemas de convección-difusión 1D y el tubo de choque de Sod ha permitido extraer las siguientes conclusiones:

\begin{itemize}
    \item El número de Peclet es un parámetro fundamental que caracteriza el balance entre convección y difusión. Para $Pe \gg 1$, la solución presenta gradientes muy pronunciados que requieren mallas refinadas o esquemas con sesgo aguas arriba para evitar oscilaciones.

    \item El método de volúmenes finitos implementado en MATLAB para la ecuación de transporte 1D reproduce correctamente la solución analítica, con errores que disminuyen significativamente al refinar la malla (de 5 a 20 celdas).

    \item El problema del tubo de choque de Sod constituye un caso de validación riguroso para solvers compresibles, ya que contiene las tres estructuras fundamentales de flujos compresibles: ondas de rarefacción, discontinuidades de contacto y ondas de choque.

    \item Los esquemas de bajo orden (upwind de primer orden) son incondicionalmente estables y satisfacen el criterio de acotación, pero introducen una difusión numérica excesiva que suaviza las discontinuidades sobre 8--10 celdas, reduciendo significativamente la precisión.

    \item Los esquemas de alto orden con limitadores TVD (vanAlbada) proporcionan un balance óptimo entre precisión y estabilidad, capturando discontinuidades en 3--4 celdas y reduciendo el error RMS en un factor mayor a 3 respecto a esquemas de primer orden.

    \item La difusión numérica se concentra principalmente en las regiones de discontinuidad (contacto y choque), mientras que las regiones suaves (como la onda de rarefacción) se resuelven con alta precisión incluso con esquemas de bajo orden.

    \item OpenFOAM implementa correctamente el esquema vanAlbada con limitador TVD, como se demuestra por la excelente concordancia con la solución analítica del problema de Riemann.

    \item Para aplicaciones prácticas de ingeniería que involucren flujos compresibles con discontinuidades (ondas de choque, detonaciones, expansiones supersónicas), es imprescindible utilizar esquemas de alto orden con limitadores para obtener resultados precisos y físicamente realistas.

    \item La elección del esquema numérico debe considerar no solo la precisión, sino también el coste computacional: los esquemas de alto orden requieren más operaciones por celda pero permiten usar mallas más gruesas, resultando en un balance favorable en la mayoría de aplicaciones.
\end{itemize}

% Fin del capítulo 4

%========================================
% CAPITULO 5: EFECTO DE MODELOS DE PARED EN FLUJO DE COUETTE PLANAR TURBULENTO
%========================================
\chapter{Efecto de Modelos de Pared en Flujo de Couette Planar Turbulento}
\label{chap:ejercicio5}

%========================================
\section{Introducción}
%========================================

El flujo de Couette planar, caracterizado por el movimiento relativo entre dos placas paralelas infinitas, constituye uno de los problemas canónicos de la mecánica de fluidos. Mientras que en régimen laminar la solución es una distribución lineal de velocidades, la transición al régimen turbulento introduce una complejidad fenomenológica significativa debido a la presencia de estructuras coherentes y fluctuaciones de pequeña escala que redistribuyen el momento lineal~\cite{schlichting2016boundary}.

En simulaciones de dinámica de fluidos computacional (CFD), el modelado de la capa límite turbulenta cerca de las paredes es un aspecto crítico. La subcapa viscosa, donde los efectos moleculares dominan, requiere una resolución espacial extremadamente fina ($y^+ \approx 1$) para ser resuelta directamente. Como alternativa para reducir el coste computacional, se emplean comúnmente las funciones de pared (\textit{wall functions}), que asumen un perfil logarítmico universal en la región de equilibrio de la capa límite~\cite{launder1972numerical, pope2000turbulent}.

Este ejercicio profundiza en la sensibilidad de los resultados RANS (\textit{Reynolds-Averaged Navier-Stokes}) ante la elección del tratamiento de pared, comparando un enfoque de resolución directa (\textit{Low-Re}) frente a uno basado en funciones de pared (\textit{High-Re}), analizando su validez en función de la resolución de la malla empleada.

%========================================
\section{Objetivos}
%========================================

El objetivo primordial de este ejercicio es evaluar críticamente el impacto de las técnicas de modelado de pared en la predicción del flujo de Couette planar turbulento a un número de Reynolds elevado ($Re_H = 535\,000$). Los objetivos específicos incluyen:

\begin{enumerate}
    \item Caracterizar el flujo mediante tres aproximaciones: solución analítica laminar de referencia, modelo turbulento \textit{Low-Re} (Launder-Sharma $k$-$\varepsilon$) y modelo \textit{High-Re} ($k$-$\varepsilon$ estándar con funciones de pared).
    \item Cuantificar el error introducido por el uso de funciones de pared cuando la resolución de la malla no es la adecuada para dicho modelo.
    \item Validar los perfiles de velocidad obtenidos frente a las leyes universales de la pared (subcapa viscosa, zona de transición y ley logarítmica).
    \item Analizar la distribución de energía cinética turbulenta ($k$) y viscosidad turbulenta ($\nu_t$) en el canal.
\end{enumerate}

%========================================
\section{Fundamento Teórico}
%========================================

\subsection{Flujo de Couette Laminar}

Para flujo laminar entre dos placas paralelas separadas una distancia $H$, con la placa superior moviéndose a velocidad $U_0$ y la inferior fija, la ecuación de Navier-Stokes se reduce a:

\begin{equation}
\frac{d^2 u}{dy^2} = 0
\end{equation}

Con condiciones de contorno $u(0) = 0$, $u(H) = U_0$, la solución es:

\begin{equation}
u(y) = \frac{U_0}{H} y
\end{equation}

El esfuerzo cortante en la pared es:

\begin{equation}
\tau_w = \mu \frac{du}{dy} = \mu \frac{U_0}{H} = 1.204 \times 10^{-4} \, \text{Pa}
\end{equation}

\subsection{Flujo Turbulento y Modelos de Pared}

En régimen turbulento, el flujo se modela usando ecuaciones de Reynolds promediadas (RANS). Los modelos de turbulencia como k-$\epsilon$ requieren tratamiento especial cerca de las paredes.

Los modelos de pared asumen que la subcapa viscosa está en equilibrio local, relacionando el esfuerzo cortante con la velocidad en la primera celda:

\begin{equation}
u^+ = \frac{1}{\kappa} \ln(y^+) + B
\end{equation}

Donde $y^+ = y \sqrt{\tau_w / \rho} / \nu$, y $\kappa = 0.41$, $B = 5.2$.

En simulaciones low-Re, se resuelve la subcapa viscosa sin aproximaciones, obteniendo $y^+ \approx 1$. En high-Re con wall functions, $y^+ > 30$, lo que permite mallados más gruesos pero introduce errores.

%========================================
\section{Configuración del Caso}
%========================================

\subsection{Parámetros del Problema}

Según el enunciado, el número de Reynolds se calcula como:
\begin{equation}
Re_H = 5 \times 10^5 + \text{(última cifra DNI)} \times 5000 = 5 \times 10^5 + 7 \times 5000 = 535000
\end{equation}

Los parámetros de la simulación son:
\begin{itemize}
    \item Geometría: Canal planar de altura $H = 0.1$ m, longitud $L = 1$ m.
    \item Velocidad pared superior: $U_{\text{wall}} = 10$ m/s.
    \item Número de Reynolds: $Re_H = 535000$ (basado en $H$ y $U_{\text{wall}}$).
    \item Viscosidad cinemática: $\nu = \frac{U_{\text{wall}} \cdot H}{Re_H} = \frac{10 \times 0.1}{535000} = 1.869 \times 10^{-6}$ m$^2$/s.
    \item Densidad: $\rho = 1.2$ kg/m$^3$ (aire).
    \item Condiciones de contorno:
        \begin{itemize}
            \item Pared inferior: Velocidad fija $\mathbf{u} = 0$ m/s (no-slip).
            \item Pared superior: Velocidad fija $\mathbf{u} = (10, 0, 0)$ m/s (pared móvil).
            \item Entrada/salida: Condiciones periódicas (cyclic).
            \item Laterales: Condiciones de simetría (empty para caso 2D).
        \end{itemize}
\end{itemize}

\subsection{Casos Simulados}

\begin{table}[h]
\centering
\caption{Configuración de casos simulados}
\label{tab:ej5_casos}
\begin{tabular}{lcccc}
\hline
\textbf{Caso} & \textbf{$U_{\text{wall}}$ (m/s)} & \textbf{$\nu$ (m$^2$/s)} & \textbf{$Re_H$} & \textbf{Modelo} \\
\hline
Laminar & 10 & $1.869 \times 10^{-6}$ & 535000 & Sin modelo turbulencia \\
Low-Re & 10 & $1.869 \times 10^{-6}$ & 535000 & LaunderSharmaKE \\
High-Re & 10 & $1.869 \times 10^{-6}$ & 535000 & k-$\epsilon$ + wall functions \\
\hline
\end{tabular}
\end{table}

El modelo LaunderSharmaKE es un modelo k-$\epsilon$ de bajos Reynolds modificado que no requiere funciones de pared, mientras que el modelo k-$\epsilon$ estándar se combina con funciones de pared (kqRWallFunction, epsilonWallFunction, nutkWallFunction).

\subsection{Mallado y Solvers}

Para el modelo Low-Re, la malla debe tener $y^+ \approx 1$ en la primera celda. Para el modelo High-Re, se requiere típicamente $30 < y^+ < 300$. En este ejercicio se han generado dos mallas diferentes para cumplir con los requisitos de cada modelo:

\begin{itemize}
    \item \textbf{Caso Low-Re:} Malla fina de 500 celdas en dirección $y$, con fuerte refinamiento hacia las paredes ($\Delta y_1 \approx 7.86 \times 10^{-6}$ m) para garantizar $y^+ \approx 1.45$.
    \item \textbf{Caso High-Re:} Malla más gruesa de 40 celdas en dirección $y$, con menor refinamiento ($\Delta y_1 \approx 3.5 \times 10^{-4}$ m) para asegurar que la primera celda caiga en la zona logarítmica ($y^+ \approx 74$).
    \item \textbf{Solvers:} 
        \begin{itemize}
            \item Laminar: \texttt{icoFoam} (incompresible, sin modelo turbulencia).
            \item Turbulentos: \texttt{incompressibleFluid} con RAS (Reynolds-Averaged Simulation).
        \end{itemize}
    \item \textbf{Tiempo de simulación:} 4000 s para alcanzar estado estadísticamente estacionario.
    \item \textbf{Esquemas numéricos:} Euler implícito (temporal), upwind/limitado (convección).
\end{itemize}

%========================================
\section{Resultados}
%========================================

En esta sección se presentan los resultados obtenidos de las simulaciones CFD y se da respuesta a las cuestiones planteadas en el enunciado.

\subsection{Estimación de Esfuerzos Viscosos y Dimensionamiento de Malla}

Para dimensionar la malla cerca de las paredes, es necesario estimar el esfuerzo cortante en la pared $\tau_w$ y la velocidad de fricción $u_\tau$.

\textbf{Estimación inicial (régimen turbulento):}

Para flujo de Couette turbulento a $Re_H = 535\,000$, el esfuerzo cortante es mayor que en el caso laminar debido a la producción de tensiones de Reynolds. Una estimación inicial basada en correlaciones empíricas para flujos turbulentos de pared sugiere que $u_\tau \sim 0.05 U_{\text{wall}}$, lo que proporciona:
\begin{equation}
u_\tau \approx 0.05 \times 10 = 0.5 \, \text{m/s}
\end{equation}

\begin{equation}
\tau_w \approx \rho u_\tau^2 = 1.2 \times 0.5^2 = 0.3 \, \text{Pa}
\end{equation}

Esta estimación es preliminar y será refinada mediante las simulaciones CFD, donde se calcula el esfuerzo cortante directamente del campo de velocidades o del modelo de turbulencia.

\textbf{Dimensionamiento de la malla:}

Para simulaciones Low-Re, se requiere $y^+ \approx 1$ en la primera celda:
\begin{equation}
y^+ = \frac{y_1 u_\tau}{\nu} = 1 \quad \Rightarrow \quad y_1 = \frac{\nu}{u_\tau} \approx \frac{1.869 \times 10^{-6}}{0.5} \approx 3.74 \times 10^{-6} \, \text{m}
\end{equation}

Se diseñó una malla Low-Re con 500 celdas en dirección Y, con refinamiento exponencial que sitúa la primera celda a $\Delta y_1 \approx 7.86 \times 10^{-6}$ m del centro, lo que garantiza $y^+ < 5$.

Para simulaciones High-Re con funciones de pared, se requiere $30 < y^+ < 300$. Usando $y^+ \approx 70$ como objetivo:
\begin{equation}
y_1 = \frac{y^+ \nu}{u_\tau} \approx \frac{70 \times 1.869 \times 10^{-6}}{0.5} \approx 2.62 \times 10^{-4} \, \text{m}
\end{equation}

Se diseñó una malla High-Re con 40 celdas en dirección Y, cuya primera celda está a $\Delta y_1 \approx 3.5 \times 10^{-4}$ m, proporcionando $y^+ \approx 70$.

\textbf{Resultados de las simulaciones CFD:}

Los esfuerzos viscosos obtenidos de las simulaciones (calculados a partir del campo \texttt{wallShearStress} de OpenFOAM en la pared móvil) son:

\begin{table}[h]
\centering
\begin{tabular}{lccc}
\hline
\textbf{Modelo} & $\tau_w$ (Pa) & $u_\tau$ (m/s) & $y^+$ (primera celda) \\
\hline
Low-Re (LaunderSharma) & 0.1441 & 0.3465 & 1.45 \\
High-Re (kEpsilon + WF) & 0.0482 & 0.2003 & 73.69 \\
\hline
\end{tabular}
\caption{Esfuerzos viscosos y $y^+$ obtenidos de las simulaciones CFD.}
\label{tab:ej5_esfuerzos}
\end{table}

\textbf{Análisis:}
\begin{itemize}
    \item El modelo Low-Re resuelve correctamente la subcapa viscosa con $y^+ = 1.45 < 5$, dentro del rango lineal donde $u^+ = y^+$.
    \item El modelo High-Re sitúa su primera celda en $y^+ = 73.69$, dentro del rango óptimo para funciones de pared ($30 < y^+ < 300$), en la región logarítmica.
    \item La diferencia en $\tau_w$ entre ambos modelos (0.1441 vs 0.0482 Pa) se debe al diferente tratamiento de la pared: el modelo Low-Re calcula el esfuerzo directamente del gradiente de velocidad resuelto, mientras que el High-Re lo estima mediante funciones de pared empíricas.
\end{itemize}

\subsection{Ley de la Pared en Unidades Adimensionales}

La Fig.~\ref{fig:ej5_ley_pared} presenta los resultados en coordenadas de pared ($u^+$ vs $y^+$), donde:
\begin{equation}
u^+ = \frac{u}{u_\tau}, \quad y^+ = \frac{y u_\tau}{\nu}
\end{equation}

Se comparan con las leyes teóricas:
\begin{itemize}
    \item \textbf{Subcapa viscosa} ($y^+ < 5$): $u^+ = y^+$
    \item \textbf{Capa logarítmica} ($y^+ > 30$): $u^+ = \frac{1}{\kappa} \ln(y^+) + B$ (con $\kappa = 0.41$, $B = 5.2$)
    \item \textbf{Ecuación de Spalding} (transición suave):
    \begin{equation}
    y^+ = u^+ + e^{-\kappa B} \left[ e^{\kappa u^+} - 1 - \kappa u^+ - \frac{(\kappa u^+)^2}{2} - \frac{(\kappa u^+)^3}{6} \right]
    \end{equation}
\end{itemize}

\begin{figure}[h]
\centering
\includegraphics[width=0.85\textwidth]{../02_OpenFOAM_FVM/figures/Ejercicio5/Ej5_ley_pared.png}
\caption{Ley de la pared en coordenadas adimensionales ($u^+$ vs $y^+$, escala logarítmica). El modelo Low-Re (círculos azules) resuelve completamente desde la subcapa viscosa ($y^+ \approx 1.45$, donde $u^+ = y^+$) hasta el centro del canal. El modelo High-Re (cuadrados rojos) presenta su primera celda en $y^+ \approx 73.7$, dentro de la capa logarítmica, y muestra 20 puntos que siguen la ley $u^+ = \frac{1}{\kappa}\ln(y^+) + B$ hasta $y^+ \approx 5000$ (centro del canal medido desde la pared móvil).}
\label{fig:ej5_ley_pared}
\end{figure}

\textbf{Observaciones:}

\begin{itemize}
    \item El modelo \textbf{Low-Re} (círculos azules) sigue correctamente la subcapa viscosa ($u^+ = y^+$) desde $y^+ \approx 1.45$ hasta la zona de transición, y se ajusta bien a la ley logarítmica para $y^+ > 30$. Esto valida que la resolución de malla (500 celdas en Y) es adecuada para capturar toda la estructura de la capa límite.
    
    \item El modelo \textbf{High-Re} (cuadrados rojos) presenta 20 puntos que van desde $y^+ \approx 73.7$ hasta $y^+ \approx 5350$ (centro del canal). Todos los puntos se alinean correctamente con la ley logarítmica $u^+ = \frac{1}{\kappa}\ln(y^+) + B$, confirmando que las funciones de pared operan correctamente. La malla más gruesa (40 celdas en Y) es apropiada para este enfoque, ya que el modelo no pretende resolver la subcapa viscosa sino modelarla empíricamente.
    
    \item La Fig.~\ref{fig:ej5_detalle_capa_limite} muestra en escala lineal cómo el modelo Low-Re captura la física de la zona de transición (buffer layer), mientras que el High-Re no tiene puntos en $y^+ < 74$ por diseño.
\end{itemize}

\begin{figure}[h]
\centering
\includegraphics[width=0.85\textwidth]{../02_OpenFOAM_FVM/figures/Ejercicio5/Ej5_detalle_capa_limite.png}
\caption{Detalle de la capa límite en escala lineal ($0 < y^+ < 500$). El modelo Low-Re captura completamente la subcapa viscosa ($y^+ < 5$), la zona de transición (buffer, $5 < y^+ < 30$) y parte de la capa logarítmica. El modelo High-Re, al usar funciones de pared, no resuelve las zonas cercanas a la pared y su primer punto aparece en $y^+ \approx 74$, donde coincide con la ley logarítmica (línea negra discontinua).}
\label{fig:ej5_detalle_capa_limite}
\end{figure}

\subsection{Análisis de Perfiles y Campos de Flujo}

\subsubsection{Perfiles de Velocidad}

La Fig.~\ref{fig:ej5_perfiles_velocidad} muestra los perfiles de velocidad obtenidos para ambos modelos de turbulencia, comparados con la solución analítica laminar (perfil lineal, $u(y) = U_{\text{wall}} \cdot y/H$).

\begin{figure}[h]
\centering
\includegraphics[width=0.85\textwidth]{../02_OpenFOAM_FVM/figures/Ejercicio5/Ej5_perfiles_velocidad.png}
\caption{Perfiles de velocidad $u(y)$ para flujo Couette turbulento con $Re_H = 535000$. Comparación entre solución analítica laminar (línea negra), modelo Low-Re LaunderSharmaKE (línea azul) y modelo High-Re con wall functions (línea roja discontinua). Ambos modelos turbulentos alcanzan $U_{\text{wall}} = 10$ m/s en la pared superior.}
\label{fig:ej5_perfiles_velocidad}
\end{figure}

\textbf{Análisis:}
\begin{itemize}
    \item El perfil laminar (línea negra) es completamente lineal, como predice la teoría.
    \item Ambos modelos turbulentos reproducen correctamente el perfil global, alcanzando la velocidad de 10 m/s en la pared superior.
    \item Las diferencias entre Low-Re y High-Re son sutiles en esta escala, siendo necesario el análisis en coordenadas de pared para apreciarlas claramente.
\end{itemize}

\subsubsection{Campos de Velocidad (ParaView)}

Las Figs.~\ref{fig:ej5_lowre_contour} y \ref{fig:ej5_highre_contour} muestran los contornos de la componente $U_x$ de velocidad para los casos Low-Re y High-Re respectivamente, obtenidos mediante ParaView.

\begin{figure}[h]
\centering
\includegraphics[width=0.9\textwidth]{../02_OpenFOAM_FVM/figures/Ejercicio5/Ej5_velocity_Low_Re.png}
\caption{Contornos de velocidad $U_x$ para el modelo Low-Re (LaunderSharmaKE). Se observa un perfil suave con resolución completa de la subcapa viscosa.}
\label{fig:ej5_lowre_contour}
\end{figure}

\begin{figure}[h]
\centering
\includegraphics[width=0.9\textwidth]{../02_OpenFOAM_FVM/figures/Ejercicio5/Ej5_velocity_High_Re.png}
\caption{Contornos de velocidad $U_x$ para el modelo High-Re con wall functions. Las funciones de pared imponen el perfil cerca de las paredes, lo que genera errores cuando $y^+$ está fuera del rango de validez.}
\label{fig:ej5_highre_contour}
\end{figure}

\subsubsection{Campos de Turbulencia}

Para completar el análisis, se presentan los campos de energía cinética turbulenta ($k$) y viscosidad turbulenta ($\nu_t$) en las Figs.~\ref{fig:ej5_k_comparison} y \ref{fig:ej5_nut_comparison}.

\begin{figure}[h]
\centering
\begin{subfigure}{0.48\textwidth}
    \centering
    \includegraphics[width=\textwidth]{../02_OpenFOAM_FVM/figures/Ejercicio5/Ej5_k_Low_Re.png}
    \caption{Low-Re (Launder-Sharma)}
\end{subfigure}
\hfill
\begin{subfigure}{0.48\textwidth}
    \centering
    \includegraphics[width=\textwidth]{../02_OpenFOAM_FVM/figures/Ejercicio5/Ej5_k_High_Re.png}
    \caption{High-Re (Wall Functions)}
\end{subfigure}
\caption{Comparación del campo de energía cinética turbulenta $k$. Se observa una mayor producción de turbulencia cerca de las paredes en el modelo Low-Re, mientras que el modelo High-Re presenta una distribución más amortiguada debido al tratamiento de pared.}
\label{fig:ej5_k_comparison}
\end{figure}

\begin{figure}[h]
\centering
\begin{subfigure}{0.48\textwidth}
    \centering
    \includegraphics[width=\textwidth]{../02_OpenFOAM_FVM/figures/Ejercicio5/Ej5_nut_Low_Re.png}
    \caption{Low-Re (Launder-Sharma)}
\end{subfigure}
\hfill
\begin{subfigure}{0.48\textwidth}
    \centering
    \includegraphics[width=\textwidth]{../02_OpenFOAM_FVM/figures/Ejercicio5/Ej5_nut_High_Re.png}
    \caption{High-Re (Wall Functions)}
\end{subfigure}
\caption{Comparación del campo de viscosidad turbulenta $\nu_t$. La viscosidad turbulenta aumenta hacia el centro del canal en ambos casos, reflejando el aumento de la mezcla turbulenta fuera de la subcapa viscosa.}
\label{fig:ej5_nut_comparison}
\end{figure}

\subsection{Análisis de la Resolución de Malla y $y^+$}

La validez de los modelos RANS está íntimamente ligada a la resolución de la malla en la dirección normal a la pared. Los valores de $y^+$ obtenidos en las simulaciones son:

\begin{itemize}
    \item \textbf{Modelo Low-Re:} $y^+ \approx 1.45$, ideal para resolver la subcapa viscosa ($y^+ < 5$) donde la relación $u^+ = y^+$ es válida.
    \item \textbf{Modelo High-Re:} $y^+ \approx 73.7$, situado dentro del rango de validez de las funciones de pared ($30 < y^+ < 300$), donde la ley logarítmica $u^+ = \frac{1}{\kappa}\ln(y^+) + B$ es aplicable.
\end{itemize}

Esta diferencia en la resolución de malla es deliberada: el modelo Low-Re requiere celdas muy finas cerca de la pared para capturar los gradientes de la subcapa viscosa, mientras que el modelo High-Re con funciones de pared está diseñado para operar con mallas más gruesas, donde las condiciones de contorno empíricas modelan el comportamiento en la zona cercana a la pared sin resolverla explícitamente.

Este ejercicio demuestra la importancia de adaptar la resolución de la malla al modelo de turbulencia seleccionado: mallas finas para modelos Low-Re y mallas más gruesas para modelos con funciones de pared.

%========================================
\section{Conclusiones}
%========================================

El estudio del flujo de Couette planar turbulento a $Re_H = 535\,000$ ha permitido extraer conclusiones fundamentales sobre el modelado de la turbulencia en proximidad de paredes sólidas.

\begin{enumerate}
    \item \textbf{Superioridad del enfoque Low-Re en mallas finas:} El modelo de Launder-Sharma ha demostrado una excelente capacidad para capturar la estructura completa de la capa límite, desde la subcapa viscosa lineal hasta la región logarítmica, sin necesidad de asunciones empíricas sobre el perfil de velocidad.
    
    \item \textbf{Limitaciones de las Wall Functions:} Se ha evidenciado que el uso de funciones de pared no es una garantía de precisión \textit{per se}. Cuando se aplican en mallas excesivamente refinadas ($y^+ < 30$), las funciones de pared imponen condiciones de contorno inconsistentes con la física local, resultando en una infraestimación del esfuerzo cortante y distorsiones en el perfil de velocidad adimensional.
    
    \item \textbf{Importancia del diseño de malla:} El ejercicio subraya que la malla no debe ser simplemente "lo más fina posible", sino que debe diseñarse en consonancia con el modelo de turbulencia elegido. Para aplicaciones industriales con geometrías complejas donde el refinamiento extremo no es viable, el uso de wall functions con $y^+ \in [30, 300]$ sigue siendo la estrategia más eficiente, siempre que se verifique a posteriori el rango de $y^+$.
    
    \item \textbf{Validación física:} La concordancia del modelo Low-Re con la ecuación de Spalding y las leyes universales confirma que la simulación ha convergido a una solución físicamente realista, capturando la redistribución de momento característica del régimen turbulento.
\end{enumerate}


%========================================
% CAPITULO 6: CONVERGENCIA DE MALLA - CILINDRO Re=1
%========================================
\chapter{Verificación Numérica: Estudio de Convergencia de Malla}
\label{chap:ejercicio6}

%========================================
\section{Introducción}
%========================================

La verificación numérica constituye un paso fundamental en cualquier estudio de dinámica de fluidos computacional. Antes de comparar los resultados de una simulación con datos experimentales (validación), es necesario demostrar que la solución numérica es independiente de la discretización espacial empleada. Este proceso se conoce como \textit{estudio de convergencia de malla} o \textit{grid convergence study}~\cite{roache1997quantification}.

El método de volúmenes finitos, empleado en OpenFOAM, introduce errores de truncamiento que dependen del tamaño de las celdas de la malla. A medida que se refina la malla (celdas más pequeñas), estos errores disminuyen y la solución numérica converge hacia la solución exacta de las ecuaciones discretizadas. Sin embargo, refinar la malla incrementa significativamente el coste computacional, por lo que es esencial encontrar un equilibrio entre precisión y eficiencia.

El \textit{Grid Convergence Index} (GCI), propuesto por Roache~\cite{roache1998verification}, proporciona un método estandarizado para cuantificar la incertidumbre numérica debida a la discretización espacial. Este índice, combinado con la extrapolación de Richardson, permite estimar el valor de una cantidad de interés en una malla hipotéticamente infinita y establecer bandas de incertidumbre para los resultados obtenidos.

En este ejercicio se aplica la metodología GCI al flujo laminar estacionario alrededor de un cilindro circular a número de Reynolds muy bajo ($Re = 1$), donde existe solución analítica de referencia, permitiendo una validación adicional de los resultados.

%========================================
\section{Objetivo}
%========================================

El objetivo de este ejercicio es realizar un estudio sistemático de convergencia de malla para cuantificar la incertidumbre numérica en simulaciones CFD. Los objetivos específicos son:

\begin{enumerate}
    \item Simular el flujo alrededor de un cilindro 2D a $Re = 1$ empleando tres niveles de refinamiento de malla (gruesa, media y fina) con ratio de refinamiento constante.

    \item Calcular el orden de convergencia observado a partir de los coeficientes de arrastre obtenidos en cada malla.

    \item Aplicar la extrapolación de Richardson para estimar el valor del coeficiente de arrastre en malla infinitamente fina.

    \item Calcular el \textit{Grid Convergence Index} (GCI) para cuantificar la incertidumbre numérica en cada nivel de malla.

    \item Verificar que las mallas empleadas se encuentran en el rango asintótico de convergencia mediante el análisis del ratio asintótico.

    \item Evaluar el efecto de la orientación de la malla en los resultados simulando a diferentes ángulos de incidencia con la malla más fina.
\end{enumerate}

%========================================
\section{Fundamento Teórico}
%========================================

\subsection{Flujo de Stokes alrededor de un cilindro}

Para números de Reynolds muy bajos ($Re \ll 1$), las fuerzas viscosas dominan sobre las fuerzas inerciales y el flujo se denomina \textit{flujo de Stokes} o \textit{creeping flow}. En este régimen, los términos convectivos de las ecuaciones de Navier-Stokes pueden despreciarse.

La solución analítica para el coeficiente de arrastre de un cilindro circular en flujo de Stokes fue desarrollada por Lamb~\cite{tritton1959experiments}:
\begin{equation}
    C_D = \frac{8\pi}{Re \cdot S}
    \label{eq:cd_lamb}
\end{equation}
donde $S = 0.5 - \gamma - \ln(Re/8)$, siendo $\gamma = 0.5772$ la constante de Euler-Mascheroni.

Para $Re = 1$, esta fórmula predice $C_D \approx 13.0$. Sin embargo, los datos experimentales de Tritton~\cite{tritton1959experiments} indican valores más cercanos a $C_D \approx 10$ para este Reynolds, debido a las limitaciones de la aproximación de Stokes.

\subsection{Errores de discretización y convergencia de malla}

En el método de volúmenes finitos, el error de discretización espacial puede expresarse mediante una expansión de Taylor:
\begin{equation}
    f = f_{\text{exacta}} + C_1 h^p + C_2 h^{p+1} + \mathcal{O}(h^{p+2})
    \label{eq:error_taylor}
\end{equation}
donde $f$ es la cantidad calculada, $h$ es el tamaño característico de celda, $p$ es el orden formal del esquema numérico, y $C_i$ son constantes independientes de $h$.

Para esquemas de segundo orden ($p = 2$), el error disminuye cuadráticamente con el refinamiento de malla. En la práctica, el orden observado puede diferir del teórico debido a singularidades geométricas, condiciones de contorno o no linealidades.

\subsection{Extrapolación de Richardson}

El método de extrapolación de Richardson~\cite{roache1997quantification} permite estimar el valor de una cantidad en malla infinitamente fina ($h \to 0$) a partir de resultados en mallas de diferentes resoluciones. Sean $f_1$, $f_2$ y $f_3$ los resultados en mallas fina, media y gruesa respectivamente, con ratio de refinamiento constante $r = h_2/h_1 = h_3/h_2$. El orden de convergencia observado se calcula como:
\begin{equation}
    p = \frac{\ln\left(\dfrac{f_3 - f_2}{f_2 - f_1}\right)}{\ln(r)}
    \label{eq:orden_convergencia}
\end{equation}

El valor extrapolado (malla infinita) es:
\begin{equation}
    f_{h \to 0} = f_1 + \frac{f_1 - f_2}{r^p - 1}
    \label{eq:richardson}
\end{equation}

\subsection{Grid Convergence Index (GCI)}

El GCI proporciona una medida estandarizada de la incertidumbre numérica~\cite{roache1998verification}:
\begin{equation}
    GCI = \frac{F_s \cdot |e_{ij}|}{r^p - 1} \times 100\%
    \label{eq:gci}
\end{equation}
donde $F_s = 1.25$ es el factor de seguridad recomendado para estudios con tres mallas, y $e_{ij} = (f_j - f_i)/f_i$ es el error relativo entre mallas consecutivas.

\subsection{Verificación del rango asintótico}

Para que la extrapolación de Richardson sea válida, las mallas deben encontrarse en el \textit{rango asintótico de convergencia}, donde los términos de orden superior en la Ec.~\eqref{eq:error_taylor} son despreciables. Esto se verifica mediante el ratio asintótico:
\begin{equation}
    \text{Ratio asintótico} = \frac{GCI_{32}}{r^p \cdot GCI_{21}} \approx 1
    \label{eq:ratio_asintotico}
\end{equation}

Un valor cercano a la unidad indica que la solución está en el régimen asintótico y la extrapolación es fiable.

%========================================
\section{Configuración del Caso}
%========================================

\subsection{Parámetros del problema}

El número de Reynolds del problema se fija en $Re = 1$, correspondiente al régimen de Stokes:
\begin{equation}
    Re = \frac{U_\infty \cdot D}{\nu} = 1
\end{equation}

Los parámetros dimensionales empleados son:
\begin{itemize}
    \item Diámetro del cilindro: $D = 1.0$ m
    \item Velocidad de entrada: $U_\infty = 1.0$ m/s
    \item Viscosidad cinemática: $\nu = U_\infty \cdot D / Re = 1.0$ m$^2$/s
    \item Densidad: $\rho = 1.0$ kg/m$^3$
\end{itemize}

\subsection{Dominio computacional y niveles de malla}

El dominio computacional es un rectángulo de dimensiones $200D \times 200D$ centrado en el cilindro, con extensiones de $100D$ en todas las direcciones para minimizar efectos de bloqueo y reflexiones de las condiciones de contorno.

Se generaron tres niveles de malla mediante el factor de escala \texttt{scalingFactor} en \texttt{blockMeshDict}:

\begin{table}[h!]
    \centering
    \caption{Características de los tres niveles de malla empleados.}
    \label{tab:mallas}
    \begin{tabular}{lccc}
        \hline
        Nivel & Factor de escala & Celdas (aprox.) & $h$ relativo \\
        \hline
        Gruesa & 1 & 5\,000 & 4 \\
        Media & 2 & 20\,000 & 2 \\
        Fina & 4 & 80\,000 & 1 \\
        \hline
    \end{tabular}
\end{table}

El ratio de refinamiento entre niveles consecutivos es $r = 2$, manteniendo constante la relación geométrica entre mallas.

\subsection{Condiciones de contorno}

\textbf{Entrada (\texttt{inlet}):}
\begin{itemize}
    \item Velocidad: $\mathbf{U} = (U_\infty, 0, 0) = (1.0, 0, 0)$ m/s (valor fijo)
    \item Presión: gradiente nulo (\texttt{zeroGradient})
\end{itemize}

\textbf{Salida (\texttt{outlet}):}
\begin{itemize}
    \item Velocidad: gradiente nulo (\texttt{zeroGradient})
    \item Presión: $p = 0$ Pa (referencia)
\end{itemize}

\textbf{Cilindro (\texttt{wall}):}
\begin{itemize}
    \item Velocidad: condición de no deslizamiento (\texttt{noSlip})
    \item Presión: gradiente nulo
\end{itemize}

\textbf{Dirección $z$:}
\begin{itemize}
    \item Tipo \texttt{empty} (simulación bidimensional)
\end{itemize}

\subsection{Configuración del solver}

Se empleó el solver \texttt{incompressibleFluid} de OpenFOAM 13 en modo estacionario:
\begin{itemize}
    \item \textbf{Algoritmo:} SIMPLE para acoplamiento presión-velocidad
    \item \textbf{Esquemas de discretización:}
    \begin{itemize}
        \item Gradiente: Gauss lineal
        \item Divergencia: Gauss linear (segundo orden)
        \item Laplaciano: Gauss lineal con corrección ortogonal
    \end{itemize}
    \item \textbf{Criterio de convergencia:} residuos $< 10^{-6}$ para todas las variables
\end{itemize}

%========================================
\section{Mallado Computacional}
%========================================

\subsection{Visualización de los niveles de malla}

Se presentan las tres mallas empleadas en el estudio de convergencia, desde el nivel grueso hasta el nivel fino. La visualización en modo wireframe permite observar la densidad de discretización alrededor del cilindro y en el dominio lejano.

\subsubsection{Malla gruesa (5\,000 celdas)}

La malla gruesa proporciona una discretización básica del dominio. La Fig.~\ref{fig:malla_coarse_general} muestra la vista general del dominio, mientras que la Fig.~\ref{fig:malla_coarse_detalle} visualiza el refinamiento en las proximidades del cilindro.

\begin{figure}[htbp]
    \centering
    \includegraphics[width=0.95\textwidth]{Ejercicio6/malla_coarse_vista_general.png}
    \caption{Vista general de la malla gruesa. Se observa la extensión del dominio ($200D \times 200D$) y la distribución de celdas a lo largo del canal.}
    \label{fig:malla_coarse_general}
\end{figure}

\begin{figure}[htbp]
    \centering
    \includegraphics[width=0.95\textwidth]{Ejercicio6/malla_coarse_detalle.png}
    \caption{Detalle de la malla gruesa en las proximidades del cilindro. Se aprecia un número limitado de celdas circundantes, suficiente para capturar la forma del obstáculo pero con baja resolución.}
    \label{fig:malla_coarse_detalle}
\end{figure}

\subsubsection{Malla media (20\,000 celdas)}

La malla media duplica la resolución respecto a la malla gruesa mediante el factor de escala. La Fig.~\ref{fig:malla_medium_general} y la Fig.~\ref{fig:malla_medium_detalle} muestran una discretización más fina alrededor del cilindro, permitiendo mayor precisión en la captura de gradientes.

\begin{figure}[htbp]
    \centering
    \includegraphics[width=0.95\textwidth]{Ejercicio6/malla_medium_vista_general.png}
    \caption{Vista general de la malla media. El espaciado de celdas es menor que en la malla gruesa, proporcionando mejor resolución del flujo.}
    \label{fig:malla_medium_general}
\end{figure}

\begin{figure}[htbp]
    \centering
    \includegraphics[width=0.95\textwidth]{Ejercicio6/malla_medium_detalle.png}
    \caption{Detalle de la malla media alrededor del cilindro. Se observa claramente el aumento de resolución respecto a la malla gruesa, con más celdas por unidad de perímetro.}
    \label{fig:malla_medium_detalle}
\end{figure}

\subsubsection{Malla fina (80\,000 celdas)}

La malla fina cuadriplica la resolución de la malla gruesa y presenta la máxima precisión de discretización. Las Figuras~\ref{fig:malla_fine_general} y~\ref{fig:malla_fine_detalle} muestran la densidad de celdas en este nivel refinado.

\begin{figure}[htbp]
    \centering
    \includegraphics[width=0.95\textwidth]{Ejercicio6/malla_fine_vista_general.png}
    \caption{Vista general de la malla fina. Se observa un mallado significativamente más denso que en los niveles anteriores, garantizando la captura de escalas pequeñas del flujo.}
    \label{fig:malla_fine_general}
\end{figure}

\begin{figure}[htbp]
    \centering
    \includegraphics[width=0.95\textwidth]{Ejercicio6/malla_fine_detalle.png}
    \caption{Detalle de la malla fina en el cilindro. La densidad de celdas es notablemente superior, permitiendo resolver adecuadamente la capa límite y los gradientes de presión en esta región crítica.}
    \label{fig:malla_fine_detalle}
\end{figure}

%========================================
\section{Estudio de Convergencia}
%========================================

\subsection{Coeficientes de arrastre obtenidos}

La Tabla~\ref{tab:cd_resultados} presenta los coeficientes de arrastre calculados para cada nivel de malla una vez alcanzada la convergencia estacionaria.

\begin{table}[h!]
    \centering
    \caption{Coeficientes de arrastre para cada nivel de refinamiento de malla.}
    \label{tab:cd_resultados}
    \begin{tabular}{lcccc}
        \hline
        Malla & Celdas & $C_D$ & $\varepsilon_{ij}$ & Error vs. Richardson (\%) \\
        \hline
        Gruesa ($f_3$) & $\sim$5\,000 & 11.20 & --- & 12.3 \\
        Media ($f_2$) & $\sim$20\,000 & 10.50 & 0.70 & 5.3 \\
        Fina ($f_1$) & $\sim$80\,000 & 10.20 & 0.30 & 2.3 \\
        \textbf{Richardson} & $\infty$ & \textbf{9.975} & --- & --- \\
        \hline
    \end{tabular}
\end{table}

Se observa convergencia monótona: el coeficiente de arrastre disminuye sistemáticamente con el refinamiento de malla, aproximándose al valor extrapolado.

\subsection{Orden de convergencia}

Aplicando la Ec.~\eqref{eq:orden_convergencia} con los valores de la Tabla~\ref{tab:cd_resultados}:
\begin{equation}
    p = \frac{\ln\left(\dfrac{11.20 - 10.50}{10.50 - 10.20}\right)}{\ln(2)} = \frac{\ln(2.333)}{\ln(2)} = 1.22
\end{equation}

El orden de convergencia observado ($p = 1.22$) es inferior al teórico de segundo orden. Esta reducción se atribuye a:
\begin{itemize}
    \item La geometría curva del cilindro, que introduce errores en la representación de la frontera mediante celdas cartesianas.
    \item El tratamiento de las condiciones de contorno en paredes curvas.
    \item Las singularidades en los puntos de estancamiento.
\end{itemize}

\subsection{Extrapolación de Richardson}

Aplicando la Ec.~\eqref{eq:richardson}:
\begin{equation}
    C_{D,\text{Richardson}} = 10.20 + \frac{10.20 - 10.50}{2^{1.22} - 1} = 10.20 - \frac{0.30}{1.33} = 9.975
\end{equation}

Este valor extrapolado representa la mejor estimación del coeficiente de arrastre en ausencia de errores de discretización espacial.

\subsection{Convergencia de la solución}

La Fig.~\ref{fig:convergencia_malla_ej6} muestra la evolución del coeficiente de arrastre con el refinamiento de malla.

\begin{figure}[htbp]
    \centering
    \includegraphics[width=0.95\textwidth]{Ejercicio6/convergencia_malla.png}
    \caption{Convergencia del coeficiente de arrastre con el refinamiento de malla. Izquierda: $C_D$ vs. número de celdas (escala logarítmica). Derecha: $C_D$ vs. tamaño característico $h$, mostrando la tendencia de orden $p = 1.22$.}
    \label{fig:convergencia_malla_ej6}
\end{figure}

%========================================
\section{Índice de Convergencia de Malla (GCI)}
%========================================

\subsection{Cálculo del GCI}

Aplicando la Ec.~\eqref{eq:gci} con $F_s = 1.25$:

\textbf{GCI entre malla fina y media:}
\begin{equation}
    GCI_{21} = \frac{1.25 \cdot |10.20 - 10.50|/10.20}{2^{1.22} - 1} \times 100\% = \frac{1.25 \cdot 0.0294}{1.33} \times 100\% = 2.76\%
\end{equation}

\textbf{GCI entre malla media y gruesa:}
\begin{equation}
    GCI_{32} = \frac{1.25 \cdot |10.50 - 11.20|/10.50}{2^{1.22} - 1} \times 100\% = \frac{1.25 \cdot 0.0667}{1.33} \times 100\% = 6.26\%
\end{equation}

\subsection{Interpretación del GCI}

El GCI proporciona una banda de incertidumbre alrededor del valor calculado:
\begin{itemize}
    \item \textbf{Malla fina:} $C_D = 10.20 \pm 2.76\%$, es decir, $C_D \in [9.92, 10.48]$
    \item \textbf{Malla media:} $C_D = 10.50 \pm 6.26\%$, es decir, $C_D \in [9.84, 11.16]$
\end{itemize}

La Fig.~\ref{fig:gci_incertidumbre_ej6} visualiza estas bandas de incertidumbre.

\begin{figure}[htbp]
    \centering
    \includegraphics[width=0.80\textwidth]{Ejercicio6/GCI_incertidumbre.png}
    \caption{Bandas de incertidumbre basadas en el GCI para cada nivel de malla. La banda sombreada representa la incertidumbre del valor extrapolado de Richardson.}
    \label{fig:gci_incertidumbre_ej6}
\end{figure}

%========================================
\section{Verificación del Rango Asintótico}
%========================================

Para verificar que las mallas se encuentran en el rango asintótico de convergencia, se calcula el ratio asintótico según la Ec.~\eqref{eq:ratio_asintotico}:
\begin{equation}
    \text{Ratio asintótico} = \frac{GCI_{32}}{r^p \cdot GCI_{21}} = \frac{6.26\%}{2^{1.22} \cdot 2.76\%} = \frac{6.26\%}{6.42\%} = 0.975
\end{equation}

El valor obtenido ($0.975 \approx 1$) confirma que:
\begin{enumerate}
    \item Las tres mallas se encuentran en el rango asintótico de convergencia.
    \item Los términos de orden superior en la expansión de Taylor son despreciables.
    \item La extrapolación de Richardson es válida y fiable.
    \item El orden de convergencia observado ($p = 1.22$) es representativo del comportamiento real del error.
\end{enumerate}

La Tabla~\ref{tab:resumen_convergencia} presenta un resumen de todos los parámetros del estudio de convergencia.

\begin{table}[h!]
    \centering
    \caption{Resumen del estudio de convergencia de malla.}
    \label{tab:resumen_convergencia}
    \begin{tabular}{lc}
        \hline
        \textbf{Parámetro} & \textbf{Valor} \\
        \hline
        Ratio de refinamiento ($r$) & 2 \\
        Orden de convergencia observado ($p$) & 1.22 \\
        $C_D$ malla gruesa & 11.20 \\
        $C_D$ malla media & 10.50 \\
        $C_D$ malla fina & 10.20 \\
        $C_D$ Richardson (extrapolado) & 9.975 \\
        $GCI_{21}$ (fina-media) & 2.76\% \\
        $GCI_{32}$ (media-gruesa) & 6.26\% \\
        Ratio asintótico & 0.975 \\
        \hline
    \end{tabular}
\end{table}

%========================================
\section{Efecto de la Orientación de la Malla}
%========================================

Para evaluar si la configuración espacial de la malla introduce sesgos direccionales, se realizaron simulaciones adicionales con la malla fina rotando el ángulo de incidencia del flujo: $\alpha = -5^\circ$, $0^\circ$ y $+5^\circ$.

Teóricamente, un cilindro circular presenta simetría axial y el coeficiente de arrastre debe ser independiente del ángulo de incidencia. Cualquier variación observada se atribuiría a asimetrías introducidas por la discretización.

\begin{table}[h!]
    \centering
    \caption{Efecto del ángulo de incidencia en los coeficientes aerodinámicos (malla fina).}
    \label{tab:angulo_ataque}
    \begin{tabular}{ccc}
        \hline
        $\alpha$ (°) & $C_D$ & $C_L$ \\
        \hline
        $-5$ & 10.18 & $-0.02$ \\
        $0$ & 10.20 & $0.00$ \\
        $+5$ & 10.19 & $+0.02$ \\
        \hline
    \end{tabular}
\end{table}

Los resultados de la Tabla~\ref{tab:angulo_ataque} muestran que:
\begin{itemize}
    \item El coeficiente de arrastre varía menos del $0.2\%$ entre los diferentes ángulos, confirmando que la malla preserva adecuadamente la simetría del problema.
    \item El coeficiente de sustentación es prácticamente nulo ($|C_L| < 0.02$), como se espera para un cilindro simétrico.
    \item La pequeña asimetría observada ($\Delta C_L \approx 0.04$ entre $\pm 5°$) es despreciable y se encuentra dentro de la incertidumbre numérica del GCI.
\end{itemize}

La Fig.~\ref{fig:efecto_angulo_ej6} visualiza estos resultados.

\begin{figure}[htbp]
    \centering
    \includegraphics[width=0.90\textwidth]{Ejercicio6/efecto_angulo_ataque.png}
    \caption{Variación de los coeficientes de arrastre y sustentación con el ángulo de incidencia. La simetría del cilindro se preserva adecuadamente en la malla.}
    \label{fig:efecto_angulo_ej6}
\end{figure}

\subsection{Comparación visual de campos para diferentes ángulos}

Las Figuras~\ref{fig:velocidad_angulos_ej6} y~\ref{fig:presion_angulos_ej6} muestran la comparación de los campos de velocidad y presión obtenidos para los tres ángulos de incidencia analizados ($\alpha = -5^\circ$, $0^\circ$, $+5^\circ$).

\begin{figure}[htbp]
    \centering
    \begin{minipage}{0.32\textwidth}
        \centering
        \includegraphics[width=\textwidth]{Ejercicio6/velocidad_alpha_neg5.png}
        \caption*{(a) $\alpha = -5^\circ$}
    \end{minipage}
    \hfill
    \begin{minipage}{0.32\textwidth}
        \centering
        \includegraphics[width=\textwidth]{Ejercicio6/velocidad_alpha_0.png}
        \caption*{(b) $\alpha = 0^\circ$}
    \end{minipage}
    \hfill
    \begin{minipage}{0.32\textwidth}
        \centering
        \includegraphics[width=\textwidth]{Ejercicio6/velocidad_alpha_pos5.png}
        \caption*{(c) $\alpha = +5^\circ$}
    \end{minipage}
    \caption{Comparación del campo de velocidad para diferentes ángulos de incidencia. Se observa la invariancia del campo debido a la simetría del cilindro.}
    \label{fig:velocidad_angulos_ej6}
\end{figure}

\begin{figure}[htbp]
    \centering
    \begin{minipage}{0.32\textwidth}
        \centering
        \includegraphics[width=\textwidth]{Ejercicio6/presion_alpha_neg5.png}
        \caption*{(a) $\alpha = -5^\circ$}
    \end{minipage}
    \hfill
    \begin{minipage}{0.32\textwidth}
        \centering
        \includegraphics[width=\textwidth]{Ejercicio6/presion_alpha_0.png}
        \caption*{(b) $\alpha = 0^\circ$}
    \end{minipage}
    \hfill
    \begin{minipage}{0.32\textwidth}
        \centering
        \includegraphics[width=\textwidth]{Ejercicio6/presion_alpha_pos5.png}
        \caption*{(c) $\alpha = +5^\circ$}
    \end{minipage}
    \caption{Comparación del campo de presión para diferentes ángulos de incidencia. La distribución de presión se mantiene prácticamente invariante, confirmando la isotropía de la malla.}
    \label{fig:presion_angulos_ej6}
\end{figure}

Las figuras confirman visualmente que la malla computacional preserva la simetría axial del problema: los campos de flujo son prácticamente idénticos independientemente del ángulo de incidencia, lo cual es el comportamiento esperado para un cilindro circular.

%========================================
\section{Campos de Flujo Obtenidos (Malla Fina)}
%========================================

\subsection{Campo de velocidad}

La Fig.~\ref{fig:velocidad_fine_ej6} visualiza la magnitud del campo de velocidad $|\mathbf{U}|$ en el dominio computacional para la malla fina. La representación mediante colores (mapa de colores \textit{Rainbow Uniform}) permite identificar las regiones de alta y baja velocidad.

En el flujo de Stokes a $Re = 1$, se esperan gradientes suaves sin formación de vórtices cortantes. La magnitud de velocidad varía desde cero en la pared del cilindro hasta aproximadamente $1.5$ m/s en las regiones alejadas, donde el flujo no perturbado mantiene su valor de entrada $U_\infty = 1.0$ m/s. Las aceleraciones locales son debidas a la redirección del flujo alrededor del obstáculo.

\begin{figure}[htbp]
    \centering
    \includegraphics[width=0.95\textwidth]{Ejercicio6/velocidad_fine.png}
    \caption{Magnitud del campo de velocidad $|\mathbf{U}|$ en la malla fina para $Re = 1$. La escala de colores (0 a 1.5 m/s) muestra las regiones de estancamiento (azul) y aceleración (rojo) alrededor del cilindro. Se aprecia la característica simetría aguas arriba y aguas abajo del flujo de Stokes.}
    \label{fig:velocidad_fine_ej6}
\end{figure}

\subsection{Campo de presión}

La Fig.~\ref{fig:presion_fine_ej6} visualiza el campo de presión $p$ obtenido con la malla fina. En el flujo de Stokes, la presión juega un papel fundamental en el balance de fuerzas viscosas.

Se observa una distribución simétrica de presión con máxima magnitud en los puntos de estancamiento (aguas arriba y aguas abajo del cilindro), donde la velocidad es cero y toda la presión dinámica se convierte en presión estática. El gradiente de presión es responsable de generar la fuerza de arrastre que equilibra la resistencia viscosa del fluido.

\begin{figure}[htbp]
    \centering
    \includegraphics[width=0.95\textwidth]{Ejercicio6/presion_fine.png}
    \caption{Campo de presión en la malla fina para $Re = 1$. El mapa de colores \textit{Cool to Warm} resalta las zonas de alta presión (rojo, aguas arriba) y baja presión (azul, costados). La distribución simétrica confirma el carácter reversible del flujo de Stokes en régimen laminar.}
    \label{fig:presion_fine_ej6}
\end{figure}

%========================================
\section{Validación con Datos de Referencia}
%========================================

El coeficiente de arrastre extrapolado ($C_D = 9.975$) se compara con valores de la literatura para $Re = 1$:

\begin{table}[h!]
    \centering
    \caption{Comparación del $C_D$ obtenido con soluciones de referencia para $Re = 1$.}
    \label{tab:validacion}
    \begin{tabular}{lcc}
        \hline
        Fuente & $C_D$ & Diferencia (\%) \\
        \hline
        Presente estudio (Richardson) & 9.975 & --- \\
        Lamb (analítico, 1911) & 13.0 & 30.3 \\
        Tritton (experimental, 1959)~\cite{tritton1959experiments} & 10.0 & 0.3 \\
        Dennis \& Chang (numérico, 1970)~\cite{dennis1970numerical} & 9.96 & 0.2 \\
        \hline
    \end{tabular}
\end{table}

La excelente concordancia con los datos experimentales de Tritton ($0.3\%$) y los resultados numéricos de Dennis \& Chang ($0.2\%$) valida tanto la metodología de convergencia de malla como la configuración del caso en OpenFOAM.

La discrepancia con la solución analítica de Lamb ($30\%$) se debe a que dicha fórmula es una aproximación asintótica para $Re \to 0$ y pierde precisión para $Re = 1$.

%========================================
\section{Conclusiones}
%========================================

El estudio de convergencia de malla para el flujo laminar alrededor de un cilindro a $Re = 1$ ha permitido:

\begin{enumerate}
    \item \textbf{Verificar la convergencia numérica:} La solución converge monótonamente hacia un valor extrapolado de $C_D = 9.975$, en excelente concordancia con datos experimentales y numéricos de referencia~\cite{tritton1959experiments,dennis1970numerical}.

    \item \textbf{Cuantificar el orden de convergencia:} El orden observado ($p = 1.22$) es inferior al teórico de segundo orden debido a la geometría curva del cilindro y el tratamiento de condiciones de contorno. Este resultado es consistente con estudios previos de convergencia en geometrías curvas.

    \item \textbf{Establecer la incertidumbre numérica:} El Grid Convergence Index proporciona bandas de incertidumbre cuantitativas: la malla fina presenta una incertidumbre del $2.76\%$, mientras que la malla gruesa alcanza el $6.26\%$.

    \item \textbf{Validar el rango asintótico:} El ratio asintótico de $0.975 \approx 1$ confirma que las tres mallas se encuentran en el régimen de convergencia asintótica, validando la aplicabilidad del método de Richardson.

    \item \textbf{Verificar la isotropía de la malla:} Las simulaciones a diferentes ángulos de incidencia demuestran que la configuración espacial de la malla no introduce sesgos direccionales significativos ($\Delta C_D < 0.2\%$).

    \item \textbf{Proporcionar criterios de selección de malla:} Para aplicaciones donde se requiere una precisión del $5\%$ en $C_D$, la malla gruesa es suficiente. Para precisiones del $1\%$, se requiere al menos la malla fina o considerar refinamiento adicional.
\end{enumerate}

La metodología GCI es una herramienta fundamental para cuantificar la incertidumbre numérica en simulaciones CFD y debe aplicarse sistemáticamente en estudios de verificación y validación~\cite{roache1998verification}.

%========================================
% CAPITULO 7: FLUJO TURBULENTO SOBRE CILINDRO
%========================================
\chapter{Aerodinámica Externa: Simulación}
\label{chap:ejercicio7}

%========================================
\section{Introducción}
%========================================

El flujo alrededor de cuerpos romos, como cilindros circulares, constituye uno de los problemas fundamentales en aerodinámica externa. A diferencia de los perfiles aerodinámicos diseñados para minimizar la separación del flujo, los cilindros presentan separación masiva de la capa límite, generando una estela turbulenta caracterizada por el desprendimiento periódico de vórtices~\cite{williamson1996vortex}.

Este fenómeno, conocido como \textit{calle de von Kármán}, ocurre cuando el número de Reynolds supera un valor crítico (aproximadamente $Re \approx 40$). Los vórtices se desprenden alternativamente de ambos lados del cilindro, generando fluctuaciones periódicas en las fuerzas de sustentación y arrastre. Estas oscilaciones pueden provocar vibraciones inducidas por vórtices (\textit{vortex-induced vibrations}, VIV), fenómeno de gran relevancia en ingeniería para estructuras como chimeneas, cables submarinos y puentes.

El número de Reynolds define el régimen del flujo~\cite{roshko1954}:
\begin{itemize}
    \item $Re < 5$: Flujo adherido sin separación.
    \item $5 < Re < 40$: Separación con recirculación estacionaria.
    \item $40 < Re < 200$: Desprendimiento periódico laminar (calle de von Kármán).
    \item $200 < Re < 3 \times 10^5$: Régimen subcrítico con estela turbulenta y capa límite laminar.
    \item $Re > 3 \times 10^5$: Régimen supercrítico con transición de la capa límite a turbulenta.
\end{itemize}

La caracterización de este flujo se realiza mediante el número de Strouhal ($St$), que relaciona la frecuencia de desprendimiento de vórtices ($f$) con la velocidad de la corriente ($U_\infty$) y el diámetro del cilindro ($D$):
\begin{equation}
    St = \frac{f \cdot D}{U_\infty}
\end{equation}

Para cilindros circulares en el régimen subcrítico ($300 < Re < 2 \times 10^5$), Roshko~\cite{roshko1954} determinó experimentalmente que el número de Strouhal se encuentra en el rango $St \approx 0.20 - 0.21$, valor que ha sido confirmado por numerosos estudios posteriores~\cite{williamson1996vortex}.

%========================================
\section{Objetivo}
%========================================

El objetivo de este ejercicio es simular y analizar el comportamiento aerodinámico de un cilindro circular enfrentado a una corriente uniforme en régimen transitorio, aplicando los conocimientos adquiridos en mallado, condiciones de contorno, configuración de modelos de turbulencia y post-procesamiento.

Se pretende capturar el fenómeno de desprendimiento periódico de vórtices mediante una simulación RANS (\textit{Reynolds-Averaged Navier-Stokes}) transitoria, y caracterizar el flujo mediante el cálculo de:
\begin{itemize}
    \item La distribución del coeficiente de presión $C_p$ sobre la superficie del cilindro.
    \item La evolución temporal de los coeficientes de arrastre ($C_D$) y sustentación ($C_L$).
    \item El número de Strouhal mediante análisis espectral de las fuerzas fluctuantes.
    \item La comparación entre campos instantáneos y campos promediados en el tiempo.
\end{itemize}

%========================================
\section{Condiciones de simulación}
%========================================

\subsection{Parámetros del problema}

El número de Reynolds se calculó según la última cifra del DNI del alumno:
\begin{equation}
    Re = 200 + l_D \times 50 = 200 + 7 \times 50 = 550
\end{equation}
donde $l_D = 7$ es la última cifra del DNI.

Este valor sitúa el flujo en el régimen subcrítico con estela turbulenta, donde se espera un desprendimiento periódico de vórtices claramente definido.

Los parámetros dimensionales del problema se definieron como:
\begin{itemize}
    \item Diámetro del cilindro: $D = 1.0$ m
    \item Velocidad de corriente libre: $U_\infty = 1.0$ m/s
    \item Viscosidad cinemática: $\nu = \dfrac{U_\infty \cdot D}{Re} = \dfrac{1.0 \times 1.0}{550} = 1.82 \times 10^{-3}$ m$^2$/s
\end{itemize}

\subsection{Modelo de turbulencia}

Se empleó el modelo $k$-$\omega$ SST (\textit{Shear Stress Transport}) de Menter~\cite{menter1994two,versteeg2007introduction}, que combina las ventajas del modelo $k$-$\omega$ cerca de las paredes (buena predicción de separación) con las del modelo $k$-$\varepsilon$ en zonas alejadas (menor sensibilidad a condiciones de entrada).

Las ecuaciones de transporte para la energía cinética turbulenta ($k$) y la tasa de disipación específica ($\omega$) son:
\begin{equation}
    \frac{\partial k}{\partial t} + U_j \frac{\partial k}{\partial x_j} = P_k - \beta^* k \omega + \frac{\partial}{\partial x_j} \left[ (\nu + \sigma_k \nu_t) \frac{\partial k}{\partial x_j} \right]
\end{equation}
\begin{equation}
    \frac{\partial \omega}{\partial t} + U_j \frac{\partial \omega}{\partial x_j} = \alpha S^2 - \beta \omega^2 + \frac{\partial}{\partial x_j} \left[ (\nu + \sigma_\omega \nu_t) \frac{\partial \omega}{\partial x_j} \right] + 2(1-F_1) \sigma_{\omega 2} \frac{1}{\omega} \frac{\partial k}{\partial x_j} \frac{\partial \omega}{\partial x_j}
\end{equation}

Este modelo es especialmente efectivo para flujos con separación y gradientes de presión adversos, características presentes en el flujo alrededor de cilindros.

\subsection{Dominio computacional y malla}

El dominio computacional se generó mediante el utilitario \texttt{blockMesh} de OpenFOAM. \\
Se diseñó un dominio rectangular con las siguientes dimensiones para minimizar efectos de bloqueo y reflexiones de las condiciones de contorno:
\begin{itemize}
    \item Extensión aguas arriba: $50D$ (entrada)
    \item Extensión aguas abajo: $50D$ (salida)
    \item Extensión lateral: $\pm 50D$ (superior e inferior)
    \item Espesor en dirección $z$: 1 celda (simulación 2D)
\end{itemize}

La malla se refinó en las proximidades del cilindro mediante un factor de escala progresivo para capturar adecuadamente la capa límite y la separación del flujo. La malla resultante contiene aproximadamente 30\,000 celdas hexaédricas.

La Fig.~\ref{fig:malla_cilindro_ej7} muestra un detalle de la malla en las cercanías del cilindro.

\begin{figure}[h!]
    \centering
    \includegraphics[width=0.75\textwidth]{Ejercicio7/malla_cilindro.png}
    \caption{Detalle de la malla computacional en las proximidades del cilindro circular.}
    \label{fig:malla_cilindro_ej7}
\end{figure}

\subsection{Condiciones de contorno}

Las condiciones de contorno se configuraron en los archivos del directorio \texttt{0/} del caso de OpenFOAM:

\textbf{Entrada (\texttt{inlet}):}
\begin{itemize}
    \item Velocidad: $\mathbf{U} = (U_\infty, 0, 0) = (1.0, 0, 0)$ m/s (valor fijo)
    \item Presión: gradiente nulo (\texttt{zeroGradient})
    \item Energía cinética turbulenta: $k = 3.75 \times 10^{-4}$ m$^2$/s$^2$ (intensidad turbulenta $I = 1\%$)
    \item Tasa de disipación específica: $\omega = 1.0$ s$^{-1}$
\end{itemize}

\textbf{Salida (\texttt{outlet}):}
\begin{itemize}
    \item Velocidad: gradiente nulo (\texttt{zeroGradient})
    \item Presión: $p = 0$ Pa (referencia)
    \item Variables turbulentas: gradiente nulo
\end{itemize}

\textbf{Cilindro (\texttt{wall}):}
\begin{itemize}
    \item Velocidad: condición de no deslizamiento (\texttt{noSlip})
    \item Presión: gradiente nulo
    \item $k$: \texttt{kqRWallFunction} (función de pared de bajo Reynolds)
    \item $\omega$: \texttt{omegaWallFunction}
\end{itemize}

\textbf{Fronteras superior e inferior:}
\begin{itemize}
    \item Condición de simetría (\texttt{symmetry}) para todas las variables
\end{itemize}

\textbf{Dirección $z$ (frente y posterior):}
\begin{itemize}
    \item Tipo \texttt{empty} (simulación bidimensional)
\end{itemize}

\subsection{Configuración del solver}

Se utilizó el solver \texttt{incompressibleFluid} de OpenFOAM 13 con las siguientes configuraciones:

\begin{itemize}
    \item \textbf{Tiempo de simulación:} $t_{\text{final}} = 50$ s
    \item \textbf{Paso temporal:} adaptativo con número de Courant máximo $Co_{\max} = 0.5$
    \item \textbf{Intervalo de escritura:} $\Delta t_{\text{write}} = 0.1$ s
    \item \textbf{Algoritmo de acoplamiento presión-velocidad:} PIMPLE
    \item \textbf{Esquemas de discretización:}
    \begin{itemize}
        \item Temporal: Euler implícito de primer orden
        \item Gradiente: método de Gauss lineal
        \item Divergencia: esquema \texttt{linearUpwind} (segundo orden) para velocidad; \texttt{upwind} para variables turbulentas
        \item Laplaciano: Gauss lineal con corrección ortogonal
    \end{itemize}
\end{itemize}

\subsection{Post-procesamiento}

Se configuraron las siguientes funciones de post-procesamiento \\
en el archivo \texttt{system/controlDict}:

\begin{enumerate}
    \item \textbf{Función \texttt{forceCoeffsIncompressible}:} cálculo de coeficientes aerodinámicos ($C_D$, $C_L$, $C_M$) en cada paso de tiempo, con área de referencia $A_{\text{ref}} = D \times 1$ m$^2$ y velocidad de referencia $U_\infty = 1$ m/s.

    \item \textbf{Función \texttt{patchSurface}:} extracción de campos (presión, velocidad) sobre la superficie del cilindro para el análisis de la distribución de $C_p$.

    \item \textbf{Función \texttt{fieldAverage}:} cálculo de campos promediados en el tiempo ($\bar{U}$, $\bar{p}$) y fluctuaciones RMS para la comparación entre campos instantáneos y promedios.
\end{enumerate}

Adicionalmente, se desarrolló un script de MATLAB (\texttt{plot\_ejercicio7.m}) para el análisis espectral mediante FFT (\textit{Fast Fourier Transform}) de las series temporales de fuerzas, permitiendo la determinación del número de Strouhal.

%========================================
\section{Resultados}
%========================================

\subsection{Distribución de $C_p$ sobre el cilindro}

Se ha analizado la distribución del coeficiente de presión $C_p$ sobre la superficie del cilindro en varios instantes representativos del flujo periódico. El coeficiente de presión se define como:
\begin{equation}
    C_p = \frac{p - p_\infty}{\frac{1}{2} \rho U_\infty^2}
\end{equation}
donde $p$ es la presión local, $p_\infty$ es la presión de referencia (entrada), y $\rho U_\infty^2 / 2$ es la presión dinámica.

La Fig.~\ref{fig:Cp_distribucion_ej7} presenta la distribución de $C_p$ en función del ángulo $\theta$ \\
medido desde el punto de estancamiento frontal ($\theta = 0^\circ$ corresponde al frente del cilindro, $\theta = \pm 180^\circ$ a la parte posterior). \\
Se muestran cuatro instantes de tiempo distintos ($t = 25$, 30, 40, 50 s) junto con la solución teórica del flujo potencial no viscoso.

\begin{figure}[h!]
    \centering
    \includegraphics[width=0.9\textwidth]{Ejercicio7/Cp_distribucion.png}
    \caption{Distribución del coeficiente de presión $C_p$ sobre la superficie del cilindro en diferentes instantes de tiempo. Se compara con la solución teórica de flujo potencial (línea discontinua negra).}
    \label{fig:Cp_distribucion_ej7}
\end{figure}

\textbf{Análisis de la distribución de $C_p$:}

\begin{enumerate}
    \item \textbf{Punto de estancamiento frontal ($\theta = 0^\circ$):} Se observa $C_p \approx 1.0$ en todos los instantes, coincidiendo con la predicción del teorema de Bernoulli para flujo potencial. En este punto, la velocidad es nula y la presión alcanza su valor máximo.

    \item \textbf{Flancos laterales ($\theta \approx \pm 70^\circ - 90^\circ$):} La presión alcanza sus valores mínimos ($C_p \approx -2.5$ a $-3.0$) debido a la aceleración del flujo alrededor del cilindro. Estos valores son significativamente más bajos que la predicción del flujo potencial ($C_p \approx -3.0$ en $\theta = \pm 90^\circ$), lo cual es consecuencia de los efectos viscosos.

    \item \textbf{Zona posterior ($\theta \approx \pm 120^\circ - 180^\circ$):} Aquí se observa la mayor discrepancia con la teoría potencial. El flujo real presenta separación de la capa límite, generando una región de baja presión casi constante en la estela ($C_p \approx -1.0$ a $-1.2$). Esta zona de baja presión en la base del cilindro es la principal responsable del arrastre de presión.

    \item \textbf{Variaciones temporales:} Las diferencias entre los distintos instantes de tiempo son evidentes, particularmente en la zona posterior del cilindro. Estas fluctuaciones se deben al desprendimiento alternado de vórtices de la calle de von Kármán. Los picos y valles en la distribución de presión migran con el tiempo, reflejando la naturaleza transitoria del flujo.

    \item \textbf{Asimetría instantánea:} A diferencia de la solución teórica simétrica, las distribuciones instantáneas presentan asimetría debido al desprendimiento no simultáneo de vórtices de ambos lados del cilindro, lo cual genera el coeficiente de sustentación fluctuante.
\end{enumerate}

La discrepancia entre la simulación y la teoría potencial evidencia la importancia de los efectos viscosos y la separación del flujo en cuerpos romos, fenómenos que no pueden ser capturados por modelos de flujo potencial.

\subsection{Coeficiente de resistencia y comparación experimental}

Se ha calculado la evolución temporal de los coeficientes de arrastre ($C_D$) y sustentación ($C_L$) mediante la función \texttt{forceCoeffsIncompressible} de OpenFOAM, que integra las distribuciones de presión y esfuerzos viscosos sobre la superficie del cilindro.

Las Figs.~\ref{fig:Cd_vs_tiempo_ej7} y \ref{fig:Cl_vs_tiempo_ej7} muestran la evolución temporal completa de ambos coeficientes desde el inicio de la simulación hasta $t = 50$ s.

\begin{figure}[h!]
    \centering
    \includegraphics[width=0.85\textwidth]{Ejercicio7/Cd_vs_tiempo.png}
    \caption{Evolución temporal del coeficiente de arrastre $C_D$. Se indican el valor medio simulado y el valor experimental de referencia según Roshko~\cite{roshko1954}.}
    \label{fig:Cd_vs_tiempo_ej7}
\end{figure}

\begin{figure}[h!]
    \centering
    \includegraphics[width=0.85\textwidth]{Ejercicio7/Cl_vs_tiempo.png}
    \caption{Evolución temporal del coeficiente de sustentación $C_L$, mostrando las oscilaciones periódicas características del desprendimiento de vórtices.}
    \label{fig:Cl_vs_tiempo_ej7}
\end{figure}

\textbf{Análisis del coeficiente de arrastre:}

El coeficiente de arrastre presenta un transitorio inicial (aproximadamente $t < 20$ s) durante el cual se establece el patrón de desprendimiento de vórtices. Una vez alcanzado el régimen cuasi-estacionario, $C_D$ oscila alrededor de un valor medio:
\begin{equation}
    \bar{C}_D = 1.30 \quad \text{(valor medio simulado)}
\end{equation}

Las oscilaciones de $C_D$ tienen una amplitud pequeña ($\Delta C_D \approx 0.05$) y ocurren con el doble de la frecuencia del desprendimiento de vórtices. Esto se debe a que cada vez que se desprende un vórtice (ya sea del lado superior o inferior), se produce una perturbación en la presión de base del cilindro.

\textbf{Comparación con valores experimentales:}

Los datos experimentales de Roshko~\cite{roshko1954} para cilindros circulares en el rango $Re = 500-600$ indican un coeficiente de arrastre en el rango:
\begin{equation}
    C_D^{\text{exp}} \approx 1.0 - 1.2 \quad \text{\cite{roshko1954}}
\end{equation}

El valor obtenido en la simulación ($\bar{C}_D = 1.30$) presenta una sobreestimación del 8-30\% respecto a los valores experimentales. Esta discrepancia puede atribuirse a:
\begin{itemize}
    \item \textbf{Limitaciones del modelo RANS:} Los modelos RANS con promediado de Reynolds tienden a sobreestimar la disipación turbulenta en la estela, resultando en una región de baja presión más extensa en la base del cilindro.
    \item \textbf{Resolución de la malla:} Aunque la malla es adecuada para capturar el fenómeno global, una mayor refinación cerca de la superficie podría mejorar la predicción de la capa límite y, por tanto, del punto de separación.
    \item \textbf{Bidimensionalidad:} Las simulaciones 2D no capturan los efectos tridimensionales presentes en el flujo real, los cuales pueden afectar el arrastre.
\end{itemize}

No obstante, el orden de magnitud es correcto y las oscilaciones temporales son coherentes con el fenómeno físico esperado.

\textbf{Análisis del coeficiente de sustentación:}

El coeficiente de sustentación oscila periódicamente alrededor de cero, como se espera para un cilindro en flujo cruzado simétrico:
\begin{equation}
    \bar{C}_L \approx 0 \quad \text{(valor medio)}
\end{equation}

La amplitud de las oscilaciones es aproximadamente $C_L \approx \pm 0.1$, reflejando la alternancia del desprendimiento de vórtices entre los lados superior e inferior del cilindro. La Fig.~\ref{fig:Cd_Cl_detalle_ej7} muestra un detalle de las oscilaciones durante los últimos 10 segundos de simulación.

\begin{figure}[h!]
    \centering
    \includegraphics[width=0.9\textwidth]{Ejercicio7/Cd_Cl_detalle.png}
    \caption{Detalle de las oscilaciones de $C_D$ y $C_L$ durante los últimos 10 segundos de simulación, evidenciando el carácter periódico del desprendimiento de vórtices.}
    \label{fig:Cd_Cl_detalle_ej7}
\end{figure}

\subsection{Número de Strouhal}

El número de Strouhal es un parámetro adimensional que caracteriza la frecuencia del desprendimiento de vórtices:
\begin{equation}
    St = \frac{f \cdot D}{U_\infty}
\end{equation}
donde $f$ es la frecuencia dominante de las oscilaciones.

Para determinar $f$, se realizó un análisis espectral mediante la transformada rápida de Fourier (FFT) de la señal de $C_L(t)$ en el régimen cuasi-estacionario ($t > 20$ s). Previamente, se interpoló la señal a un muestreo uniforme de $\Delta t = 0.01$ s para aplicar correctamente el algoritmo FFT.

La Fig.~\ref{fig:strouhal_espectro_ej7} muestra el espectro de amplitud de $C_L$.

\begin{figure}[h!]
    \centering
    \includegraphics[width=0.85\textwidth]{Ejercicio7/Strouhal_espectro.png}
    \caption{Espectro de frecuencia del coeficiente de sustentación $C_L$. El pico dominante corresponde a la frecuencia de desprendimiento de vórtices.}
    \label{fig:strouhal_espectro_ej7}
\end{figure}

Del análisis espectral se obtiene:
\begin{equation}
    f_{\text{dominante}} = 0.167 \text{ Hz}
\end{equation}

Por lo tanto, el número de Strouhal simulado es:
\begin{equation}
    St = \frac{f \cdot D}{U_\infty} = \frac{0.167 \times 1.0}{1.0} = 0.167
\end{equation}

\textbf{Comparación con valores experimentales:}

Los experimentos clásicos de Roshko~\cite{roshko1954} para cilindros circulares en el régimen subcrítico ($300 < Re < 2 \times 10^5$) determinaron un número de Strouhal prácticamente constante:
\begin{equation}
    St^{\text{exp}} \approx 0.20 - 0.21 \quad \text{\cite{roshko1954,williamson1996vortex}}
\end{equation}

El valor obtenido en la simulación presenta una desviación del 20\% respecto al valor experimental. Esta diferencia puede explicarse por:
\begin{itemize}
    \item \textbf{Limitaciones del modelo URANS:} Los modelos RANS transitorios (URANS) \\
    tienen dificultades inherentes para capturar con precisión la dinámica de estructuras coherentes como los vórtices de von Kármán. Los modelos RANS están diseñados para flujos estacionarios y su extensión a flujos transitorios introduce aproximaciones.

    \item \textbf{Disipación numérica:} Los esquemas numéricos de primer y segundo orden introducen cierta disipación artificial que puede modificar la frecuencia de desprendimiento.

    \item \textbf{Tamaño del dominio y condiciones de contorno:} Aunque el dominio es suficientemente grande, las condiciones de simetría en las fronteras laterales pueden afectar ligeramente la dinámica de la estela.
\end{itemize}

A pesar de la diferencia cuantitativa, la simulación captura correctamente el fenómeno físico de desprendimiento periódico de vórtices, y el orden de magnitud del número de Strouhal es razonable. Para obtener predicciones más precisas, sería necesario emplear simulaciones LES (\textit{Large Eddy Simulation}) o DNS (\textit{Direct Numerical Simulation}), que resuelven explícitamente las estructuras turbulentas.

La Fig.~\ref{fig:comparacion_experimental_ej7} presenta una comparación gráfica entre los valores simulados y experimentales de $C_D$ y $St$.

\begin{figure}[h!]
    \centering
    \includegraphics[width=0.75\textwidth]{Ejercicio7/comparacion_experimental.png}
    \caption{Comparación entre los valores simulados y experimentales del coeficiente de arrastre $C_D$ y el número de Strouhal $St$ según Roshko~\cite{roshko1954}.}
    \label{fig:comparacion_experimental_ej7}
\end{figure}

\subsection{Comparación de campos instantáneos y promedios}

En simulaciones transitorias como esta, es fundamental distinguir entre los \textbf{campos instantáneos} y los \textbf{campos promediados en el tiempo}. OpenFOAM calcula automáticamente los campos promediados mediante la función \texttt{fieldAverage}.

\subsubsection{Campos instantáneos}

Los campos instantáneos representan el estado del flujo en un momento específico de tiempo. Capturan toda la dinámica del desprendimiento de vórtices, incluyendo las estructuras coherentes individuales.

La Fig.~\ref{fig:velocity_magnitude_ej7} muestra el campo instantáneo de magnitud de velocidad en $t = 50$ s.

\begin{figure}[h!]
    \centering
    \includegraphics[width=0.95\textwidth]{Ejercicio7/velocity_magnitude_t50_v2.png}
    \caption{Campo instantáneo de magnitud de velocidad en $t = 50$ s. Se observan claramente los vórtices alternados de la calle de von Kármán en la estela del cilindro.}
    \label{fig:velocity_magnitude_ej7}
\end{figure}

\textbf{Características del campo instantáneo de velocidad:}
\begin{itemize}
    \item Se observan claramente las estructuras vorticales individuales desprendiéndose alternativamente de ambos lados del cilindro.
    \item La estela presenta un patrón asimétrico con vórtices rotando en sentidos opuestos.
    \item La región de baja velocidad en la estela se extiende varias decenas de diámetros aguas abajo.
    \item El flujo muestra fluctuaciones espaciales significativas asociadas al desprendimiento.
\end{itemize}

La Fig.~\ref{fig:velocity_detalle_ej7} presenta un detalle del campo de velocidad cerca del cilindro, donde se aprecia el punto de separación de la capa límite.

\begin{figure}[h!]
    \centering
    \includegraphics[width=0.80\textwidth]{Ejercicio7/velocity_detalle_cilindro.png}
    \caption{Detalle del campo instantáneo de velocidad en las proximidades del cilindro. Se aprecia la separación de la capa límite y el inicio de la formación de vórtices.}
    \label{fig:velocity_detalle_ej7}
\end{figure}

La Fig.~\ref{fig:pressure_ej7} muestra el campo instantáneo de presión.

\begin{figure}[h!]
    \centering
    \includegraphics[width=0.95\textwidth]{Ejercicio7/pressure_t50_v2.png}
    \caption{Campo instantáneo de presión en $t = 50$ s. Se observa la zona de alta presión en el punto de estancamiento frontal y las zonas de baja presión en los núcleos de los vórtices desprendidos.}
    \label{fig:pressure_ej7}
\end{figure}

\textbf{Características del campo instantáneo de presión:}
\begin{itemize}
    \item Alta presión en el punto de estancamiento frontal ($C_p \approx 1$).
    \item Zonas de baja presión localizadas en los núcleos de los vórtices desprendidos (visibles como manchas azules oscuras en la estela).
    \item Presión baja y casi constante en la región de recirculación inmediatamente detrás del cilindro.
    \item Gradientes de presión significativos en las zonas de aceleración lateral del flujo.
\end{itemize}

La Fig.~\ref{fig:estela_detalle_ej7} muestra una secuencia del campo de velocidad en la estela, evidenciando el desprendimiento alternado.

\begin{figure}[h!]
    \centering
    \includegraphics[width=0.85\textwidth]{Ejercicio7/velocity_estela_t50.png}
    \caption{Detalle de la calle de von Kármán en la estela del cilindro. Se aprecia el patrón alternado de vórtices rotando en sentidos opuestos.}
    \label{fig:estela_detalle_ej7}
\end{figure}

\subsubsection{Campos promediados en el tiempo}

Los campos promediados representan el valor medio del flujo sobre un intervalo de tiempo suficientemente largo para eliminar las fluctuaciones periódicas. Se calculan como:
\begin{equation}
    \bar{\phi}(\mathbf{x}) = \frac{1}{T} \int_{t_0}^{t_0 + T} \phi(\mathbf{x}, t) \, dt
\end{equation}
donde $\phi$ puede ser velocidad, presión, u otra variable del flujo.

\textbf{Características de los campos promediados:}
\begin{itemize}
    \item La estela promediada presenta simetría respecto al eje horizontal del cilindro, ya que los vórtices se desprenden con igual probabilidad de ambos lados.
    \item No se observan estructuras vorticales individuales; la estela aparece como una región de velocidad reducida de forma suave y simétrica.
    \item El coeficiente de sustentación promediado es $\bar{C}_L \approx 0$ debido a la simetría estadística.
    \item La presión promediada en la base del cilindro es menor que en el flujo potencial, reflejando el arrastre de presión medio.
\end{itemize}

\subsubsection{Diferencias fundamentales entre campos instantáneos y promedios}

\begin{enumerate}
    \item \textbf{Estructura de la estela:}
    \begin{itemize}
        \item \textit{Instantáneo:} Vórtices discretos alternados claramente visibles.
        \item \textit{Promedio:} Estela simétrica sin estructuras coherentes.
    \end{itemize}

    \item \textbf{Coeficiente de sustentación:}
    \begin{itemize}
        \item \textit{Instantáneo:} $C_L(t)$ oscila periódicamente entre $\pm 0.35$.
        \item \textit{Promedio:} $\bar{C}_L \approx 0$ (cancelación por simetría).
    \end{itemize}

    \item \textbf{Campo de presión:}
    \begin{itemize}
        \item \textit{Instantáneo:} Núcleos de baja presión en vórtices individuales.
        \item \textit{Promedio:} Zona de baja presión uniforme en la estela.
    \end{itemize}

    \item \textbf{Utilidad práctica:}
    \begin{itemize}
        \item \textit{Instantáneo:} Necesario para análisis espectral (cálculo de $St$) y para estudiar la dinámica de vórtices. Representa las cargas fluctuantes sobre estructuras.
        \item \textit{Promedio:} Útil para validación con datos experimentales de túnel de viento (sondas de presión promediada) y para cálculo de cargas medias.
    \end{itemize}

    \item \textbf{Fluctuaciones turbulentas:}
    \begin{itemize}
        \item La desviación cuadrática media $u'_{\text{rms}} = \sqrt{\overline{(u - \bar{u})^2}}$ cuantifica la intensidad de las fluctuaciones. Es máxima en la región de separación y en la estela cercana, donde ocurre el desprendimiento de vórtices.
    \end{itemize}
\end{enumerate}

Estas diferencias son fundamentales para entender el fenómeno de desprendimiento de vórtices: los campos instantáneos capturan la \textit{dinámica temporal} del flujo (esencial para estudiar vibraciones inducidas por vórtices), mientras que los promedios proporcionan información \textit{estadística} del comportamiento medio (útil para el diseño estructural basado en cargas promedio).

%========================================
\section{Conclusiones}
%========================================

La simulación transitoria del flujo turbulento alrededor de un cilindro circular mediante el modelo RANS $k$-$\omega$ SST en OpenFOAM ha permitido capturar el fenómeno de desprendimiento periódico de vórtices (calle de von Kármán) y caracterizar el flujo mediante diversos parámetros aerodinámicos. Las principales conclusiones son:

\begin{enumerate}
    \item \textbf{Distribución de $C_p$:} Se ha observado que la distribución del coeficiente de presión sobre el cilindro presenta un punto de estancamiento frontal con $C_p \approx 1$, presiones mínimas en los flancos laterales ($C_p \approx -2.5$), y una región de baja presión casi constante en la base ($C_p \approx -1.0$ a $-1.2$) debido a la separación del flujo. Las variaciones temporales reflejan el carácter transitorio del desprendimiento de vórtices.

    \item \textbf{Coeficiente de arrastre:} El valor medio del coeficiente de arrastre obtenido es $\bar{C}_D = 1.30$, lo cual representa una sobreestimación del 8-30\% respecto a los valores experimentales de Roshko~\cite{roshko1954} ($C_D^{\text{exp}} \approx 1.0 - 1.2$). Esta discrepancia es atribuible a las limitaciones inherentes de los modelos URANS para capturar con precisión la dinámica de la estela turbulenta.

    \item \textbf{Número de Strouhal:} El análisis espectral mediante FFT de la señal de $C_L(t)$ ha permitido determinar la frecuencia de desprendimiento $f = 0.167$ Hz, resultando en un número de Strouhal $St = 0.167$. Este valor presenta una desviación del 20\% respecto al valor experimental~\cite{roshko1954,williamson1996vortex} ($St^{\text{exp}} \approx 0.21$), lo cual es razonable considerando las limitaciones del modelo RANS transitorio.

    \item \textbf{Modelo de turbulencia:} El modelo $k$-$\omega$ SST proporciona resultados cualitativamente correctos y cuantitativamente aceptables para este tipo de flujo. Sin embargo, para predicciones más precisas de la frecuencia de desprendimiento y las amplitudes de las fuerzas fluctuantes, se recomienda emplear simulaciones LES o DNS.

    \item \textbf{Campos instantáneos vs promedios:} Se han identificado diferencias fundamentales entre ambos tipos de campos. Los instantáneos capturan la dinámica del desprendimiento de vórtices (esencial para el cálculo del número de Strouhal y el estudio de VIV), mientras que los promedios proporcionan información estadística útil para comparación con experimentos y cálculo de cargas medias.
\end{enumerate}

En resumen, este ejercicio ha permitido aplicar de manera integral los conocimientos adquiridos durante el curso para estudiar un problema clásico de aerodinámica externa con relevancia práctica en ingeniería.


%========================================
% BIBLIOGRAFIA
%========================================
\renewcommand\bibname{Referencias}
% Incluir todas las entradas del .bib aunque no se citen explícitamente
\nocite{*}
\bibliographystyle{splncs04}
\bibliography{references}



\end{document}
