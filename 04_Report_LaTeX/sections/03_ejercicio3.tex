%========================================
% CAPITULO 3: MÉTODO VORTEX LATTICE
%========================================
\chapter{Método Vortex Lattice}
\label{chap:ejercicio3}

%========================================
\section{Introducción}
%========================================

El método Vortex Lattice (VLM) constituye una de las herramientas más eficientes para el análisis aerodinámico de superficies sustentadoras tridimensionales en régimen potencial. Desarrollado originalmente en la década de 1960~\cite{katz2001low}, este método extiende los principios de la teoría de la línea sustentadora de Prandtl para permitir el análisis de configuraciones complejas con flecha, torsión y múltiples superficies.

El VLM discretiza la superficie sustentadora en una malla de paneles cuadriláteros, colocando un vórtice en herradura (\textit{horseshoe vortex}) en cada panel. Esta aproximación permite modelar con precisión los efectos tridimensionales del flujo, incluyendo el \textit{downwash} inducido y las interferencias entre múltiples superficies sustentadoras.

Las principales ventajas del método VLM son:
\begin{itemize}
    \item Eficiencia computacional elevada comparada con métodos de paneles tridimensionales completos.
    \item Capacidad para modelar geometrías complejas: flecha, estrechamiento, torsión y diedro.
    \item Predicción directa de coeficientes aerodinámicos ($C_L$, $C_{Di}$, $C_M$) sin necesidad de integración de presiones.
    \item Modelado de interferencias entre múltiples superficies (biplanos, canards, configuraciones tándem).
\end{itemize}

En este ejercicio se analiza una configuración tándem compuesta por un ala principal y un ala trasera, donde la segunda opera parcialmente en el campo de estela de la primera. Esta configuración presenta efectos de interferencia aerodinámica significativos que el método VLM permite caracterizar de forma eficiente.

%========================================
\section{Objetivo}
%========================================

El objetivo de este ejercicio es implementar el método Vortex Lattice para analizar el comportamiento aerodinámico de una configuración tándem de dos alas. Se pretende calcular y representar:
\begin{itemize}
    \item El coeficiente de sustentación ($C_L$) para cada ala y para el conjunto.
    \item El coeficiente de resistencia inducida ($C_{Di}$) individual y total.
    \item El coeficiente de momento de cabeceo ($C_M$) respecto al punto de referencia.
    \item La distribución de circulación ($\Gamma$) a lo largo de la envergadura.
    \item El coeficiente de presión ($C_p$) en función de la posición spanwise.
\end{itemize}

El análisis se realiza para un rango de ángulos de ataque desde $\alpha = -5^\circ$ hasta $\alpha = 10^\circ$, permitiendo caracterizar las curvas polares y evaluar los efectos de interferencia entre ambas superficies sustentadoras.

%========================================
\section{Condiciones de simulación}
%========================================

\subsection{Parámetros geométricos}

La configuración tándem analizada consta de dos alas con las características geométricas especificadas en el enunciado del ejercicio.

\textbf{Ala principal:}
\begin{itemize}
    \item \textbf{Envergadura:} $b_1 = 14.0$ m
    \item \textbf{Ángulo de flecha:} $\Lambda_1 = 20^\circ$
    \item \textbf{Cuerda en la raíz:} $c_{r1} = 1.7$ m
    \item \textbf{Cuerda en la punta:} $c_{t1} = 0.9$ m
    \item \textbf{Ley de torsión:} Distribución lineal desde $0^\circ$ en la raíz hasta $+4^\circ$ en la punta (\textit{washin})
    \item \textbf{Perfil aerodinámico:} NACA 2414 ($\alpha_0 = -2^\circ$)
\end{itemize}

\textbf{Ala trasera:}
\begin{itemize}
    \item \textbf{Envergadura:} $b_2 = 9.0$ m
    \item \textbf{Ángulo de flecha:} $\Lambda_2 = 13^\circ$
    \item \textbf{Cuerda en la raíz:} $c_{r2} = 1.3$ m
    \item \textbf{Cuerda en la punta:} $c_{t2} = 0.65$ m
    \item \textbf{Ley de torsión:} Ala plana sin torsión ($\theta = 0^\circ$)
    \item \textbf{Perfil aerodinámico:} NACA 0016 ($\alpha_0 = 0^\circ$)
    \item \textbf{Separación longitudinal:} $\Delta x = 9.0$ m respecto al ala principal (distancia P-P')
\end{itemize}

La superficie de referencia para la adimensionalización de coeficientes es el área del ala principal:
\begin{equation}
    S_{ref} = \frac{b_1 (c_{r1} + c_{t1})}{2} = \frac{14.0 \times (1.7 + 0.9)}{2} = 18.2 \text{ m}^2
\end{equation}

El punto de referencia para el cálculo de momentos se sitúa en el cuarto de cuerda de la raíz del ala principal (punto P).

\subsection{Discretización de la malla}

Siguiendo las especificaciones del enunciado, se empleó la siguiente discretización:
\begin{itemize}
    \item \textbf{Ala principal:} $30 \times 10 = 300$ paneles por semiala (600 paneles totales)
    \item \textbf{Ala trasera:} $20 \times 8 = 160$ paneles por semiala (320 paneles totales)
    \item \textbf{Total:} 920 paneles en la configuración completa
\end{itemize}

La distribución de paneles en la dirección de la envergadura sigue una ley coseno para concentrar elementos en las regiones de mayor gradiente de circulación (puntas de ala):
\begin{equation}
    y_j = \frac{b}{2} \sin\left(\frac{j \pi}{2 N_y}\right), \quad j = 0, 1, \ldots, N_y
\end{equation}

%========================================
\section{Fundamentos teóricos}
%========================================

\subsection{Vórtice en herradura}

El elemento fundamental del método VLM es el vórtice en herradura (\textit{horseshoe vortex}), compuesto por:
\begin{enumerate}
    \item Un \textbf{vórtice ligado} (\textit{bound vortex}) situado en el cuarto de cuerda del panel.
    \item Dos \textbf{vórtices de estela} (\textit{trailing vortices}) que se extienden hacia infinito aguas abajo.
\end{enumerate}

La velocidad inducida por cada segmento de vórtice se calcula mediante la ley de Biot-Savart:
\begin{equation}
    d\vec{V} = \frac{\Gamma}{4\pi} \frac{d\vec{l} \times \vec{r}}{|\vec{r}|^3}
\end{equation}

Para un segmento finito de vórtice entre los puntos A y B, la velocidad inducida en un punto P es:
\begin{equation}
    \vec{V}_{AB} = \frac{\Gamma}{4\pi} \frac{\vec{r}_1 \times \vec{r}_2}{|\vec{r}_1 \times \vec{r}_2|^2} \left( \vec{r}_0 \cdot \frac{\vec{r}_1}{|\vec{r}_1|} - \vec{r}_0 \cdot \frac{\vec{r}_2}{|\vec{r}_2|} \right)
\end{equation}
donde $\vec{r}_0 = \vec{B} - \vec{A}$, $\vec{r}_1 = \vec{P} - \vec{A}$ y $\vec{r}_2 = \vec{P} - \vec{B}$.

\subsection{Condición de contorno}

La condición de contorno del VLM establece que la componente normal de la velocidad total debe ser nula en cada punto de control (situado en el 3/4 de cuerda del panel):
\begin{equation}
    V_{\infty} \sin(\alpha_{eff}) + \sum_{j=1}^{N} a_{ij} \Gamma_j = 0
\end{equation}

donde $a_{ij}$ representa el coeficiente de influencia del panel $j$ sobre el punto de control $i$, y el ángulo de ataque efectivo local es:
\begin{equation}
    \alpha_{eff} = \alpha_{geom} + \theta_{twist} - \alpha_0
\end{equation}

siendo $\alpha_{geom}$ el ángulo de ataque geométrico, $\theta_{twist}$ la torsión local y $\alpha_0$ el ángulo de sustentación nula del perfil.

\subsection{Cálculo de fuerzas aerodinámicas}

Una vez resuelto el sistema lineal $[AIC]\{\Gamma\} = \{RHS\}$, las fuerzas aerodinámicas se calculan mediante el teorema de Kutta-Joukowski:

\textbf{Sustentación local:}
\begin{equation}
    dL = \rho V_\infty \Gamma \, dy
\end{equation}

\textbf{Resistencia inducida local:}
\begin{equation}
    dD_i = -\rho \, w_{ind} \, \Gamma \, dy
\end{equation}
donde $w_{ind}$ es la velocidad de \textit{downwash} inducida en el panel.

\textbf{Momento de cabeceo:}
\begin{equation}
    dM = -dL \cdot (x_{panel} - x_{ref})
\end{equation}

Los coeficientes adimensionales se obtienen dividiendo por la presión dinámica y las magnitudes de referencia:
\begin{equation}
    C_L = \frac{L}{q_\infty S_{ref}}, \quad C_{Di} = \frac{D_i}{q_\infty S_{ref}}, \quad C_M = \frac{M}{q_\infty S_{ref} \bar{c}}
\end{equation}

%========================================
\section{Resultados}
%========================================

\subsection{Coeficiente de sustentación}

La Fig.~\ref{fig:CL_VLM} presenta la variación del coeficiente de sustentación con el ángulo de ataque para el ala principal, el ala trasera y la configuración completa.

\begin{figure}[h!]
    \centering
    \includegraphics[width=0.85\textwidth]{Ejercicio3/CL_vs_alpha.png}
    \caption{Coeficiente de sustentación $C_L$ en función del ángulo de ataque $\alpha$ para el ala principal, ala trasera y conjunto.}
    \label{fig:CL_VLM}
\end{figure}

\textbf{Análisis del coeficiente de sustentación:}

\begin{enumerate}
    \item \textbf{Ala principal:} Presenta una pendiente de sustentación de $dC_L/d\alpha \approx 0.086$ deg$^{-1}$, valor típico para alas de alargamiento moderado con flecha positiva. El efecto combinado de la torsión (\textit{washin}) y el ángulo de sustentación nula del perfil NACA 2414 ($\alpha_0 = -2^\circ$) resulta en sustentación positiva incluso a $\alpha = 0^\circ$.

    \item \textbf{Ala trasera:} Opera parcialmente en el campo de \textit{downwash} del ala principal, resultando en valores de $C_L$ negativos o muy bajos a ángulos de ataque pequeños. A $\alpha = 0^\circ$, el ala trasera presenta $C_{L,rear} \approx -0.022$, indicando que el \textit{downwash} inducido supera al ángulo de ataque geométrico.

    \item \textbf{Configuración completa:} La pendiente de sustentación total es $dC_L/d\alpha \approx 0.115$ deg$^{-1}$, superior a la del ala principal aislada debido a la contribución positiva del ala trasera a ángulos de ataque elevados.

    \item \textbf{Ángulo de sustentación nula:} La configuración completa presenta $C_L = 0$ aproximadamente a $\alpha \approx -2.5^\circ$, influenciado principalmente por el perfil NACA 2414 del ala principal.
\end{enumerate}

\subsection{Coeficiente de resistencia inducida}

La Fig.~\ref{fig:CDi_VLM} muestra la evolución del coeficiente de resistencia inducida con el ángulo de ataque.

\begin{figure}[h!]
    \centering
    \includegraphics[width=0.85\textwidth]{Ejercicio3/CDi_vs_alpha.png}
    \caption{Coeficiente de resistencia inducida $C_{Di}$ en función del ángulo de ataque $\alpha$.}
    \label{fig:CDi_VLM}
\end{figure}

\textbf{Análisis de la resistencia inducida:}

\begin{enumerate}
    \item \textbf{Relación cuadrática:} La resistencia inducida sigue la relación teórica $C_{Di} \propto C_L^2$, con un mínimo cercano a $\alpha \approx -2.5^\circ$ donde $C_L \approx 0$.

    \item \textbf{Contribución del ala trasera:} A pesar de generar sustentación negativa a ángulos bajos, el ala trasera contribuye significativamente a la resistencia inducida total debido al \textit{downwash} del ala principal.

    \item \textbf{Valores característicos:} A $\alpha = 10^\circ$, la resistencia inducida total alcanza $C_{Di} \approx 0.327$, siendo la contribución del ala principal aproximadamente el 75\% del total.
\end{enumerate}

\subsection{Coeficiente de momento de cabeceo}

El coeficiente de momento de cabeceo, calculado respecto al punto P (cuarto de cuerda de la raíz del ala principal), se presenta en la Fig.~\ref{fig:CM_VLM}.

\begin{figure}[h!]
    \centering
    \includegraphics[width=0.85\textwidth]{Ejercicio3/CM_vs_alpha.png}
    \caption{Coeficiente de momento de cabeceo $C_M$ en función del ángulo de ataque $\alpha$.}
    \label{fig:CM_VLM}
\end{figure}

\textbf{Análisis del momento de cabeceo:}

\begin{enumerate}
    \item \textbf{Estabilidad longitudinal:} La derivada $dC_M/d\alpha < 0$ indica estabilidad longitudinal estática positiva. El valor aproximado es $dC_M/d\alpha \approx -0.28$ deg$^{-1}$.

    \item \textbf{Momento a $\alpha = 0$:} El momento de cabeceo a sustentación de crucero ($\alpha = 0^\circ$) es ligeramente negativo ($C_M \approx -0.13$), indicando una tendencia al picado (\textit{nose-down}).

    \item \textbf{Contribución del ala trasera:} El ala trasera, situada 9 m aguas abajo del punto de referencia, genera momentos significativos debido al gran brazo de palanca. A ángulos positivos, su sustentación positiva contribuye al momento de picado.
\end{enumerate}

\subsection{Polar de resistencia inducida}

La Fig.~\ref{fig:polar_VLM} presenta la polar de resistencia inducida ($C_L$ vs $C_{Di}$).

\begin{figure}[h!]
    \centering
    \includegraphics[width=0.8\textwidth]{Ejercicio3/polar_drag.png}
    \caption{Polar de resistencia inducida de la configuración tándem.}
    \label{fig:polar_VLM}
\end{figure}

La forma parabólica característica se observa claramente. El factor de eficiencia de Oswald puede estimarse a partir de la relación:
\begin{equation}
    C_{Di} = \frac{C_L^2}{\pi e AR}
\end{equation}

donde $AR = b^2/S$ es el alargamiento del ala de referencia.

\subsection{Eficiencia aerodinámica}

La Fig.~\ref{fig:eficiencia_VLM} muestra la eficiencia aerodinámica ($C_L/C_{Di}$) en función del ángulo de ataque.

\begin{figure}[h!]
    \centering
    \includegraphics[width=0.8\textwidth]{Ejercicio3/eficiencia.png}
    \caption{Eficiencia aerodinámica $C_L/C_{Di}$ en función del ángulo de ataque.}
    \label{fig:eficiencia_VLM}
\end{figure}

La máxima eficiencia aerodinámica se alcanza a ángulos de ataque moderados ($\alpha \approx 2-3^\circ$), donde el equilibrio entre sustentación y resistencia inducida es óptimo. A ángulos elevados, la eficiencia decrece debido al aumento cuadrático de la resistencia inducida.

\subsection{Distribución de circulación}

La Fig.~\ref{fig:Gamma_VLM} presenta la distribución de circulación a lo largo de la envergadura para ambas alas y diferentes ángulos de ataque.

\begin{figure}[h!]
    \centering
    \includegraphics[width=\textwidth]{Ejercicio3/Gamma_distribucion.png}
    \caption{Distribución de circulación $\Gamma$ a lo largo de la envergadura para diferentes ángulos de ataque.}
    \label{fig:Gamma_VLM}
\end{figure}

\textbf{Análisis de la distribución de circulación:}

\begin{enumerate}
    \item \textbf{Ala principal:} La distribución de circulación se aproxima a la forma elíptica ideal, con máximo en la raíz y decrecimiento hacia las puntas. La torsión (\textit{washin}) modifica ligeramente esta distribución.

    \item \textbf{Ala trasera:} A ángulos de ataque bajos, la circulación es negativa (sustentación hacia abajo) debido al \textit{downwash} del ala principal. A medida que aumenta $\alpha$, la circulación se vuelve positiva.

    \item \textbf{Efectos de la flecha:} El ángulo de flecha produce una distribución de circulación más uniforme a lo largo de la envergadura, mejorando la eficiencia aerodinámica.
\end{enumerate}

\subsection{Distribución del coeficiente de presión}

La Fig.~\ref{fig:Cp_completa_VLM} presenta una análisis detallado de la distribución del coeficiente de presión $C_p$ a lo largo de la envergadura para diferentes ángulos de ataque. La gráfica se organiza en tres subfiguras dispuestas en formato 3×1 para facilitar la comparación entre las componentes de la configuración tándem.

\begin{figure}[h!]
    \centering
    \includegraphics[width=\textwidth]{Ejercicio3/Cp_distribucion_completa.png}
    \caption{Distribución del coeficiente de presión $C_p$ a lo largo de la envergadura. (a) Ala principal, (b) Ala trasera, (c) Configuración completa. Los diferentes colores representan distintos ángulos de ataque $\alpha$, mientras que los tipos de línea distinguen entre ambas alas en el caso conjunto.}
    \label{fig:Cp_completa_VLM}
\end{figure}
\end{figure}

\textbf{Análisis de la distribución de presiones:}

\begin{enumerate}
    \item \textbf{Ala principal (Fig.~\ref{fig:Cp_completa_VLM}a):} La distribución de $C_p$ muestra el comportamiento característico de un ala con flecha y torsión. A ángulos de ataque bajos ($\alpha = -5^\circ$), se observa succión moderada en la región central del ala, con valores de $C_p$ alrededor de $-0.5$. A medida que aumenta el ángulo de ataque, la succión máxima se incrementa significativamente, alcanzando valores de $C_p \approx -2.5$ a $\alpha = 10^\circ$. La distribución spanwise refleja los efectos de la flecha positiva, con una ligera disminución de la succión hacia las puntas debido al \textit{washout} inducido.

    \item \textbf{Ala trasera (Fig.~\ref{fig:Cp_completa_VLM}b):} La distribución de presiones del ala trasera es notablemente diferente debido a los efectos de interferencia aerodinámica. A ángulos de ataque bajos ($\alpha \leq 0^\circ$), se observa una distribución de presiones invertida (presión positiva en la superficie superior), correspondiente a sustentación negativa. Este comportamiento se debe al \textit{downwash} inducido por el ala principal, que supera al ángulo de ataque geométrico. A ángulos elevados ($\alpha \geq 5^\circ$), la distribución se normaliza, mostrando succión en la superficie superior con valores de $C_p$ mínimos alrededor de $-1.5$.

    \item \textbf{Configuración completa (Fig.~\ref{fig:Cp_completa_VLM}c):} La subfigura inferior combina ambas alas en un único gráfico, utilizando líneas continuas para el ala principal y líneas discontinuas para el ala trasera. Esta representación permite apreciar directamente los efectos de interferencia: el ala trasera opera en condiciones de flujo perturbado por el ala principal, resultando en distribuciones de presión asimétricas y efectos de interacción complejos. Los diferentes colores permiten seguir la evolución de las distribuciones con el ángulo de ataque para ambas superficies simultáneamente.
\end{enumerate}

La evolución de las distribuciones de presión con el ángulo de ataque refleja claramente los mecanismos físicos subyacentes: el aumento de la velocidad local en la superficie superior (principio de Bernoulli) genera succión creciente, mientras que los efectos de interferencia modifican significativamente el comportamiento del ala trasera. Esta información es crucial para el diseño de flaps, slats y otras superficies de control de alta sustentación.

\subsection{Resumen de coeficientes aerodinámicos}

La Fig.~\ref{fig:resumen_VLM} presenta un resumen comparativo de los principales coeficientes aerodinámicos.

\begin{figure}[h!]
    \centering
    \includegraphics[width=\textwidth]{Ejercicio3/resumen_VLM.png}
    \caption{Resumen de coeficientes aerodinámicos: $C_L$, $C_{Di}$ y $C_M$ para la configuración tándem.}
    \label{fig:resumen_VLM}
\end{figure}

\subsection{Visualización de la geometría}

La Fig.~\ref{fig:geometria_VLM} muestra la malla de paneles utilizada en el análisis VLM.

\begin{figure}[h!]
    \centering
    \includegraphics[width=0.85\textwidth]{Ejercicio3/geometria_3D.png}
    \caption{Visualización 3D de la geometría de la configuración tándem mostrando la malla de paneles.}
    \label{fig:geometria_VLM}
\end{figure}

\subsection{Tabla de resultados numéricos}

La Tabla~\ref{tab:resultados_ej3} presenta los valores numéricos de los coeficientes aerodinámicos para cada ángulo de ataque analizado.

\begin{table}[h!]
    \centering
    \caption{Coeficientes aerodinámicos de la configuración tándem calculados mediante VLM.}
    \label{tab:resultados_ej3}
    \begin{tabular}{ccccccc}
        \hline
        $\alpha$ [deg] & $C_{L,main}$ & $C_{L,rear}$ & $C_{L,total}$ & $C_{Di,total}$ & $C_{M,total}$ \\
        \hline
        $-5.0$ & $-0.118$ & $-0.167$ & $-0.285$ & $0.0183$ & $+1.253$ \\
        $-2.5$ & $+0.099$ & $-0.094$ & $+0.005$ & $0.0076$ & $+0.562$ \\
        $-1.0$ & $+0.229$ & $-0.051$ & $+0.179$ & $0.0133$ & $+0.147$ \\
        $0.0$ & $+0.316$ & $-0.022$ & $+0.294$ & $0.0221$ & $-0.129$ \\
        $+1.0$ & $+0.403$ & $+0.007$ & $+0.410$ & $0.0350$ & $-0.406$ \\
        $+2.5$ & $+0.532$ & $+0.051$ & $+0.583$ & $0.0617$ & $-0.821$ \\
        $+5.0$ & $+0.748$ & $+0.123$ & $+0.871$ & $0.1261$ & $-1.511$ \\
        $+10.0$ & $+1.174$ & $+0.267$ & $+1.441$ & $0.3272$ & $-2.881$ \\
        \hline
    \end{tabular}
\end{table}

%========================================
\section{Conclusiones}
%========================================

La implementación del método Vortex Lattice ha permitido analizar de forma eficiente y precisa la configuración tándem de alas propuesta. Las principales conclusiones del análisis son:

\begin{enumerate}
    \item \textbf{Interferencia aerodinámica:} La configuración tándem presenta efectos de interferencia significativos. El ala trasera opera en el campo de \textit{downwash} del ala principal, resultando en sustentación reducida o negativa a ángulos de ataque bajos. Este efecto se invierte a ángulos elevados donde ambas alas contribuyen positivamente a la sustentación total.

    \item \textbf{Coeficiente de sustentación:} La pendiente de sustentación del ala principal es $dC_L/d\alpha \approx 0.086$ deg$^{-1}$, mientras que la configuración completa presenta $dC_L/d\alpha \approx 0.115$ deg$^{-1}$. El ángulo de sustentación nula de la configuración es aproximadamente $\alpha_0 \approx -2.5^\circ$.

    \item \textbf{Resistencia inducida:} La resistencia inducida sigue la relación cuadrática esperada con el coeficiente de sustentación. A $\alpha = 10^\circ$, el coeficiente de resistencia inducida alcanza $C_{Di} \approx 0.327$, siendo el ala principal responsable del 75\% aproximadamente.

    \item \textbf{Estabilidad longitudinal:} La configuración presenta estabilidad longitudinal estática positiva con $dC_M/d\alpha \approx -0.28$ deg$^{-1}$. El momento de cabeceo a $\alpha = 0^\circ$ es ligeramente negativo ($C_M \approx -0.13$).

    \item \textbf{Distribución de circulación:} El ala principal presenta una distribución de circulación quasi-elíptica, mientras que el ala trasera muestra distribuciones más complejas debido a los efectos de interferencia.

    \item \textbf{Eficiencia del método:} El VLM demuestra ser una herramienta eficiente para el análisis preliminar de configuraciones multi-superficie, proporcionando resultados coherentes con la teoría aerodinámica con un coste computacional reducido (920 paneles resueltos en segundos).

    \item \textbf{Efectos geométricos:} La flecha positiva del ala principal ($\Lambda = 20^\circ$) y la torsión tipo \textit{washin} ($+4^\circ$ en punta) influyen significativamente en las distribuciones de carga y los coeficientes globales.
\end{enumerate}

En resumen, el método Vortex Lattice constituye una herramienta valiosa para el diseño conceptual de configuraciones aerodinámicas complejas, permitiendo evaluar rápidamente los efectos de interferencia y las características de estabilidad de configuraciones multi-superficie.
