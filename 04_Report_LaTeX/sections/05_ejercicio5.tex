%========================================
% CAPITULO 5: EFECTO DE MODELOS DE PARED EN FLUJO DE COUETTE PLANAR TURBULENTO
%========================================
\chapter{Efecto de Modelos de Pared en Flujo de Couette Planar Turbulento}
\label{chap:ejercicio5}

%========================================
\section{Introducción}
%========================================

El flujo de Couette planar, caracterizado por el movimiento relativo entre dos placas paralelas infinitas, constituye uno de los problemas canónicos de la mecánica de fluidos. Mientras que en régimen laminar la solución es una distribución lineal de velocidades, la transición al régimen turbulento introduce una complejidad fenomenológica significativa debido a la presencia de estructuras coherentes y fluctuaciones de pequeña escala que redistribuyen el momento lineal~\cite{schlichting2016boundary}.

En simulaciones de dinámica de fluidos computacional (CFD), el modelado de la capa límite turbulenta cerca de las paredes es un aspecto crítico. La subcapa viscosa, donde los efectos moleculares dominan, requiere una resolución espacial extremadamente fina ($y^+ \approx 1$) para ser resuelta directamente. Como alternativa para reducir el coste computacional, se emplean comúnmente las funciones de pared (\textit{wall functions}), que asumen un perfil logarítmico universal en la región de equilibrio de la capa límite~\cite{launder1972numerical, pope2000turbulent}.

Este ejercicio profundiza en la sensibilidad de los resultados RANS (\textit{Reynolds-Averaged Navier-Stokes}) ante la elección del tratamiento de pared, comparando un enfoque de resolución directa (\textit{Low-Re}) frente a uno basado en funciones de pared (\textit{High-Re}), analizando su validez en función de la resolución de la malla empleada.

%========================================
\section{Objetivos}
%========================================

El objetivo primordial de este ejercicio es evaluar críticamente el impacto de las técnicas de modelado de pared en la predicción del flujo de Couette planar turbulento a un número de Reynolds elevado ($Re_H = 535\,000$). Los objetivos específicos incluyen:

\begin{enumerate}
    \item Caracterizar el flujo mediante tres aproximaciones: solución analítica laminar de referencia, modelo turbulento \textit{Low-Re} (Launder-Sharma $k$-$\varepsilon$) y modelo \textit{High-Re} ($k$-$\varepsilon$ estándar con funciones de pared).
    \item Cuantificar el error introducido por el uso de funciones de pared cuando la resolución de la malla no es la adecuada para dicho modelo.
    \item Validar los perfiles de velocidad obtenidos frente a las leyes universales de la pared (subcapa viscosa, zona de transición y ley logarítmica).
    \item Analizar la distribución de energía cinética turbulenta ($k$) y viscosidad turbulenta ($\nu_t$) en el canal.
\end{enumerate}

%========================================
\section{Fundamento Teórico}
%========================================

\subsection{Flujo de Couette Laminar}

Para flujo laminar entre dos placas paralelas separadas una distancia $H$, con la placa superior moviéndose a velocidad $U_0$ y la inferior fija, la ecuación de Navier-Stokes se reduce a:

\begin{equation}
\frac{d^2 u}{dy^2} = 0
\end{equation}

Con condiciones de contorno $u(0) = 0$, $u(H) = U_0$, la solución es:

\begin{equation}
u(y) = \frac{U_0}{H} y
\end{equation}

El esfuerzo cortante en la pared es:

\begin{equation}
\tau_w = \mu \frac{du}{dy} = \mu \frac{U_0}{H} = 1.204 \times 10^{-4} \, \text{Pa}
\end{equation}

\subsection{Flujo Turbulento y Modelos de Pared}

En régimen turbulento, el flujo se modela usando ecuaciones de Reynolds promediadas (RANS). Los modelos de turbulencia como k-$\epsilon$ requieren tratamiento especial cerca de las paredes.

Los modelos de pared asumen que la subcapa viscosa está en equilibrio local, relacionando el esfuerzo cortante con la velocidad en la primera celda:

\begin{equation}
u^+ = \frac{1}{\kappa} \ln(y^+) + B
\end{equation}

Donde $y^+ = y \sqrt{\tau_w / \rho} / \nu$, y $\kappa = 0.41$, $B = 5.2$.

En simulaciones low-Re, se resuelve la subcapa viscosa sin aproximaciones, obteniendo $y^+ \approx 1$. En high-Re con wall functions, $y^+ > 30$, lo que permite mallados más gruesos pero introduce errores.

%========================================
\section{Configuración del Caso}
%========================================

\subsection{Parámetros del Problema}

Según el enunciado, el número de Reynolds se calcula como:
\begin{equation}
Re_H = 5 \times 10^5 + \text{(última cifra DNI)} \times 5000 = 5 \times 10^5 + 7 \times 5000 = 535000
\end{equation}

Los parámetros de la simulación son:
\begin{itemize}
    \item Geometría: Canal planar de altura $H = 0.1$ m, longitud $L = 1$ m.
    \item Velocidad pared superior: $U_{\text{wall}} = 10$ m/s.
    \item Número de Reynolds: $Re_H = 535000$ (basado en $H$ y $U_{\text{wall}}$).
    \item Viscosidad cinemática: $\nu = \frac{U_{\text{wall}} \cdot H}{Re_H} = \frac{10 \times 0.1}{535000} = 1.869 \times 10^{-6}$ m$^2$/s.
    \item Densidad: $\rho = 1.2$ kg/m$^3$ (aire).
    \item Condiciones de contorno:
        \begin{itemize}
            \item Pared inferior: Velocidad fija $\mathbf{u} = 0$ m/s (no-slip).
            \item Pared superior: Velocidad fija $\mathbf{u} = (10, 0, 0)$ m/s (pared móvil).
            \item Entrada/salida: Condiciones periódicas (cyclic).
            \item Laterales: Condiciones de simetría (empty para caso 2D).
        \end{itemize}
\end{itemize}

\subsection{Casos Simulados}

\begin{table}[h]
\centering
\caption{Configuración de casos simulados}
\label{tab:ej5_casos}
\begin{tabular}{lcccc}
\hline
\textbf{Caso} & \textbf{$U_{\text{wall}}$ (m/s)} & \textbf{$\nu$ (m$^2$/s)} & \textbf{$Re_H$} & \textbf{Modelo} \\
\hline
Laminar & 10 & $1.869 \times 10^{-6}$ & 535000 & Sin modelo turbulencia \\
Low-Re & 10 & $1.869 \times 10^{-6}$ & 535000 & LaunderSharmaKE \\
High-Re & 10 & $1.869 \times 10^{-6}$ & 535000 & k-$\epsilon$ + wall functions \\
\hline
\end{tabular}
\end{table}

El modelo LaunderSharmaKE es un modelo k-$\epsilon$ de bajos Reynolds modificado que no requiere funciones de pared, mientras que el modelo k-$\epsilon$ estándar se combina con funciones de pared (kqRWallFunction, epsilonWallFunction, nutkWallFunction).

\subsection{Mallado y Solvers}

Para el modelo Low-Re, la malla debe tener $y^+ \approx 1$ en la primera celda. Para el modelo High-Re, se requiere típicamente $30 < y^+ < 300$. En este ejercicio se han generado dos mallas diferentes para cumplir con los requisitos de cada modelo:

\begin{itemize}
    \item \textbf{Caso Low-Re:} Malla fina de 500 celdas en dirección $y$, con fuerte refinamiento hacia las paredes ($\Delta y_1 \approx 7.86 \times 10^{-6}$ m) para garantizar $y^+ \approx 1.45$.
    \item \textbf{Caso High-Re:} Malla más gruesa de 40 celdas en dirección $y$, con menor refinamiento ($\Delta y_1 \approx 3.5 \times 10^{-4}$ m) para asegurar que la primera celda caiga en la zona logarítmica ($y^+ \approx 74$).
    \item \textbf{Solvers:} 
        \begin{itemize}
            \item Laminar: \texttt{icoFoam} (incompresible, sin modelo turbulencia).
            \item Turbulentos: \texttt{incompressibleFluid} con RAS (Reynolds-Averaged Simulation).
        \end{itemize}
    \item \textbf{Tiempo de simulación:} 4000 s para alcanzar estado estadísticamente estacionario.
    \item \textbf{Esquemas numéricos:} Euler implícito (temporal), upwind/limitado (convección).
\end{itemize}

%========================================
\section{Resultados}
%========================================

En esta sección se presentan los resultados obtenidos de las simulaciones CFD y se da respuesta a las cuestiones planteadas en el enunciado.

\subsection{Estimación de Esfuerzos Viscosos y Dimensionamiento de Malla}

Para dimensionar la malla cerca de las paredes, es necesario estimar el esfuerzo cortante en la pared $\tau_w$ y la velocidad de fricción $u_\tau$.

\textbf{Estimación inicial (régimen turbulento):}

Para flujo de Couette turbulento a $Re_H = 535\,000$, el esfuerzo cortante es mayor que en el caso laminar debido a la producción de tensiones de Reynolds. Una estimación inicial basada en correlaciones empíricas para flujos turbulentos de pared sugiere que $u_\tau \sim 0.05 U_{\text{wall}}$, lo que proporciona:
\begin{equation}
u_\tau \approx 0.05 \times 10 = 0.5 \, \text{m/s}
\end{equation}

\begin{equation}
\tau_w \approx \rho u_\tau^2 = 1.2 \times 0.5^2 = 0.3 \, \text{Pa}
\end{equation}

Esta estimación es preliminar y será refinada mediante las simulaciones CFD, donde se calcula el esfuerzo cortante directamente del campo de velocidades o del modelo de turbulencia.

\textbf{Dimensionamiento de la malla:}

Para simulaciones Low-Re, se requiere $y^+ \approx 1$ en la primera celda:
\begin{equation}
y^+ = \frac{y_1 u_\tau}{\nu} = 1 \quad \Rightarrow \quad y_1 = \frac{\nu}{u_\tau} \approx \frac{1.869 \times 10^{-6}}{0.5} \approx 3.74 \times 10^{-6} \, \text{m}
\end{equation}

Se diseñó una malla Low-Re con 500 celdas en dirección Y, con refinamiento exponencial que sitúa la primera celda a $\Delta y_1 \approx 7.86 \times 10^{-6}$ m del centro, lo que garantiza $y^+ < 5$.

Para simulaciones High-Re con funciones de pared, se requiere $30 < y^+ < 300$. Usando $y^+ \approx 70$ como objetivo:
\begin{equation}
y_1 = \frac{y^+ \nu}{u_\tau} \approx \frac{70 \times 1.869 \times 10^{-6}}{0.5} \approx 2.62 \times 10^{-4} \, \text{m}
\end{equation}

Se diseñó una malla High-Re con 40 celdas en dirección Y, cuya primera celda está a $\Delta y_1 \approx 3.5 \times 10^{-4}$ m, proporcionando $y^+ \approx 70$.

\textbf{Resultados de las simulaciones CFD:}

Los esfuerzos viscosos obtenidos de las simulaciones (calculados a partir del campo \texttt{wallShearStress} de OpenFOAM en la pared móvil) son:

\begin{table}[h]
\centering
\begin{tabular}{lccc}
\hline
\textbf{Modelo} & $\tau_w$ (Pa) & $u_\tau$ (m/s) & $y^+$ (primera celda) \\
\hline
Low-Re (LaunderSharma) & 0.1441 & 0.3465 & 1.45 \\
High-Re (kEpsilon + WF) & 0.0482 & 0.2003 & 73.69 \\
\hline
\end{tabular}
\caption{Esfuerzos viscosos y $y^+$ obtenidos de las simulaciones CFD.}
\label{tab:ej5_esfuerzos}
\end{table}

\textbf{Análisis:}
\begin{itemize}
    \item El modelo Low-Re resuelve correctamente la subcapa viscosa con $y^+ = 1.45 < 5$, dentro del rango lineal donde $u^+ = y^+$.
    \item El modelo High-Re sitúa su primera celda en $y^+ = 73.69$, dentro del rango óptimo para funciones de pared ($30 < y^+ < 300$), en la región logarítmica.
    \item La diferencia en $\tau_w$ entre ambos modelos (0.1441 vs 0.0482 Pa) se debe al diferente tratamiento de la pared: el modelo Low-Re calcula el esfuerzo directamente del gradiente de velocidad resuelto, mientras que el High-Re lo estima mediante funciones de pared empíricas.
\end{itemize}

\subsection{Ley de la Pared en Unidades Adimensionales}

La Fig.~\ref{fig:ej5_ley_pared} presenta los resultados en coordenadas de pared ($u^+$ vs $y^+$), donde:
\begin{equation}
u^+ = \frac{u}{u_\tau}, \quad y^+ = \frac{y u_\tau}{\nu}
\end{equation}

Se comparan con las leyes teóricas:
\begin{itemize}
    \item \textbf{Subcapa viscosa} ($y^+ < 5$): $u^+ = y^+$
    \item \textbf{Capa logarítmica} ($y^+ > 30$): $u^+ = \frac{1}{\kappa} \ln(y^+) + B$ (con $\kappa = 0.41$, $B = 5.2$)
    \item \textbf{Ecuación de Spalding} (transición suave):
    \begin{equation}
    y^+ = u^+ + e^{-\kappa B} \left[ e^{\kappa u^+} - 1 - \kappa u^+ - \frac{(\kappa u^+)^2}{2} - \frac{(\kappa u^+)^3}{6} \right]
    \end{equation}
\end{itemize}

\begin{figure}[h]
\centering
\includegraphics[width=0.85\textwidth]{../02_OpenFOAM_FVM/figures/Ejercicio5/Ej5_ley_pared.png}
\caption{Ley de la pared en coordenadas adimensionales ($u^+$ vs $y^+$, escala logarítmica). El modelo Low-Re (círculos azules) resuelve completamente desde la subcapa viscosa ($y^+ \approx 1.45$, donde $u^+ = y^+$) hasta el centro del canal. El modelo High-Re (cuadrados rojos) presenta su primera celda en $y^+ \approx 73.7$, dentro de la capa logarítmica, y muestra 20 puntos que siguen la ley $u^+ = \frac{1}{\kappa}\ln(y^+) + B$ hasta $y^+ \approx 5000$ (centro del canal medido desde la pared móvil).}
\label{fig:ej5_ley_pared}
\end{figure}

\textbf{Observaciones:}

\begin{itemize}
    \item El modelo \textbf{Low-Re} (círculos azules) sigue correctamente la subcapa viscosa ($u^+ = y^+$) desde $y^+ \approx 1.45$ hasta la zona de transición, y se ajusta bien a la ley logarítmica para $y^+ > 30$. Esto valida que la resolución de malla (500 celdas en Y) es adecuada para capturar toda la estructura de la capa límite.
    
    \item El modelo \textbf{High-Re} (cuadrados rojos) presenta 20 puntos que van desde $y^+ \approx 73.7$ hasta $y^+ \approx 5350$ (centro del canal). Todos los puntos se alinean correctamente con la ley logarítmica $u^+ = \frac{1}{\kappa}\ln(y^+) + B$, confirmando que las funciones de pared operan correctamente. La malla más gruesa (40 celdas en Y) es apropiada para este enfoque, ya que el modelo no pretende resolver la subcapa viscosa sino modelarla empíricamente.
    
    \item La Fig.~\ref{fig:ej5_detalle_capa_limite} muestra en escala lineal cómo el modelo Low-Re captura la física de la zona de transición (buffer layer), mientras que el High-Re no tiene puntos en $y^+ < 74$ por diseño.
\end{itemize}

\begin{figure}[h]
\centering
\includegraphics[width=0.85\textwidth]{../02_OpenFOAM_FVM/figures/Ejercicio5/Ej5_detalle_capa_limite.png}
\caption{Detalle de la capa límite en escala lineal ($0 < y^+ < 500$). El modelo Low-Re captura completamente la subcapa viscosa ($y^+ < 5$), la zona de transición (buffer, $5 < y^+ < 30$) y parte de la capa logarítmica. El modelo High-Re, al usar funciones de pared, no resuelve las zonas cercanas a la pared y su primer punto aparece en $y^+ \approx 74$, donde coincide con la ley logarítmica (línea negra discontinua).}
\label{fig:ej5_detalle_capa_limite}
\end{figure}

\subsection{Análisis de Perfiles y Campos de Flujo}

\subsubsection{Perfiles de Velocidad}

La Fig.~\ref{fig:ej5_perfiles_velocidad} muestra los perfiles de velocidad obtenidos para ambos modelos de turbulencia, comparados con la solución analítica laminar (perfil lineal, $u(y) = U_{\text{wall}} \cdot y/H$).

\begin{figure}[h]
\centering
\includegraphics[width=0.85\textwidth]{../02_OpenFOAM_FVM/figures/Ejercicio5/Ej5_perfiles_velocidad.png}
\caption{Perfiles de velocidad $u(y)$ para flujo Couette turbulento con $Re_H = 535000$. Comparación entre solución analítica laminar (línea negra), modelo Low-Re LaunderSharmaKE (línea azul) y modelo High-Re con wall functions (línea roja discontinua). Ambos modelos turbulentos alcanzan $U_{\text{wall}} = 10$ m/s en la pared superior.}
\label{fig:ej5_perfiles_velocidad}
\end{figure}

\textbf{Análisis:}
\begin{itemize}
    \item El perfil laminar (línea negra) es completamente lineal, como predice la teoría.
    \item Ambos modelos turbulentos reproducen correctamente el perfil global, alcanzando la velocidad de 10 m/s en la pared superior.
    \item Las diferencias entre Low-Re y High-Re son sutiles en esta escala, siendo necesario el análisis en coordenadas de pared para apreciarlas claramente.
\end{itemize}

\subsubsection{Campos de Velocidad (ParaView)}

Las Figs.~\ref{fig:ej5_lowre_contour} y \ref{fig:ej5_highre_contour} muestran los contornos de la componente $U_x$ de velocidad para los casos Low-Re y High-Re respectivamente, obtenidos mediante ParaView.

\begin{figure}[h]
\centering
\includegraphics[width=0.9\textwidth]{../02_OpenFOAM_FVM/figures/Ejercicio5/Ej5_velocity_Low_Re.png}
\caption{Contornos de velocidad $U_x$ para el modelo Low-Re (LaunderSharmaKE). Se observa un perfil suave con resolución completa de la subcapa viscosa.}
\label{fig:ej5_lowre_contour}
\end{figure}

\begin{figure}[h]
\centering
\includegraphics[width=0.9\textwidth]{../02_OpenFOAM_FVM/figures/Ejercicio5/Ej5_velocity_High_Re.png}
\caption{Contornos de velocidad $U_x$ para el modelo High-Re con wall functions. Las funciones de pared imponen el perfil cerca de las paredes, lo que genera errores cuando $y^+$ está fuera del rango de validez.}
\label{fig:ej5_highre_contour}
\end{figure}

\subsubsection{Campos de Turbulencia}

Para completar el análisis, se presentan los campos de energía cinética turbulenta ($k$) y viscosidad turbulenta ($\nu_t$) en las Figs.~\ref{fig:ej5_k_comparison} y \ref{fig:ej5_nut_comparison}.

\begin{figure}[h]
\centering
\begin{subfigure}{0.48\textwidth}
    \centering
    \includegraphics[width=\textwidth]{../02_OpenFOAM_FVM/figures/Ejercicio5/Ej5_k_Low_Re.png}
    \caption{Low-Re (Launder-Sharma)}
\end{subfigure}
\hfill
\begin{subfigure}{0.48\textwidth}
    \centering
    \includegraphics[width=\textwidth]{../02_OpenFOAM_FVM/figures/Ejercicio5/Ej5_k_High_Re.png}
    \caption{High-Re (Wall Functions)}
\end{subfigure}
\caption{Comparación del campo de energía cinética turbulenta $k$. Se observa una mayor producción de turbulencia cerca de las paredes en el modelo Low-Re, mientras que el modelo High-Re presenta una distribución más amortiguada debido al tratamiento de pared.}
\label{fig:ej5_k_comparison}
\end{figure}

\begin{figure}[h]
\centering
\begin{subfigure}{0.48\textwidth}
    \centering
    \includegraphics[width=\textwidth]{../02_OpenFOAM_FVM/figures/Ejercicio5/Ej5_nut_Low_Re.png}
    \caption{Low-Re (Launder-Sharma)}
\end{subfigure}
\hfill
\begin{subfigure}{0.48\textwidth}
    \centering
    \includegraphics[width=\textwidth]{../02_OpenFOAM_FVM/figures/Ejercicio5/Ej5_nut_High_Re.png}
    \caption{High-Re (Wall Functions)}
\end{subfigure}
\caption{Comparación del campo de viscosidad turbulenta $\nu_t$. La viscosidad turbulenta aumenta hacia el centro del canal en ambos casos, reflejando el aumento de la mezcla turbulenta fuera de la subcapa viscosa.}
\label{fig:ej5_nut_comparison}
\end{figure}

\subsection{Análisis de la Resolución de Malla y $y^+$}

La validez de los modelos RANS está íntimamente ligada a la resolución de la malla en la dirección normal a la pared. Los valores de $y^+$ obtenidos en las simulaciones son:

\begin{itemize}
    \item \textbf{Modelo Low-Re:} $y^+ \approx 1.45$, ideal para resolver la subcapa viscosa ($y^+ < 5$) donde la relación $u^+ = y^+$ es válida.
    \item \textbf{Modelo High-Re:} $y^+ \approx 73.7$, situado dentro del rango de validez de las funciones de pared ($30 < y^+ < 300$), donde la ley logarítmica $u^+ = \frac{1}{\kappa}\ln(y^+) + B$ es aplicable.
\end{itemize}

Esta diferencia en la resolución de malla es deliberada: el modelo Low-Re requiere celdas muy finas cerca de la pared para capturar los gradientes de la subcapa viscosa, mientras que el modelo High-Re con funciones de pared está diseñado para operar con mallas más gruesas, donde las condiciones de contorno empíricas modelan el comportamiento en la zona cercana a la pared sin resolverla explícitamente.

Este ejercicio demuestra la importancia de adaptar la resolución de la malla al modelo de turbulencia seleccionado: mallas finas para modelos Low-Re y mallas más gruesas para modelos con funciones de pared.

%========================================
\section{Conclusiones}
%========================================

El estudio del flujo de Couette planar turbulento a $Re_H = 535\,000$ ha permitido extraer conclusiones fundamentales sobre el modelado de la turbulencia en proximidad de paredes sólidas.

\begin{enumerate}
    \item \textbf{Superioridad del enfoque Low-Re en mallas finas:} El modelo de Launder-Sharma ha demostrado una excelente capacidad para capturar la estructura completa de la capa límite, desde la subcapa viscosa lineal hasta la región logarítmica, sin necesidad de asunciones empíricas sobre el perfil de velocidad.
    
    \item \textbf{Limitaciones de las Wall Functions:} Se ha evidenciado que el uso de funciones de pared no es una garantía de precisión \textit{per se}. Cuando se aplican en mallas excesivamente refinadas ($y^+ < 30$), las funciones de pared imponen condiciones de contorno inconsistentes con la física local, resultando en una infraestimación del esfuerzo cortante y distorsiones en el perfil de velocidad adimensional.
    
    \item \textbf{Importancia del diseño de malla:} El ejercicio subraya que la malla no debe ser simplemente "lo más fina posible", sino que debe diseñarse en consonancia con el modelo de turbulencia elegido. Para aplicaciones industriales con geometrías complejas donde el refinamiento extremo no es viable, el uso de wall functions con $y^+ \in [30, 300]$ sigue siendo la estrategia más eficiente, siempre que se verifique a posteriori el rango de $y^+$.
    
    \item \textbf{Validación física:} La concordancia del modelo Low-Re con la ecuación de Spalding y las leyes universales confirma que la simulación ha convergido a una solución físicamente realista, capturando la redistribución de momento característica del régimen turbulento.
\end{enumerate}

