%========================================
% CAPITULO 4: ESQUEMAS NUMERICOS - FVM Y TUBO DE CHOQUE
%========================================
\chapter{Esquemas Numéricos: Método de Volúmenes Finitos}
\label{chap:ejercicio4}

%========================================
\section*{Introducción}
\addcontentsline{toc}{section}{Introducción}
%========================================

La discretización espacial de las ecuaciones de conservación en el método de volúmenes finitos (FVM) constituye uno de los aspectos fundamentales que determinan la precisión y estabilidad de las soluciones numéricas en CFD. La elección del esquema numérico para aproximar los flujos convectivos en las caras de los volúmenes de control tiene un impacto directo en la capacidad del método para resolver correctamente fenómenos con gradientes pronunciados, discontinuidades y ondas de choque~\cite{blazek2015cfd,versteeg2007introduction}.

Este capítulo aborda el estudio de los esquemas numéricos desde dos perspectivas complementarias:

\begin{itemize}
    \item \textbf{Sección~\ref{sec:fvm_matlab}:} Implementación del método de volúmenes finitos en MATLAB para resolver la ecuación de convección-difusión unidimensional, verificando los resultados con la solución analítica exacta.
    \item \textbf{Sección~\ref{sec:shocktube}:} Simulación del tubo de choque de Sod en OpenFOAM, comparando esquemas de bajo orden (\textit{upwind}) y alto orden (\textit{vanAlbada} TVD) para evaluar su capacidad de captura de discontinuidades.
\end{itemize}

%========================================
%========================================
\section{Método de Volúmenes Finitos en MATLAB}
\label{sec:fvm_matlab}
%========================================
%========================================

%========================================
\subsection{Objetivo}
%========================================

El objetivo de esta sección es implementar el método de volúmenes finitos (FVM) para resolver la ecuación de convección-difusión unidimensional en régimen estacionario:

\begin{equation}
    \frac{d(\rho u \phi)}{dx} = \frac{d}{dx}\left(\Gamma \frac{d\phi}{dx}\right)
    \label{eq:convdiff}
\end{equation}

donde $\phi$ es la variable transportada, $\rho$ la densidad, $u$ la velocidad de convección y $\Gamma$ el coeficiente de difusión.

Los objetivos específicos son:
\begin{itemize}
    \item Discretizar la ecuación~\eqref{eq:convdiff} mediante el esquema \textit{upwind} para el término convectivo y diferencias centrales para el difusivo.
    \item Verificar la implementación comparando con los valores de referencia del enunciado.
    \item Analizar el efecto del número de Péclet ($Pe = \rho u L / \Gamma$) en la solución.
    \item Estudiar la convergencia de malla del esquema numérico.
\end{itemize}

%========================================
\subsection{Fundamento teórico: Discretización FVM}
%========================================

\subsubsection{Formulación integral}

El método de volúmenes finitos parte de la forma integral de la ecuación de conservación. Integrando la Ec.~\eqref{eq:convdiff} sobre un volumen de control $\Delta V$ y aplicando el teorema de Gauss:

\begin{equation}
    \int_S (\rho u \phi) \cdot dS = \int_S \left(\Gamma \frac{d\phi}{dx}\right) \cdot dS
\end{equation}

Para una celda 1D de longitud $\Delta x$, esto se reduce a:

\begin{equation}
    (\rho u \phi)_e - (\rho u \phi)_w = \left(\Gamma \frac{d\phi}{dx}\right)_e - \left(\Gamma \frac{d\phi}{dx}\right)_w
    \label{eq:fvm_discreto}
\end{equation}

donde los subíndices $e$ y $w$ denotan las caras este y oeste del volumen de control.

\subsubsection{Esquema Upwind}

Para evaluar el flujo convectivo en las caras, el esquema \textit{upwind} utiliza el valor de la celda aguas arriba:

\begin{equation}
    \phi_e = \begin{cases}
        \phi_P & \text{si } u > 0 \\
        \phi_E & \text{si } u < 0
    \end{cases}
\end{equation}

Este esquema garantiza estabilidad pero introduce difusión numérica de orden $O(\Delta x)$.

\subsubsection{Sistema lineal resultante}

La discretización genera un sistema tridiagonal $A\phi = b$ con coeficientes:

\begin{equation}
    a_W \phi_W + a_P \phi_P + a_E \phi_E = S_u
\end{equation}

donde los coeficientes dependen del flujo convectivo $F = \rho u$ y el coeficiente difusivo $D = \Gamma/\Delta x$.

%========================================
\subsection{Configuración del problema}
%========================================

Se resolvió el problema de convección-difusión con los siguientes parámetros:

\begin{itemize}
    \item Longitud del dominio: $L = 1$ m
    \item Densidad: $\rho = 1$ kg/m$^3$
    \item Coeficiente de difusión: $\Gamma = 0.1$ kg/(m$\cdot$s)
    \item Condiciones de contorno: $\phi(0) = 1$, $\phi(L) = 0$
\end{itemize}

Se analizaron tres casos con diferentes velocidades y refinamientos de malla:

\begin{table}[h!]
\centering
\caption{Casos de estudio para el problema de convección-difusión}
\label{tab:fvm_casos}
\begin{tabular}{lccc}
\hline
\textbf{Caso} & \textbf{Velocidad} $u$ [m/s] & \textbf{Celdas} $N$ & \textbf{Péclet} $Pe$ \\
\hline
1 & 0.1 & 5 & 1.0 \\
2 & 2.5 & 5 & 25.0 \\
3 & 2.5 & 20 & 25.0 \\
\hline
\end{tabular}
\end{table}

La solución analítica exacta de la Ec.~\eqref{eq:convdiff} es:

\begin{equation}
    \phi(x) = \phi_A + (\phi_B - \phi_A) \frac{e^{Pe \cdot x/L} - 1}{e^{Pe} - 1}
    \label{eq:sol_analitica}
\end{equation}

%========================================
\subsection{Resultados}
%========================================

\subsubsection{Verificación del Caso 1}

La Fig.~\ref{fig:fvm_caso1} presenta la comparación entre la solución numérica FVM y la solución analítica para el Caso 1 ($u = 0.1$ m/s, $N = 5$ celdas, $Pe = 1$).

\begin{figure}[h!]
    \centering
    \includegraphics[width=0.85\textwidth]{Ejercicio4/FVM_caso1_validacion.png}
    \caption{Caso 1: Validación de la implementación FVM con $u = 0.1$ m/s y $N = 5$ celdas. El número de Péclet $Pe = 1$ indica un balance entre convección y difusión.}
    \label{fig:fvm_caso1}
\end{figure}

Con $Pe = 1$, la convección y difusión tienen magnitudes comparables, resultando en un perfil suave que el esquema \textit{upwind} captura con buena precisión. Los valores numéricos coinciden con los de referencia del enunciado dentro del error de redondeo.

\subsubsection{Efecto de la velocidad}

La Fig.~\ref{fig:fvm_velocidad} muestra el efecto del incremento de la velocidad de convección para una malla fija de $N = 5$ celdas.

\begin{figure}[h!]
    \centering
    \includegraphics[width=\textwidth]{Ejercicio4/FVM_efecto_velocidad.png}
    \caption{Efecto de la velocidad de convección en la solución FVM. Al aumentar $Pe$, el perfil se vuelve más pronunciado y el esquema \textit{upwind} introduce mayor difusión numérica.}
    \label{fig:fvm_velocidad}
\end{figure}

Se observa que al aumentar el número de Péclet:
\begin{itemize}
    \item El perfil de $\phi$ se vuelve más abrupto, concentrándose la variación cerca del contorno de salida ($x = L$).
    \item La difusión numérica del esquema \textit{upwind} suaviza los gradientes, alejando la solución numérica de la analítica.
    \item Para $Pe = 25$, la discrepancia es significativa con solo 5 celdas.
\end{itemize}

\subsubsection{Efecto del refinamiento de malla}

La Fig.~\ref{fig:fvm_refinamiento} compara las soluciones con $N = 5$ y $N = 20$ celdas para el caso de alta convección ($u = 2.5$ m/s).

\begin{figure}[h!]
    \centering
    \includegraphics[width=\textwidth]{Ejercicio4/FVM_efecto_refinamiento.png}
    \caption{Efecto del refinamiento de malla: con $N = 20$ celdas, la solución FVM se aproxima mejor a la analítica al reducir la difusión numérica.}
    \label{fig:fvm_refinamiento}
\end{figure}

El refinamiento de malla reduce la difusión numérica proporcional a $\Delta x$, mejorando significativamente la aproximación a la solución exacta. Esto confirma la convergencia de primer orden del esquema \textit{upwind}.

\subsubsection{Resumen}

La Fig.~\ref{fig:fvm_resumen} presenta un resumen comparativo de los tres casos analizados.

\begin{figure}[h!]
    \centering
    \includegraphics[width=\textwidth]{Ejercicio4/FVM_resumen.png}
    \caption{Resumen de los tres casos de estudio FVM, mostrando la evolución de la solución con diferentes números de Péclet y refinamientos de malla.}
    \label{fig:fvm_resumen}
\end{figure}

%========================================
%========================================
\section{Tubo de Choque de Sod en OpenFOAM}
\label{sec:shocktube}
%========================================
%========================================

%========================================
\subsection{Objetivo}
%========================================

El objetivo de esta sección es analizar el efecto de los esquemas de discretización espacial en flujos compresibles mediante la simulación del problema del tubo de choque de Sod en OpenFOAM, respondiendo a las siguientes cuestiones:

\begin{enumerate}
    \item ¿Qué es un tubo de choque de Sod y para qué se utiliza?
    \item Simular el problema con esquemas de alto orden y validar con la solución analítica a $t = 0.1$ s.
    \item Explicar las fórmulas matemáticas de los esquemas upwind y vanAlbada, con sus ventajas e inconvenientes.
    \item Comparar los resultados de alto y bajo orden mediante perfiles horizontales a $t = 0.15$ s.
\end{enumerate}

%========================================
\subsection{El tubo de choque de Sod: Definición y aplicaciones}
\label{subsec:sod_definicion}
%========================================

El tubo de choque de Sod~\cite{sod1978survey} es un problema clásico de dinámica de gases que constituye un \textit{benchmark} fundamental para la validación de códigos de CFD compresible. Consiste en un tubo cerrado dividido inicialmente por una membrana que separa dos estados termodinámicos distintos:

\begin{itemize}
    \item \textbf{Estado izquierdo ($x < 0$):} Gas a alta presión y densidad.
    \item \textbf{Estado derecho ($x > 0$):} Gas a baja presión y densidad.
\end{itemize}

Al eliminar instantáneamente la membrana en $t = 0$, la discontinuidad inicial se descompone en tres ondas características que constituyen la solución del problema de Riemann:

\begin{enumerate}
    \item \textbf{Onda de rarefacción:} Se propaga hacia la izquierda como una expansión isentrópica continua que reduce gradualmente la presión y densidad del gas. A diferencia de las discontinuidades, la onda de rarefacción es una transición suave (abanico de expansión).

    \item \textbf{Discontinuidad de contacto:} Superficie material que separa fluidos de diferente densidad pero igual presión y velocidad. No hay transporte de masa a través de ella, solo advección de la interfaz.

    \item \textbf{Onda de choque:} Discontinuidad que incrementa bruscamente la presión, densidad y temperatura, propagándose hacia la derecha a velocidad supersónica respecto al gas en reposo.
\end{enumerate}

\textbf{Utilidad del problema de Sod:}
\begin{itemize}
    \item Posee solución analítica exacta, permitiendo validación cuantitativa de solvers numéricos.
    \item Contiene las tres estructuras fundamentales de flujos compresibles: expansiones, discontinuidades de contacto y ondas de choque.
    \item Permite evaluar la difusión numérica de los esquemas al capturar frentes discontinuos.
    \item Es computacionalmente económico al ser unidimensional.
\end{itemize}

La Fig.~\ref{fig:shock_xt} muestra el diagrama $x$-$t$ conceptual con las líneas características de las tres ondas, identificando las cuatro regiones del flujo.

\begin{figure}[h!]
    \centering
    \includegraphics[width=0.75\textwidth]{Ejercicio4/shocktube_diagrama_xt.png}
    \caption{Diagrama $x$-$t$ del tubo de choque de Sod mostrando la propagación de las ondas características: onda de rarefacción (azul), discontinuidad de contacto (verde) y onda de choque (rojo). Las cuatro regiones corresponden a: (1) gas izquierdo sin perturbar, (2) gas expandido por la rarefacción, (3) gas comprimido entre contacto y choque, (4) gas derecho sin perturbar. El diafragma inicial se sitúa en $x = 0$.}
    \label{fig:shock_xt}
\end{figure}

\subsubsection{Física del problema}

La evolución temporal del sistema se rige por las ecuaciones de Euler para flujos compresibles:
\begin{equation}
    \frac{\partial \mathbf{U}}{\partial t} + \frac{\partial \mathbf{F}}{\partial x} = 0
\end{equation}
donde el vector de variables conservativas y el vector de flujos son:
\begin{equation}
    \mathbf{U} = \begin{pmatrix} \rho \\ \rho u \\ E \end{pmatrix}, \quad
    \mathbf{F} = \begin{pmatrix} \rho u \\ \rho u^2 + p \\ u(E + p) \end{pmatrix}
\end{equation}

La velocidad de propagación de cada onda viene determinada por las velocidades características del sistema:
\begin{itemize}
    \item \textbf{Onda de rarefacción:} Se propaga a velocidades entre $u - a$ (cabeza) y $u^* - a^*$ (cola), donde $a = \sqrt{\gamma p / \rho}$ es la velocidad del sonido.
    \item \textbf{Discontinuidad de contacto:} Se mueve con la velocidad del fluido $u^*$ en la región intermedia.
    \item \textbf{Onda de choque:} Velocidad supersónica respecto al gas en reposo, calculable mediante las relaciones de Rankine-Hugoniot.
\end{itemize}

%========================================
\subsection{Esquemas numéricos: Formulación matemática}
\label{subsec:esquemas_formulas}
%========================================

\subsubsection{Esquema Upwind (primer orden)}

El esquema \textit{upwind} evalúa el flujo convectivo en las caras utilizando el valor de la celda situada aguas arriba respecto a la dirección del flujo:

\begin{equation}
    \phi_f = \begin{cases}
        \phi_P & \text{si } \mathbf{u} \cdot \mathbf{n} > 0 \\
        \phi_N & \text{si } \mathbf{u} \cdot \mathbf{n} < 0
    \end{cases}
    \label{eq:upwind}
\end{equation}

donde $\phi_P$ es el valor en la celda propietaria, $\phi_N$ en la celda vecina, y $\mathbf{n}$ es el vector normal a la cara.

Mediante análisis de Taylor, el esquema upwind aproxima la derivada espacial con:
\begin{equation}
    \frac{\partial \phi}{\partial x} \approx \frac{\phi_P - \phi_W}{\Delta x} + O(\Delta x)
\end{equation}

\textbf{Ventajas:}
\begin{itemize}
    \item \textbf{Estabilidad incondicional:} No genera oscilaciones espurias cerca de discontinuidades.
    \item \textbf{Monotonicidad:} Preserva el principio del máximo, evitando valores no físicos.
    \item \textbf{Simplicidad:} Fácil implementación y bajo coste computacional.
\end{itemize}

\textbf{Inconvenientes:}
\begin{itemize}
    \item \textbf{Alta difusión numérica:} Introduce un término difusivo artificial proporcional a $\frac{1}{2}|u|\Delta x$, extendiendo las discontinuidades sobre 6--10 celdas.
    \item \textbf{Primer orden de precisión:} Error de truncación $O(\Delta x)$, requiriendo mallas muy finas para precisión aceptable.
\end{itemize}

\subsubsection{Esquema vanAlbada (TVD de alto orden)}

Los esquemas TVD (\textit{Total Variation Diminishing}) combinan alta precisión con estabilidad mediante el uso de limitadores de flujo. El limitador vanAlbada tiene la forma:

\begin{equation}
    \psi(r) = \frac{r^2 + r}{r^2 + 1}
    \label{eq:vanalbada}
\end{equation}

donde $r$ es el cociente de gradientes consecutivos:
\begin{equation}
    r = \frac{\phi_P - \phi_W}{\phi_E - \phi_P}
\end{equation}

El flujo en la cara se reconstruye como:
\begin{equation}
    \phi_f = \phi_P + \frac{1}{2}\psi(r)(\phi_E - \phi_P)
\end{equation}

El limitador vanAlbada satisface las condiciones TVD de Sweby, garantizando que:
\begin{itemize}
    \item Para $r \to 0$ (discontinuidad): $\psi \to 0$, reduciendo al esquema upwind.
    \item Para $r \to 1$ (región suave): $\psi \to 1$, recuperando precisión de segundo orden.
\end{itemize}

\textbf{Ventajas:}
\begin{itemize}
    \item \textbf{Segundo orden en regiones suaves:} Reduce significativamente la difusión numérica.
    \item \textbf{Captura de discontinuidades:} Resuelve frentes de choque en 2--4 celdas.
    \item \textbf{Libre de oscilaciones:} Evita la generación de valores no físicos (wiggles).
\end{itemize}

\textbf{Inconvenientes:}
\begin{itemize}
    \item \textbf{Mayor coste computacional:} Requiere evaluar gradientes y el limitador en cada cara.
    \item \textbf{Stencil extendido:} Necesita información de celdas adicionales.
    \item \textbf{Reducción local a primer orden:} El limitador puede activarse excesivamente.
\end{itemize}

%========================================
\subsection{Configuración del caso OpenFOAM}
%========================================

\subsubsection{Condiciones iniciales}

\begin{table}[h!]
\centering
\caption{Condiciones iniciales del problema de Sod}
\label{tab:sod_ic}
\begin{tabular}{lccc}
\hline
\textbf{Región} & \textbf{Densidad} $\rho$ [kg/m$^3$] & \textbf{Presión} $p$ [Pa] & \textbf{Velocidad} $u$ [m/s] \\
\hline
Izquierda ($x < 0$) & 1.0 & 100\,000 & 0 \\
Derecha ($x > 0$) & 0.125 & 10\,000 & 0 \\
\hline
\end{tabular}
\end{table}

\subsubsection{Parámetros numéricos}

\begin{itemize}
    \item \textbf{Solver:} \texttt{foamRun} con modelo de fluido compresible.
    \item \textbf{Dominio:} $x \in [-5, 5]$ m, con 1000 celdas uniformes.
    \item \textbf{Gas:} Aire ideal con $\gamma = 1.4$.
    \item \textbf{Tiempo de simulación:} 0.15 s con CFL $< 0.5$.
\end{itemize}

\subsubsection{Configuración de esquemas}

\textbf{Caso bajo orden (Upwind):}
\begin{lstlisting}[language=C++,basicstyle=\small\ttfamily]
divSchemes
{
    div(phi,U)   Gauss upwind;
    div(phid,p)  Gauss upwind;
    div(phi,e)   Gauss upwind;
}
\end{lstlisting}

\textbf{Caso alto orden (vanAlbada):}
\begin{lstlisting}[language=C++,basicstyle=\small\ttfamily]
divSchemes
{
    div(phi,U)   Gauss vanAlbada;
    div(phid,p)  Gauss vanAlbada;
    div(phi,e)   Gauss vanAlbada;
}
\end{lstlisting}

%========================================
\subsection{Resultados: Validación con solución analítica ($t = 0.1$ s)}
\label{subsec:validacion}
%========================================

La Fig.~\ref{fig:shock_validacion} presenta la comparación entre las soluciones numéricas y la solución analítica proporcionada en $t = 0.1$ s.

\begin{figure}[h!]
    \centering
    \includegraphics[width=\textwidth]{Ejercicio4/shocktube_validacion_t01.png}
    \caption{Validación del problema de Sod en $t = 0.1$ s: comparación de los perfiles de densidad, presión y velocidad entre los esquemas numéricos (upwind y vanAlbada) y la solución analítica.}
    \label{fig:shock_validacion}
\end{figure}

Se observa que los esquemas numéricos presentan elevada difusión numérica en comparación con la solución analítica. La estructura de ondas características (rarefacción, discontinuidad de contacto y choque) queda suavizada debido a la resolución de malla y la naturaleza de los esquemas empleados. Esta diferencia es más pronunciada en las regiones de discontinuidad.

\subsubsection{Análisis cuantitativo de errores}

La Fig.~\ref{fig:shock_errores} presenta los errores relativos L2 de cada esquema respecto a la solución analítica.

\begin{figure}[h!]
    \centering
    \includegraphics[width=0.75\textwidth]{Ejercicio4/shocktube_errores_validacion.png}
    \caption{Errores relativos de validación respecto a la solución analítica en $t = 0.1$ s. El esquema vanAlbada reduce los errores en todas las variables.}
    \label{fig:shock_errores}
\end{figure}

Los errores observados son elevados (superiores al 90\% en algunas variables), lo que indica que la malla utilizada no es suficiente para capturar las discontinuidades con precisión. Ambos esquemas muestran comportamiento similar, sugiriendo que la resolución espacial es el factor limitante más que la selección del esquema numérico.

\subsubsection{Detalle de las discontinuidades}

La Fig.~\ref{fig:shock_detalle} presenta ampliaciones de las tres regiones características, permitiendo analizar el comportamiento de cada esquema cerca de las discontinuidades.

\begin{figure}[h!]
    \centering
    \includegraphics[width=\textwidth]{Ejercicio4/shocktube_detalle_discontinuidades_validacion.png}
    \caption{Detalle de las tres discontinuidades en $t = 0.1$ s: onda de rarefacción (izquierda), discontinuidad de contacto (centro) y onda de choque (derecha).}
    \label{fig:shock_detalle}
\end{figure}

\textbf{Análisis por región:}
\begin{itemize}
    \item \textbf{Onda de rarefacción:} La región de expansión presenta los valores más difusos, debido a la extensión espacial del abanico de rarefacción combinada con la difusión numérica.

    \item \textbf{Discontinuidad de contacto:} La discontinuidad de contacto, que teóricamente es un salto abrupto en densidad (pero no en presión), aparece completamente suavizada en ambos esquemas.

    \item \textbf{Onda de choque:} El frente de choque, que debería ser una discontinuidad nítida, aparece extendido sobre múltiples celdas en la simulación numérica.
\end{itemize}

%========================================
\subsection{Resultados: Comparación de esquemas ($t = 0.15$ s)}
\label{subsec:comparacion}
%========================================

La Fig.~\ref{fig:shock_comparacion} presenta la comparación directa entre los esquemas mediante perfiles horizontales en $t = 0.15$ s.

\begin{figure}[h!]
    \centering
    \includegraphics[width=\textwidth]{Ejercicio4/shocktube_comparacion_t015.png}
    \caption{Comparación de esquemas numéricos en $t = 0.15$ s: perfiles horizontales de densidad, presión y velocidad para esquemas de alto orden (vanAlbada, línea azul continua) y bajo orden (upwind, línea roja discontinua).}
    \label{fig:shock_comparacion}
\end{figure}

\subsubsection{Análisis comparativo detallado}

\textbf{Efecto de la difusión numérica:}

La diferencia fundamental entre ambos esquemas radica en la cantidad de difusión numérica que introducen:

\begin{itemize}
    \item \textbf{Esquema upwind (bajo orden):} Introduce un término de difusión artificial proporcional a $\frac{1}{2}|u|\Delta x$. Esto produce un ensanchamiento significativo (\textit{smearing}) de las discontinuidades, extendiéndolas sobre 8--12 celdas computacionales.

    \item \textbf{Esquema vanAlbada (alto orden):} Reduce la difusión numérica en regiones suaves mediante reconstrucción de segundo orden, pero activa el limitador cerca de discontinuidades para mantener la estabilidad. Las discontinuidades se resuelven en 3--5 celdas.
\end{itemize}

\textbf{Comportamiento por región:}

\begin{enumerate}
    \item \textbf{Onda de rarefacción:} Ambos esquemas capturan adecuadamente la pendiente del abanico de expansión. El esquema de alto orden presenta menor difusión en los extremos del abanico (cabeza y cola de la rarefacción).

    \item \textbf{Discontinuidad de contacto:} Esta es la región donde la diferencia entre esquemas es más evidente. El esquema upwind extiende la discontinuidad de densidad sobre múltiples celdas, mientras que vanAlbada la mantiene más compacta.

    \item \textbf{Onda de choque:} El frente de choque presenta menor extensión con el esquema de alto orden. Sin embargo, puede aparecer un ligero \textit{overshoot} (sobrepaso) en la presión inmediatamente después del choque debido a la naturaleza del limitador.
\end{enumerate}

\subsubsection{Tabla resumen de esquemas}

La Tabla~\ref{tab:esquemas_resumen} resume las características de cada configuración.

\begin{table}[h!]
\centering
\caption{Resumen comparativo de los esquemas numéricos empleados}
\label{tab:esquemas_resumen}
\begin{tabular}{lcc}
\hline
\textbf{Característica} & \textbf{Bajo orden} & \textbf{Alto orden} \\
\hline
Esquema temporal & Euler (1\textsuperscript{er} orden) & Backward (2\textsuperscript{do} orden) \\
Esquema convectivo & Gauss upwind & Gauss vanAlbada \\
Orden de precisión espacial & $O(\Delta x)$ & $O(\Delta x^2)$ en regiones suaves \\
Difusión numérica & Alta & Baja \\
Estabilidad & Incondicional & Condicional (requiere limitador) \\
Coste computacional & Bajo & Medio \\
Resolución de discontinuidades & 8--12 celdas & 3--5 celdas \\
\hline
\end{tabular}
\end{table}

%========================================
\subsection{Evolución temporal}
%========================================

La evolución temporal del problema de Sod permite visualizar la propagación de las ondas características. Se ha generado una animación (GIF) que muestra la evolución de los campos de densidad, presión, temperatura y velocidad desde $t = 0$ hasta $t = 0.15$ s, disponible en el directorio de figuras del repositorio.

Las principales observaciones de la evolución temporal son:

\begin{enumerate}
    \item \textbf{Establecimiento inicial ($t < 0.01$ s):} Las discontinuidades comienzan a propagarse desde la posición del diafragma ($x = 0$). El solver numérico resuelve gradualmente la condición inicial discontinua.

    \item \textbf{Propagación intermedia ($0.01 < t < 0.10$ s):} Las tres ondas se separan claramente:
    \begin{itemize}
        \item La onda de rarefacción avanza hacia la izquierda a velocidad $u - a \approx -374$ m/s.
        \item La discontinuidad de contacto se mueve hacia la derecha a velocidad $u^* \approx 92.7$ m/s.
        \item El choque avanza hacia la derecha a velocidad aproximada de 200 m/s.
    \end{itemize}

    \item \textbf{Tiempo final ($t = 0.15$ s):} Las ondas se aproximan a los extremos del dominio. La onda de rarefacción alcanza el límite izquierdo mientras que el choque continúa propagándose hacia el extremo derecho.
\end{enumerate}

La comparación visual entre los esquemas de alto y bajo orden muestra que la difusión numérica del esquema upwind produce un suavizado progresivo de todas las discontinuidades conforme avanza el tiempo, mientras que el esquema vanAlbada mantiene mejor definidas las estructuras a lo largo de toda la simulación.

%========================================
\section{Conclusiones}
%========================================

El análisis de los esquemas numéricos mediante la implementación FVM en MATLAB y la simulación del tubo de choque de Sod en OpenFOAM ha permitido extraer las siguientes conclusiones:

\textbf{Sección~\ref{sec:fvm_matlab} -- Método de Volúmenes Finitos:}
\begin{enumerate}
    \item La implementación FVM con esquema \textit{upwind} reproduce correctamente la solución analítica de la ecuación de convección-difusión.
    \item El número de Péclet determina el carácter del problema: para $Pe \gg 1$, la convección domina y se requiere mayor resolución de malla.
    \item La difusión numérica del esquema \textit{upwind} es proporcional a $\Delta x$, confirmando su convergencia de primer orden.
\end{enumerate}

\textbf{Sección~\ref{sec:shocktube} -- Tubo de Choque de Sod:}
\begin{enumerate}
    \item El tubo de choque de Sod constituye un \textit{benchmark} fundamental para validar solvers compresibles, conteniendo las tres estructuras características: rarefacción, discontinuidad de contacto y onda de choque.

    \item La comparación con la solución analítica revela elevada difusión numérica en ambos esquemas, indicando que la resolución de malla es insuficiente para capturar las discontinuidades con precisión.

    \item Los esquemas \textit{upwind} y \textit{vanAlbada} producen resultados similares cuando la malla es gruesa, sugiriendo que el refinamiento de malla es crítico para problemas con discontinuidades.

    \item Para aplicaciones con ondas de choque, se requiere una malla significativamente más fina o técnicas de refinamiento adaptativo para reducir la difusión numérica inherente a los métodos de volúmenes finitos.
\end{enumerate}

% Fin del capítulo 4
