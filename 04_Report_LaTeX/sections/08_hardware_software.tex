%========================================
% CAPITULO 8: HARDWARE Y SOFTWARE
%========================================
\chapter{Hardware y Software}
\label{chap:hardware_software}

\section{Especificaciones del hardware}

Los calculos y simulaciones presentados en esta memoria se han realizado en el siguiente equipo:

\begin{itemize}
    \item \textbf{Equipo:} MacBook Pro (2020)
    \item \textbf{Procesador:} Apple M1 (arquitectura ARM64)
    \item \textbf{Memoria RAM:} 16 GB
    \item \textbf{Sistema operativo:} macOS Sonoma
\end{itemize}

\section{Software utilizado}

\subsection{MATLAB}

\begin{itemize}
    \item \textbf{Version:} MATLAB R2025a
    \item \textbf{Uso:} Implementacion de metodos numericos (Ejercicios 1-3), post-procesamiento de datos de OpenFOAM y generacion de graficas.
\end{itemize}

\subsection{OpenFOAM}

\begin{itemize}
    \item \textbf{Version:} OpenFOAM 13
    \item \textbf{Imagen Docker:} \texttt{microfluidica/openfoam:13} (ARM64)
    \item \textbf{Uso:} Simulaciones CFD mediante el metodo de volumenes finitos (Ejercicios 4-7).
\end{itemize}

\subsection{ParaView}

\begin{itemize}
    \item \textbf{Version:} ParaView 5.12
    \item \textbf{Uso:} Visualizacion y post-procesamiento de resultados de OpenFOAM.
\end{itemize}

\subsection{XFLR5}

\begin{itemize}
    \item \textbf{Version:} XFLR5 v6.61
    \item \textbf{Uso:} Validacion de resultados del metodo de paneles (Ejercicio 2).
\end{itemize}

\subsection{LaTeX}

\begin{itemize}
    \item \textbf{Distribucion:} MacTeX 2024
    \item \textbf{Uso:} Redaccion de la presente memoria.
\end{itemize}

\subsection{Control de versiones}

\begin{itemize}
    \item \textbf{Git:} Control de versiones del codigo y documentacion.
    \item \textbf{GitHub:} Repositorio remoto para backup y colaboracion.
    \item \textbf{VS Code:} Editor de codigo con extension de GitHub Copilot.
\end{itemize}

\section{Ejecucion de OpenFOAM en Docker sobre macOS ARM}

Dado que OpenFOAM no dispone de una version nativa para macOS con procesadores Apple Silicon (ARM64), se ha utilizado Docker para ejecutar las simulaciones. A continuacion se describe el procedimiento seguido:

\subsection{Configuracion del entorno Docker}

La imagen Docker utilizada es \texttt{microfluidica/openfoam:13}, que esta compilada para arquitectura ARM64 y es compatible con procesadores Apple M1/M2/M3.

\subsection{Comandos de ejecucion}

Para ejecutar un caso de OpenFOAM, se utiliza el siguiente esquema de comandos:

\begin{verbatim}
# Ajustar permisos del caso
sudo chown -R 1000:1000 /ruta/al/caso

# Ejecutar caso completo
docker run --rm -u 1000:1000 \
   -v "$(pwd)":/home/openfoam/work \
   microfluidica/openfoam:13 \
   bash -lc "cd /home/openfoam/work/caso && ./Allrun"

# Exportar VTK para visualizacion
docker run --rm -u 1000:1000 \
   -v "$(pwd)":/home/openfoam/work \
   microfluidica/openfoam:13 \
   bash -lc "cd /home/openfoam/work/caso && foamToVTK -latestTime"
\end{verbatim}

\subsection{Notas importantes}

\begin{itemize}
    \item El flag \texttt{-u 1000:1000} es necesario para evitar problemas de permisos con las macros \texttt{\#calc} de OpenFOAM.
    \item Los archivos VTK exportados se visualizan posteriormente con ParaView instalado de forma nativa en macOS.
    \item El post-procesamiento numerico (graficas de residuos, coeficientes, etc.) se realiza con scripts de MATLAB o Python ejecutados localmente.
\end{itemize}
