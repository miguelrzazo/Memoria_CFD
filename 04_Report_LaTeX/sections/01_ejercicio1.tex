%========================================
% CAPITULO 1: METODO DE PANELES HESS-SMITH
%========================================
\chapter{Aerodinámica de Perfiles: Método de Paneles Hess-Smith}
\label{chap:ejercicio1}

%========================================
\section{Introducción}
%========================================

El análisis aerodinámico de perfiles bidimensionales constituye el primer paso fundamental en el diseño de superficies sustentadoras. Antes del desarrollo de la dinámica de fluidos computacional (CFD), los métodos de paneles representaban el estado del arte para el cálculo de flujos potenciales alrededor de geometrías arbitrarias~\cite{katz2001low}.

El método de Hess-Smith, desarrollado en la década de 1960, combina distribuciones de manantiales (fuentes) y torbellinos sobre la superficie del cuerpo para resolver el problema del flujo potencial bidimensional. A diferencia de otros métodos de paneles que utilizan solo una singularidad, la combinación de ambas permite satisfacer simultáneamente la condición de no penetración en cada panel y la condición de Kutta en el borde de salida, garantizando una solución físicamente realista con circulación finita.

Este método presenta las siguientes características fundamentales~\cite{hess1967calculation,anderson2010fundamentals}:
\begin{itemize}
    \item \textbf{Flujo potencial:} Se asume fluido incompresible, no viscoso e irrotacional, lo que permite la aplicación del principio de superposición de soluciones elementales.

    \item \textbf{Discretización de superficie:} El perfil se divide en $N$ paneles rectos, sobre los cuales se distribuyen las singularidades con intensidad constante.

    \item \textbf{Condición de Kutta:} Se impone que el flujo abandone suavemente el borde de salida, lo que determina la circulación total y, por tanto, la sustentación.

    \item \textbf{Validez:} Los resultados son aplicables en régimen subsónico con flujo adherido, donde los efectos viscosos son despreciables excepto en una capa límite delgada.
\end{itemize}

%========================================
\section{Fundamento Teórico}
%========================================

\subsection{Ecuaciones de gobierno}

Para flujo potencial incompresible e irrotacional, el campo de velocidades deriva de un potencial de velocidad $\phi$ que satisface la ecuación de Laplace:
\begin{equation}
    \nabla^2 \phi = 0
    \label{eq:laplace}
\end{equation}

La linealidad de esta ecuación permite construir la solución mediante superposición de singularidades elementales:
\begin{equation}
    \phi = \phi_\infty + \sum_{j=1}^{N} \phi_{q_j} + \phi_\gamma
\end{equation}
donde $\phi_\infty$ es el potencial de la corriente uniforme, $\phi_{q_j}$ es la contribución de los manantiales en cada panel, y $\phi_\gamma$ es la contribución de los torbellinos.

\subsection{Distribución de singularidades}

En el método de Hess-Smith, cada panel $j$ tiene:
\begin{itemize}
    \item Una distribución de \textbf{manantiales} de intensidad $q_j$ (variable para cada panel)
    \item Una distribución de \textbf{torbellinos} de intensidad $\gamma$ (constante para todos los paneles)
\end{itemize}

La velocidad inducida por un manantial distribuido sobre un panel de longitud $l_j$ en un punto $P$ ubicado a distancias $r_1$ y $r_2$ de los extremos del panel es:
\begin{equation}
    u_q = \frac{q_j}{2\pi} \ln\left(\frac{r_2}{r_1}\right), \quad
    w_q = \frac{q_j}{2\pi} (\theta_2 - \theta_1)
\end{equation}

Análogamente, para el torbellino:
\begin{equation}
    u_\gamma = -\frac{\gamma}{2\pi} (\theta_2 - \theta_1), \quad
    w_\gamma = \frac{\gamma}{2\pi} \ln\left(\frac{r_2}{r_1}\right)
\end{equation}

\subsection{Condiciones de contorno}

Se imponen dos tipos de condiciones:

\textbf{Condición de no penetración:} La componente normal de la velocidad debe ser nula en cada punto de control (centro del panel):
\begin{equation}
    \vec{V} \cdot \vec{n}_i = 0 \quad \forall i = 1, \ldots, N
    \label{eq:no_penetracion}
\end{equation}

Esta condición genera $N$ ecuaciones lineales con $N+1$ incógnitas ($q_1, \ldots, q_N, \gamma$).

\textbf{Condición de Kutta:} Para cerrar el sistema y garantizar que el flujo abandone suavemente el borde de salida, se impone que las velocidades tangenciales en los paneles adyacentes al borde de salida sean iguales en magnitud pero opuestas en signo:
\begin{equation}
    V_{t,\text{sup}} + V_{t,\text{inf}} = 0
    \label{eq:kutta}
\end{equation}

\subsection{Sistema de ecuaciones}

El sistema resultante tiene la forma matricial:
\begin{equation}
    [A]_{(N+1) \times (N+1)} \begin{Bmatrix} q_1 \\ \vdots \\ q_N \\ \gamma \end{Bmatrix} = \begin{Bmatrix} b_1 \\ \vdots \\ b_N \\ b_{N+1} \end{Bmatrix}
\end{equation}
donde los coeficientes $A_{ij}$ representan la influencia del panel $j$ sobre el punto de control $i$, y $b_i = -\vec{V}_\infty \cdot \vec{n}_i$ para las ecuaciones de no penetración.

\subsection{Coeficientes aerodinámicos}

Una vez resuelto el sistema, se calculan los coeficientes aerodinámicos:

\textbf{Coeficiente de presión:}
\begin{equation}
    C_p = 1 - \left(\frac{V_t}{U_\infty}\right)^2
    \label{eq:cp}
\end{equation}
donde $V_t$ es la velocidad tangencial en cada panel.

\textbf{Coeficiente de sustentación:} Mediante el teorema de Kutta-Joukowski:
\begin{equation}
    C_L = \frac{2\Gamma}{U_\infty \, c}
    \label{eq:cl}
\end{equation}
donde $\Gamma$ es la circulación total alrededor del perfil.

\textbf{Coeficiente de momento:} Respecto al borde de ataque:
\begin{equation}
    C_{M,O} = \sum_{i=1}^{N} C_{p,i} \cdot \frac{x_i}{c} \cdot \frac{l_i}{c}
    \label{eq:cm_o}
\end{equation}

Respecto al cuarto de cuerda:
\begin{equation}
    C_{M,c/4} = C_{M,O} + \frac{C_L}{4}
    \label{eq:cm_c4}
\end{equation}

%========================================
\section{Implementación y Configuración}
%========================================

\subsection{Geometría del perfil}

Las coordenadas del perfil fueron proporcionadas en el enunciado del ejercicio, con 17 puntos definidos tanto para el extradós como para el intradós. El perfil presenta las siguientes características geométricas:

\begin{itemize}
    \item Cuerda: $c = 1.0$ (normalizada)
    \item Espesor máximo: aproximadamente $12\%$ de la cuerda
    \item Curvatura positiva: genera sustentación a ángulo de ataque nulo
\end{itemize}

\subsection{Discretización en paneles}

Para cumplir el requisito de al menos 40 paneles, se implementó una interpolación con distribución coseno que proporciona mayor densidad de paneles cerca del borde de ataque y de salida, donde los gradientes de presión son más pronunciados:

\begin{equation}
    x_i = \frac{c}{2}\left(1 - \cos\left(\frac{i\pi}{N_{\text{pts}}}\right)\right), \quad i = 0, 1, \ldots, N_{\text{pts}}
\end{equation}

La configuración final utilizada es:
\begin{itemize}
    \item 41 puntos por lado (extradós e intradós)
    \item 80 paneles totales (40 en extradós + 40 en intradós)
    \item Distribución coseno para refinamiento en bordes
\end{itemize}

\subsection{Parámetros de simulación}

\begin{itemize}
    \item Velocidad de corriente libre: $U_\infty = 20$ m/s
    \item Rango de ángulos de ataque: $\alpha = -10^{\circ}$ a $25^{\circ}$ con incremento de $1^{\circ}$
    \item Total de casos analizados: 36 ángulos de ataque
\end{itemize}

%========================================
\section{Resultados y Análisis}
%========================================

\subsection{Discretización del perfil y validación geométrica}

La Fig.~\ref{fig:perfil_paneles_ej1} muestra el perfil aerodinámico discretizado en 80 paneles, con los puntos de control (centros de panel) marcados.

\begin{figure}[h!]
    \centering
    \includegraphics[width=0.90\textwidth]{../01_Matlab_Exercises/figures/Ejercicio1/perfil_paneles.png}
    \caption{Perfil aerodinámico discretizado con 80 paneles y puntos de control. La distribución coseno proporciona mayor resolución en el borde de ataque y de salida.}
    \label{fig:perfil_paneles_ej1}
\end{figure}

La discretización cumple ampliamente el requisito mínimo de 40 paneles (se utilizaron 80 paneles totales, 40 por superficie). La distribución coseno concentra paneles en las zonas críticas donde los gradientes de presión son más pronunciados, optimizando la precisión del método para un número dado de paneles.

\subsection{Distribución del coeficiente de presión $C_p$}

La Fig.~\ref{fig:Cp_distribucion_ej1} presenta la distribución del coeficiente de presión $C_p$ sobre el perfil para varios ángulos de ataque representativos ($\alpha = 0^\circ, 5^\circ, 10^\circ, 15^\circ$).

\begin{figure}[h!]
    \centering
    \includegraphics[width=0.95\textwidth]{../01_Matlab_Exercises/figures/Ejercicio1/Cp_distribucion.png}
    \caption{Distribución del coeficiente de presión $C_p$ sobre el perfil para diferentes ángulos de ataque. Líneas sólidas: extradós; líneas discontinuas: intradós. El eje $C_p$ está invertido siguiendo la convención aeronáutica.}
    \label{fig:Cp_distribucion_ej1}
\end{figure}

\textbf{Análisis de la distribución de $C_p$:}

\begin{enumerate}
    \item \textbf{Punto de estancamiento:} En el borde de ataque, $C_p \approx 1$ donde la velocidad es nula, confirmando la correcta implementación de la condición de no penetración. Este valor coincide con la predicción del teorema de Bernoulli para flujo potencial.

    \item \textbf{Succión en extradós:} Los valores negativos de $C_p$ indican velocidades superiores a $U_\infty$, generando la succión que produce sustentación. El pico de succión se localiza cerca del borde de ataque ($x/c \approx 0.1-0.2$) y se intensifica progresivamente con el ángulo de ataque. Para $\alpha = 15^\circ$, se observa un mínimo de aproximadamente $C_p \approx -4.5$.

    \item \textbf{Presión en intradós:} El intradós presenta presiones ligeramente superiores a la presión de corriente libre ($C_p > 0$) en la región del borde de ataque, especialmente para ángulos de ataque positivos. Esta sobrepresión contribuye a la generación de sustentación junto con la succión del extradós.

    \item \textbf{Efecto del ángulo de ataque:} Al aumentar $\alpha$, la diferencia de presión entre extradós e intradós se incrementa proporcionalmente, lo que explica el aumento lineal de la sustentación observado en los resultados globales. Para $\alpha = 0^\circ$, ya existe una diferencia de presión debido a la curvatura del perfil.

    \item \textbf{Recuperación de presión:} Hacia el borde de salida ($x/c > 0.6$), las presiones se recuperan gradualmente hacia valores cercanos a la presión de corriente libre. La recuperación es más suave que en perfiles simétricos debido a la curvatura del perfil, con valores ligeramente superiores en el intradós.
\end{enumerate}

Las distribuciones obtenidas muestran un comportamiento físico coherente con la teoría aerodinámica, validando la correcta implementación del método de Hess-Smith.

\subsection{Coeficiente de sustentación $C_L$ vs ángulo de ataque $\alpha$}

La Fig.~\ref{fig:CL_vs_alpha_ej1} presenta la variación del coeficiente de sustentación con el ángulo de ataque.

\begin{figure}[h!]
    \centering
    \includegraphics[width=0.85\textwidth]{../01_Matlab_Exercises/figures/Ejercicio1/CL_vs_alpha.png}
    \caption{Coeficiente de sustentación $C_L$ en función del ángulo de ataque $\alpha$.}
    \label{fig:CL_vs_alpha_ej1}
\end{figure}

\textbf{Análisis de la curva $C_L$ vs $\alpha$:}

\begin{enumerate}
    \item \textbf{Linealidad:} La curva presenta linealidad perfecta en todo el rango analizado ($-10^{\circ}$ a $25^{\circ}$), característica fundamental del flujo potencial que no captura el desprendimiento de la capa límite. En la realidad, se esperaría una pérdida de linealidad y entrada en pérdida para ángulos superiores a $\alpha_{\text{stall}} \approx 12-15^\circ$.

    \item \textbf{Sustentación a $\alpha = 0^\circ$:} Se obtiene $C_L(0^{\circ}) \approx 0.134$, valor positivo que confirma el efecto de la curvatura del perfil. Un perfil con curvatura positiva genera sustentación incluso a ángulo de ataque nulo.

    \item \textbf{Ángulo de sustentación nula:} De la regresión lineal se obtiene $\alpha_{L_0} \approx -4.35^\circ$, valor negativo coherente con un perfil de curvatura positiva.
\end{enumerate}

\subsection{Coeficientes de momento $C_M$}

Las Figs.~\ref{fig:CM0_vs_alpha_ej1} y~\ref{fig:CMc4_vs_alpha_ej1} muestran la variación de los coeficientes de momento respecto al borde de ataque ($C_{M,O}$) y respecto al cuarto de cuerda ($C_{M,c/4}$).

\begin{figure}[h!]
    \centering
    \begin{minipage}{0.48\textwidth}
        \centering
        \includegraphics[width=\textwidth]{../01_Matlab_Exercises/figures/Ejercicio1/CM0_vs_alpha.png}
        \caption{Coeficiente de momento respecto al borde de ataque $C_{M,O}$.}
        \label{fig:CM0_vs_alpha_ej1}
    \end{minipage}
    \hfill
    \begin{minipage}{0.48\textwidth}
        \centering
        \includegraphics[width=\textwidth]{../01_Matlab_Exercises/figures/Ejercicio1/CMc4_vs_alpha.png}
        \caption{Coeficiente de momento respecto al cuarto de cuerda $C_{M,c/4}$.}
        \label{fig:CMc4_vs_alpha_ej1}
    \end{minipage}
\end{figure}

\textbf{Análisis de los coeficientes de momento:}

\begin{enumerate}
    \item \textbf{Momento respecto al borde de ataque ($C_{M,O}$):} Presenta valores \textbf{negativos} en todo el rango analizado, con $C_{M,O} \approx -0.046$ a $\alpha = 0^\circ$. Este signo negativo es físicamente correcto para un perfil con curvatura positiva: la distribución de presiones genera un momento de picado (nariz hacia abajo) respecto al borde de ataque.

    \item \textbf{Momento respecto a $c/4$ ($C_{M,c/4}$):} Presenta valores negativos aproximadamente constantes alrededor de $C_{M,c/4} \approx -0.09$, confirmando que el centro aerodinámico se encuentra próximo al cuarto de cuerda, como predice la teoría de perfiles delgados. La variación con $\alpha$ es mínima ($\sigma \approx 0.005$).

    \item \textbf{Centro aerodinámico:} La variación limitada de $C_{M,c/4}$ indica que el cuarto de cuerda es una buena aproximación del centro aerodinámico para este perfil. En el centro aerodinámico, el momento es independiente del ángulo de ataque.

    \item \textbf{Coherencia física:} Los valores negativos de $C_{M,O}$ y $C_{M,c/4}$ son coherentes con perfiles de curvatura positiva. Un perfil con camber genera un momento de picado inherente, independiente del ángulo de ataque.
\end{enumerate}

\subsection{Pendiente de sustentación y parámetros característicos}

La Tabla~\ref{tab:resultados_ej1} resume los coeficientes aerodinámicos calculados para ángulos de ataque representativos.

\begin{table}[h!]
    \centering
    \caption{Coeficientes aerodinámicos calculados mediante el método de Hess-Smith.}
    \label{tab:resultados_ej1}
    \begin{tabular}{cccc}
        \hline
        $\alpha$ [deg] & $C_L$ & $C_{M,O}$ & $C_{M,c/4}$ \\
        \hline
        $-10$ & $-0.180$ & $-0.042$ & $-0.076$ \\
        $-5$ & $-0.023$ & $-0.044$ & $-0.082$ \\
        $0$ & $0.134$ & $-0.046$ & $-0.086$ \\
        $5$ & $0.290$ & $-0.047$ & $-0.090$ \\
        $10$ & $0.444$ & $-0.047$ & $-0.091$ \\
        $15$ & $0.595$ & $-0.047$ & $-0.092$ \\
        $20$ & $0.741$ & $-0.046$ & $-0.090$ \\
        $25$ & $0.881$ & $-0.044$ & $-0.088$ \\
        \hline
    \end{tabular}
\end{table}

%========================================
\section{Conclusiones}
%========================================

La implementación del método de paneles de Hess-Smith para el análisis aerodinámico del perfil bidimensional ha permitido alcanzar los siguientes objetivos:

\begin{enumerate}
    \item \textbf{Cumplir los requisitos del enunciado:} Se utilizaron 80 paneles (40 por lado), superando ampliamente el mínimo de 40 paneles requerido. La distribución coseno optimiza la resolución en las zonas críticas del perfil (bordes de ataque y salida).

    \item \textbf{Calcular la distribución de presiones:} Las distribuciones de $C_p$ obtenidas muestran el comportamiento físico esperado: punto de estancamiento en el borde de ataque con $C_p \approx 1$, succión pronunciada en el extradós proporcional al ángulo de ataque (mínimos de hasta $C_p \approx -4.5$), y recuperación gradual de presión hacia el borde de salida.

    \item \textbf{Obtener coeficientes aerodinámicos:} Los valores de $C_L$ y $C_M$ calculados presentan tendencias coherentes con la teoría aerodinámica:
    \begin{itemize}
        \item $C_L$ varía linealmente con $\alpha$, pasando por $C_L = 0.134$ a $\alpha = 0^\circ$ debido a la curvatura.
        \item $C_{M,O} \approx -0.046$ (negativo), consistente con un perfil de curvatura positiva.
        \item $C_{M,c/4} \approx -0.09$ (aproximadamente constante), confirmando la ubicación del centro aerodinámico.
    \end{itemize}

    \item \textbf{Determinar parámetros característicos:} Se obtuvo un ángulo de sustentación nula $\alpha_{L_0} \approx -4.35^\circ$, valor negativo coherente con la curvatura positiva del perfil. El centro aerodinámico se confirma cercano al cuarto de cuerda.

    \item \textbf{Verificar coherencia física:} Los resultados capturan correctamente las características fundamentales del perfil: sustentación positiva a $\alpha = 0^\circ$ por efecto del camber, momento de picado negativo, y linealidad de $C_L$ vs $\alpha$ característica del régimen potencial.
\end{enumerate}

El código MATLAB desarrollado proporciona una herramienta funcional y eficiente para el análisis preliminar de perfiles aerodinámicos, útil como primera aproximación en fases tempranas de diseño antes de recurrir a simulaciones CFD con modelado completo de efectos viscosos, que son computacionalmente más costosas.
