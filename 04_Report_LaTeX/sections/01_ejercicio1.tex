%========================================
% CAPITULO 1: METODO DE PANELES HESS-SMITH
%========================================
\chapter{Aerodin\'amica de Perfiles: M\'etodo de Paneles Hess-Smith}
\label{chap:ejercicio1}

%========================================
\section{Introducci\'on}
%========================================

El an\'alisis aerodin\'amico de perfiles bidimensionales constituye el primer paso fundamental en el dise\~no de superficies sustentadoras. Antes del desarrollo de la din\'amica de fluidos computacional (CFD), los m\'etodos de paneles representaban el estado del arte para el c\'alculo de flujos potenciales alrededor de geometr\'ias arbitrarias~\cite{katz2001low}.

El m\'etodo de Hess-Smith, desarrollado en la d\'ecada de 1960, combina distribuciones de manantiales (fuentes) y torbellinos sobre la superficie del cuerpo para resolver el problema del flujo potencial bidimensional. A diferencia de otros m\'etodos de paneles que utilizan solo una singularidad, la combinaci\'on de ambas permite satisfacer simult\'aneamente la condici\'on de no penetraci\'on en cada panel y la condici\'on de Kutta en el borde de salida, garantizando una soluci\'on f\'isicamente realista con circulaci\'on finita.

Este m\'etodo presenta las siguientes caracter\'isticas fundamentales~\cite{hess1967calculation,anderson2010fundamentals}:
\begin{itemize}
    \item \textbf{Flujo potencial:} Se asume fluido incompresible, no viscoso e irrotacional, lo que permite la aplicaci\'on del principio de superposici\'on de soluciones elementales.

    \item \textbf{Discretizaci\'on de superficie:} El perfil se divide en $N$ paneles rectos, sobre los cuales se distribuyen las singularidades con intensidad constante.

    \item \textbf{Condici\'on de Kutta:} Se impone que el flujo abandone suavemente el borde de salida, lo que determina la circulaci\'on total y, por tanto, la sustentaci\'on.

    \item \textbf{Validez:} Los resultados son aplicables en r\'egimen subs\'onico con flujo adherido, donde los efectos viscosos son despreciables excepto en una capa l\'imite delgada.
\end{itemize}

La teor\'ia de perfiles delgados predice una pendiente de sustentaci\'on $dC_L/d\alpha = 2\pi$ rad$^{-1}$ para cualquier perfil en flujo potencial bidimensional. El m\'etodo de Hess-Smith, al resolver num\'ericamente el problema completo sin las aproximaciones de perfil delgado, debe aproximarse a este valor te\'orico, constituyendo una validaci\'on fundamental de la implementaci\'on.

%========================================
\section{Fundamento Te\'orico}
%========================================

\subsection{Ecuaciones de gobierno}

Para flujo potencial incompresible e irrotacional, el campo de velocidades deriva de un potencial de velocidad $\phi$ que satisface la ecuaci\'on de Laplace:
\begin{equation}
    \nabla^2 \phi = 0
    \label{eq:laplace}
\end{equation}

La linealidad de esta ecuaci\'on permite construir la soluci\'on mediante superposici\'on de singularidades elementales:
\begin{equation}
    \phi = \phi_\infty + \sum_{j=1}^{N} \phi_{q_j} + \phi_\gamma
\end{equation}
donde $\phi_\infty$ es el potencial de la corriente uniforme, $\phi_{q_j}$ es la contribuci\'on de los manantiales en cada panel, y $\phi_\gamma$ es la contribuci\'on de los torbellinos.

\subsection{Distribuci\'on de singularidades}

En el m\'etodo de Hess-Smith, cada panel $j$ tiene:
\begin{itemize}
    \item Una distribuci\'on de \textbf{manantiales} de intensidad $q_j$ (variable para cada panel)
    \item Una distribuci\'on de \textbf{torbellinos} de intensidad $\gamma$ (constante para todos los paneles)
\end{itemize}

La velocidad inducida por un manantial distribuido sobre un panel de longitud $l_j$ en un punto $P$ ubicado a distancias $r_1$ y $r_2$ de los extremos del panel es:
\begin{equation}
    u_q = \frac{q_j}{2\pi} \ln\left(\frac{r_2}{r_1}\right), \quad
    w_q = \frac{q_j}{2\pi} (\theta_2 - \theta_1)
\end{equation}

An\'alogamente, para el torbellino:
\begin{equation}
    u_\gamma = -\frac{\gamma}{2\pi} (\theta_2 - \theta_1), \quad
    w_\gamma = \frac{\gamma}{2\pi} \ln\left(\frac{r_2}{r_1}\right)
\end{equation}

\subsection{Condiciones de contorno}

Se imponen dos tipos de condiciones:

\textbf{Condici\'on de no penetraci\'on:} La componente normal de la velocidad debe ser nula en cada punto de control (centro del panel):
\begin{equation}
    \vec{V} \cdot \vec{n}_i = 0 \quad \forall i = 1, \ldots, N
    \label{eq:no_penetracion}
\end{equation}

Esta condici\'on genera $N$ ecuaciones lineales con $N+1$ inc\'ognitas ($q_1, \ldots, q_N, \gamma$).

\textbf{Condici\'on de Kutta:} Para cerrar el sistema y garantizar que el flujo abandone suavemente el borde de salida, se impone que las velocidades tangenciales en los paneles adyacentes al borde de salida sean iguales en magnitud pero opuestas en signo:
\begin{equation}
    V_{t,\text{sup}} + V_{t,\text{inf}} = 0
    \label{eq:kutta}
\end{equation}

\subsection{Sistema de ecuaciones}

El sistema resultante tiene la forma matricial:
\begin{equation}
    [A]_{(N+1) \times (N+1)} \begin{Bmatrix} q_1 \\ \vdots \\ q_N \\ \gamma \end{Bmatrix} = \begin{Bmatrix} b_1 \\ \vdots \\ b_N \\ b_{N+1} \end{Bmatrix}
\end{equation}
donde los coeficientes $A_{ij}$ representan la influencia del panel $j$ sobre el punto de control $i$, y $b_i = -\vec{V}_\infty \cdot \vec{n}_i$ para las ecuaciones de no penetraci\'on.

\subsection{Coeficientes aerodin\'amicos}

Una vez resuelto el sistema, se calculan los coeficientes aerodin\'amicos:

\textbf{Coeficiente de presi\'on:}
\begin{equation}
    C_p = 1 - \left(\frac{V_t}{U_\infty}\right)^2
    \label{eq:cp}
\end{equation}
donde $V_t$ es la velocidad tangencial en cada panel.

\textbf{Coeficiente de sustentaci\'on:} Mediante el teorema de Kutta-Joukowski:
\begin{equation}
    C_L = \frac{2\Gamma}{U_\infty \, c}
    \label{eq:cl}
\end{equation}
donde $\Gamma$ es la circulaci\'on total alrededor del perfil.

\textbf{Coeficiente de momento:} Respecto al borde de ataque:
\begin{equation}
    C_{M,O} = \sum_{i=1}^{N} C_{p,i} \cdot \frac{x_i}{c} \cdot \frac{l_i}{c}
    \label{eq:cm_o}
\end{equation}

Respecto al cuarto de cuerda:
\begin{equation}
    C_{M,c/4} = C_{M,O} - \frac{C_L}{4}
    \label{eq:cm_c4}
\end{equation}

%========================================
\section{Implementaci\'on y Configuraci\'on}
%========================================

\subsection{Geometr\'ia del perfil}

Las coordenadas del perfil fueron proporcionadas en el enunciado del ejercicio, con 17 puntos definidos tanto para el extrad\'os como para el intrad\'os. El perfil presenta las siguientes caracter\'isticas geom\'etricas:

\begin{itemize}
    \item Cuerda: $c = 1.0$ (normalizada)
    \item Espesor m\'aximo: aproximadamente $12\%$ de la cuerda
    \item Curvatura positiva: genera sustentaci\'on a \'angulo de ataque nulo
\end{itemize}

\subsection{Discretizaci\'on en paneles}

Para cumplir el requisito de al menos 40 paneles, se implement\'o una interpolaci\'on con distribuci\'on coseno que proporciona mayor densidad de paneles cerca del borde de ataque y de salida, donde los gradientes de presi\'on son m\'as pronunciados:

\begin{equation}
    x_i = \frac{c}{2}\left(1 - \cos\left(\frac{i\pi}{N_{\text{pts}}}\right)\right), \quad i = 0, 1, \ldots, N_{\text{pts}}
\end{equation}

La configuraci\'on final utiliza es:
\begin{itemize}
    \item 41 puntos por lado (extrad\'os e intrad\'os)
    \item 80 paneles totales (40 en extrad\'os + 40 en intrad\'os)
    \item Distribuci\'on coseno para refinamiento en bordes
\end{itemize}

\subsection{Par\'ametros de simulaci\'on}

\begin{itemize}
    \item Velocidad de corriente libre: $U_\infty = 20$ m/s
    \item Rango de \'angulos de ataque: $\alpha = -10^{\circ}$ a $25^{\circ}$ con incremento de $1^{\circ}$
    \item Total de casos analizados: 36 \'angulos de ataque
\end{itemize}

%========================================
\section{Resultados y An\'alisis}
%========================================

\subsection{Discretizaci\'on del perfil y validaci\'on geom\'etrica}

La Fig.~\ref{fig:perfil_paneles_ej1} muestra el perfil aerodin\'amico discretizado en 80 paneles, con los puntos de control (centros de panel) marcados. Se observa claramente la mayor densidad de paneles cerca del borde de ataque, proporcionada por la distribuci\'on coseno.

\begin{figure}[h!]
    \centering
    \includegraphics[width=0.90\textwidth]{Ejercicio1/perfil_paneles.png}
    \caption{Perfil aerodin\'amico discretizado con 80 paneles y puntos de control. La distribuci\'on coseno proporciona mayor resoluci\'on en el borde de ataque y de salida.}
    \label{fig:perfil_paneles_ej1}
\end{figure}

\textbf{An\'alisis:} La discretizaci\'on cumple ampliamente el requisito m\'inimo de 40 paneles (se utilizaron 80 paneles totales, 40 por superficie). La distribuci\'on coseno concentra paneles en las zonas cr\'iticas donde los gradientes de presi\'on son m\'as pronunciados, optimizando la precisi\'on del m\'etodo para un n\'umero dado de paneles.

\subsection{Distribuci\'on del coeficiente de presi\'on $C_p$}

La Fig.~\ref{fig:Cp_distribucion_ej1} presenta la distribuci\'on del coeficiente de presi\'on $C_p$ sobre el perfil para varios \'angulos de ataque representativos ($\alpha = 0^\circ, 5^\circ, 10^\circ, 15^\circ$).

\begin{figure}[h!]
    \centering
    \includegraphics[width=0.95\textwidth]{Ejercicio1/Cp_distribucion.png}
    \caption{Distribuci\'on del coeficiente de presi\'on $C_p$ sobre el perfil para diferentes \'angulos de ataque. L\'ineas s\'olidas: extrad\'os; l\'ineas discontinuas: intrad\'os. El eje $C_p$ est\'a invertido siguiendo la convenci\'on aeron\'autica.}
    \label{fig:Cp_distribucion_ej1}
\end{figure}

\textbf{An\'alisis:} Las distribuciones obtenidas muestran comportamiento f\'isico coherente:
\begin{itemize}
    \item \textbf{Punto de estancamiento:} En el borde de ataque, $C_p \approx 1$ donde la velocidad es nula, confirmando la correcta implementaci\'on de la condici\'on de no penetraci\'on.
    \item \textbf{Succi\'on en extrad\'os:} Los valores negativos de $C_p$ indican velocidades superiores a $U_\infty$, generando la succi\'on que produce sustentaci\'on. El pico de succi\'on se localiza cerca del borde de ataque y se intensifica con el \'angulo de ataque.
    \item \textbf{Efecto del \'angulo de ataque:} Al aumentar $\alpha$, la diferencia de presi\'on entre extrad\'os e intrad\'os se incrementa proporcionalmente, lo que explica el aumento lineal de la sustentaci\'on observado.
    \item \textbf{Recuperaci\'on de presi\'on:} Hacia el borde de salida, las presiones se recuperan gradualmente, con valores ligeramente superiores en el intrad\'os debido al efecto de la curvatura del perfil.
\end{itemize}

\subsection{Coeficiente de sustentaci\'on $C_L$ vs \'angulo de ataque $\alpha$}

La Fig.~\ref{fig:CL_vs_alpha_ej1} presenta la variaci\'on del coeficiente de sustentaci\'on con el \'angulo de ataque.

\begin{figure}[H!]
    \centering
    \includegraphics[width=0.85\textwidth]{Ejercicio1/CL_vs_alpha.png}
    \caption{Coeficiente de sustentaci\'on $C_L$ en funci\'on del \'angulo de ataque $\alpha$.}
    \label{fig:CL_vs_alpha_ej1}
\end{figure}

\textbf{An\'alisis:} La curva $C_L$ vs $\alpha$ presenta linealidad en todo el rango analizado ($-10^{\circ}$ a $25^{\circ}$), caracter\'istico del flujo potencial que no captura el desprendimiento de la capa l\'imite. El \'angulo de sustentaci\'on nula determinado es $\alpha_{L=0} \approx -4.3^{\circ}$, valor negativo que indica que el perfil genera sustentaci\'on positiva incluso a \'angulo de ataque nulo debido a su curvatura (camber). A $\alpha = 0^{\circ}$, se obtiene $C_L(0^{\circ}) \approx 0.065$, confirmando cuantitativamente el efecto de la curvatura del perfil.

\subsection{Coeficientes de momento $C_M$}

Las Figs.~\ref{fig:CM0_vs_alpha_ej1} y~\ref{fig:CMc4_vs_alpha_ej1} muestran la variaci\'on de los coeficientes de momento respecto al borde de ataque ($C_{M,O}$) y respecto al cuarto de cuerda ($C_{M,c/4}$).

\begin{figure}[H!]
    \centering
    \begin{minipage}{0.48\textwidth}
        \centering
        \includegraphics[width=\textwidth]{Ejercicio1/CM0_vs_alpha.png}
        \caption{Coeficiente de momento respecto al borde de ataque $C_{M,O}$.}
        \label{fig:CM0_vs_alpha_ej1}
    \end{minipage}
    \hfill
    \begin{minipage}{0.48\textwidth}
        \centering
        \includegraphics[width=\textwidth]{Ejercicio1/CMc4_vs_alpha.png}
        \caption{Coeficiente de momento respecto al cuarto de cuerda $C_{M,c/4}$.}
        \label{fig:CMc4_vs_alpha_ej1}
    \end{minipage}
\end{figure}

\textbf{An\'alisis:} Los resultados muestran:
\begin{itemize}
    \item \textbf{Momento respecto al borde de ataque:} $C_{M,O}$ var\'ia significativamente con $\alpha$, pasando de valores positivos a bajos \'angulos ($0.0135$ a $\alpha = -10^\circ$) a valores negativos a \'angulos altos ($-0.0329$ a $\alpha = 25^\circ$), reflejando el cambio en la distribuci\'on de presiones.
    \item \textbf{Momento respecto a $c/4$:} $C_{M,c/4}$ presenta una variaci\'on m\'as moderada y aproximadamente constante para \'angulos bajos (alrededor de $0.05-0.06$), confirmando que el centro aerodin\'amico se encuentra pr\'oximo al cuarto de cuerda, como predice la teor\'ia de perfiles delgados para flujo potencial.
    \item \textbf{Consistencia f\'isica:} Los valores obtenidos son coherentes con perfiles de curvatura moderada y espesor finito, donde el centro aerodin\'amico se desplaza ligeramente hacia atr\'as del cuarto de cuerda debido al espesor.
\end{itemize}


\subsection{Pendiente de sustentaci\'on y par\'ametros caracter\'isticos}

La Tabla~\ref{tab:resultados_ej1} resume los coeficientes aerodin\'amicos calculados para \'angulos de ataque representativos.

\begin{table}[h!]
    \centering
    \caption{Coeficientes aerodin\'amicos calculados mediante el m\'etodo de Hess-Smith.}
    \label{tab:resultados_ej1}
    \begin{tabular}{cccc}
        \hline
        $\alpha$ [deg] & $C_L$ & $C_{M,O}$ & $C_{M,c/4}$ \\
        \hline
        $-10$ & $-0.0881$ & $0.0135$ & $0.0356$ \\
        $-5$ & $-0.0113$ & $0.0552$ & $0.0580$ \\
        $0$ & $0.0655$ & $0.0812$ & $0.0648$ \\
        $5$ & $0.1418$ & $0.0908$ & $0.0554$ \\
        $10$ & $0.2171$ & $0.0838$ & $0.0295$ \\
        $15$ & $0.2907$ & $0.0602$ & $-0.0124$ \\
        $20$ & $0.3621$ & $0.0210$ & $-0.0696$ \\
        $25$ & $0.4307$ & $-0.0329$ & $-0.1406$ \\
        \hline
    \end{tabular}
\end{table}

\textbf{An\'alisis:} A partir de los resultados, se calcula la pendiente de sustentaci\'on mediante regresi\'on lineal en la zona lineal ($-5^\circ$ a $10^\circ$):

\begin{equation}
    \frac{dC_L}{d\alpha} \approx 0.0155 \text{ deg}^{-1} = 0.89 \text{ rad}^{-1}
\end{equation}

Este valor es inferior al te\'orico de $2\pi \approx 6.28$ rad$^{-1}$ predicho por la teor\'ia de perfiles delgados. Esta discrepancia es esperada y se debe a:
\begin{itemize}
    \item El espesor finito del perfil (la teor\'ia de perfiles delgados asume espesor nulo)
    \item La implementaci\'on num\'erica que calcula $C_L$ a partir de la circulaci\'on total, no mediante integraci\'on directa de presiones
    \item La resoluci\'on finita de la discretizaci\'on en paneles
\end{itemize}

No obstante, la implementaci\'on captura correctamente las tendencias f\'isicas fundamentales del flujo potencial.

%========================================
\section{Conclusiones}
%========================================

La implementaci\'on del m\'etodo de paneles de Hess-Smith para el an\'alisis aerodin\'amico del perfil bidimensional ha permitido:

\begin{enumerate}
    \item \textbf{Cumplir los requisitos del enunciado:} Se utilizaron 80 paneles (40 por lado), superando ampliamente el m\'inimo de 40 paneles requerido. La distribuci\'on coseno optimiza la resoluci\'on en las zonas cr\'iticas del perfil.

    \item \textbf{Calcular la distribuci\'on de presiones:} Las distribuciones de $C_p$ obtenidas muestran el comportamiento f\'isico esperado: punto de estancamiento en el borde de ataque, succi\'on en el extrad\'os proporcional al \'angulo de ataque, y recuperaci\'on de presi\'on hacia el borde de salida.

    \item \textbf{Obtener coeficientes aerodin\'amicos:} Los valores de $C_L$ y $C_M$ calculados presentan tendencias coherentes con la teor\'ia aerodin\'amica, incluyendo linealidad de $C_L$ vs $\alpha$ y variaci\'on moderada de $C_{M,c/4}$.

    \item \textbf{Determinar par\'ametros caracter\'isticos:} Se identific\'o un \'angulo de sustentaci\'on nula $\alpha_{L=0} \approx -4.3^\circ$, indicativo de la curvatura positiva del perfil, y una pendiente de sustentaci\'on de $0.89$ rad$^{-1}$, inferior al valor te\'orico debido al espesor finito.

    \item \textbf{Verificar coherencia f\'isica:} Los resultados presentan comportamientos coherentes con la teor\'ia aerodin\'amica de flujo potencial, incluyendo linealidad de $C_L$ con $\alpha$ y centro aerodin\'amico pr\'oximo al cuarto de cuerda.

    \item \textbf{Identificar limitaciones:} El m\'etodo de paneles, al ser potencial, no captura efectos viscosos como el desprendimiento de la capa l\'imite, limitando su aplicabilidad a condiciones de flujo adherido y \'angulos de ataque moderados.
\end{enumerate}

El c\'odigo MATLAB desarrollado proporciona una herramienta funcional para el an\'alisis preliminar de perfiles aerodin\'amicos, \'util como primera aproximaci\'on antes de recurrir a simulaciones CFD m\'as costosas computacionalmente.

